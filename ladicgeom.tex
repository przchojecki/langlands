% !TEX program = pdflatexx
\documentclass[11pt]{article}

\usepackage[margin=1in]{geometry}
\usepackage{amsmath,amssymb,amsthm,mathtools}
\usepackage{enumitem}
\usepackage{hyperref}
\usepackage{mathrsfs}
\usepackage{tikz-cd}


\hypersetup{
  colorlinks=true,
  linkcolor=blue,
  citecolor=blue,
  urlcolor=blue
}

% ---------- Theorem environments ----------
\theoremstyle{definition}
\newtheorem{theorem}{Theorem}[section]
\newtheorem{conjecture}[theorem]{Conjecture}
\newtheorem{question}[theorem]{Question}
\newtheorem{definition}[theorem]{Definition}
\newtheorem{axiom}[theorem]{Axiom}

\theoremstyle{remark}
\newtheorem{remark}[theorem]{Remark}
\newtheorem{example}[theorem]{Example}

\theoremstyle{plain}
\newtheorem{proposition}[theorem]{Proposition}
\newtheorem{lemma}[theorem]{Lemma}
\newtheorem{corollary}[theorem]{Corollary}


% ---------- Macros ----------
\newcommand{\Qp}{\mathbb{Q}_p}
\newcommand{\Zl}{\mathbb{Z}_\ell}
\newcommand{\Z}{\mathbb{Z}}
\newcommand{\Q}{\mathbb{Q}}
\newcommand{\Ql}{\mathbb{Q}_\ell}
\newcommand{\OO}{\mathcal{O}}
\newcommand{\cC}{\mathcal{C}}
\newcommand{\cD}{\mathcal{D}}
\newcommand{\cA}{\mathcal{A}}
\newcommand{\cI}{\mathcal{I}}
\newcommand{\cS}{\mathcal{S}}
\newcommand{\cK}{\mathcal{K}}
\newcommand{\cN}{\mathcal{N}}
\newcommand{\cM}{\mathcal{M}}
\newcommand{\ad}{\mathrm{ad}}
\newcommand{\Hecke}{\mathrm{Hecke}}
\newcommand{\Ind}{\mathrm{Ind}}
\newcommand{\Sat}{\mathrm{Sat}}
\newcommand{\Sing}{\mathrm{Sing}}
\newcommand{\End}{\mathrm{End}}
\newcommand{\Nilp}{\mathrm{Nilp}}
\newcommand{\Bun}{\mathrm{Bun}}
\newcommand{\LocSys}{\mathrm{LocSys}}
\newcommand{\IndCoh}{\mathrm{IndCoh}}
\newcommand{\Coh}{\mathrm{Coh}}
\newcommand{\Perf}{\mathrm{Perf}}
\newcommand{\QCoh}{\mathrm{QCoh}}
\newcommand{\Lie}{\mathrm{Lie}}
\newcommand{\GL}{\mathrm{GL}}
\newcommand{\SL}{\mathrm{SL}}
\newcommand{\Spa}{\mathrm{Spa}}
\newcommand{\Hom}{\mathrm{Hom}}
\newcommand{\Shv}{\mathrm{Shv}}
\newcommand{\Rep}{\mathrm{Rep}}
\newcommand{\Gal}{\mathrm{Gal}}
\newcommand{\WF}{W_F}
\newcommand{\ot}{\otimes}
\newcommand{\checkG}{\check G}
\newcommand{\checkM}{\check M}
\newcommand{\checkP}{\check P}
\newcommand{\checkT}{\check T}
\newcommand{\checkH}{\check H}
\newcommand{\Spec}{\mathrm{Spec}}
\newcommand{\Spf}{\mathrm{Spf}}

\title{A Gaitsgory-Style Blueprint for the Fargues--Scholze Categorical Geometrization Conjecture\\(the case $\ell\neq p$)}
\author{Przemyslaw Chojecki}
\date{\today}

\begin{document}
\maketitle

\begin{abstract}
We outline a proof strategy for the categorical geometrization conjecture of Fargues and Scholze for
$\ell$-adic sheaves on the stack $\Bun_G$ of $G$-bundles on the Fargues--Fontaine curve, in the case
$\ell\neq p$.  The conjecture predicts an equivalence between a Whittaker-generated automorphic
subcategory of $\ell$-adic sheaves on $\Bun_G$ and a spectral category of coherent or ind-coherent
sheaves on the stack of Langlands parameters, with a nilpotent singular support condition in the integral
setting.  Our goal is to place all functors and structural inputs into a single argument template that
parallels the architecture of Gaitsgory's proof of geometric Langlands: singular support, Hecke and
spectral actions, parabolic induction and constant term, gluing from Levi subgroups, and a
Barr--Beck type reduction.
\end{abstract}

\tableofcontents

\section{Introduction}\label{sec:intro}

\subsection{Background}
Let $E$ be a non-archimedean local field of residue characteristic $p$, let $G/E$ be a connected
reductive group, and fix a prime $\ell\neq p$.  The work of Fargues and Scholze constructs a robust
geometric avatar of the local Langlands correspondence using the v-stack $\Bun_G$ of $G$-bundles
on the Fargues--Fontaine curve and a corresponding category of $\ell$-adic sheaves
$D_{\mathrm{lis}}(\Bun_G,\Lambda)$, together with Hecke operators, a geometric Satake equivalence, a
stack of Langlands parameters, and a spectral action by perfect complexes
\cite{FSGeometrization,FSReview}.

A central conjecture in \cite{FSReview} upgrades the local Langlands correspondence to an
equivalence of categories, closely resembling the role played by Whittaker models in geometric Langlands.
It is this conjecture that we focus on.

\subsection{Main target: the categorical geometrization conjecture}
Assume $G$ is quasi-split and fix a Whittaker datum $(B,\psi)$.
Fargues and Scholze construct a Whittaker object $\mathcal{W}_\psi\in D_{\mathrm{lis}}(\Bun_G,\Lambda)$
and define the Whittaker-generated subcategory $D_{\mathrm{lis}}(\Bun_G,\Lambda)_\omega$
(see Section~\ref{sec:whittaker}).
On the spectral side, let $\LocSys_{\checkG}$ denote the stack of $\ell$-adic Langlands parameters
for $G$ (constructed in \cite{FSGeometrization} and also algebraized by Dat--Helm--Kurinczuk--Moss
\cite{DHKMParameters}), and let $\Coh_{\mathrm{Nilp}}(\LocSys_{\checkG})$ denote coherent sheaves
with nilpotent singular support (Section~\ref{sec:spectral}).

\begin{conjecture}[Fargues--Scholze, categorical geometrization]\label{conj:FS-cat-geom}
The spectral action functor
\[
  \Perf(\LocSys_{\checkG})\ \longrightarrow\ D_{\mathrm{lis}}(\Bun_G,\Zl),\qquad
  \mathcal{F}\ \longmapsto\ \mathcal{F}\ast \mathcal{W}_\psi
\]
takes values in compact objects when $\mathcal{F}$ has quasi-compact support, and extends to an
equivalence of stable $\Zl$-linear categories
\[
  \Coh_{\mathrm{Nilp}}(\LocSys_{\checkG})\ \xrightarrow{\ \sim\ }\ D_{\mathrm{lis}}(\Bun_G,\Zl)_\omega
\]
compatible with the spectral action.
Over $\Ql$ the nilpotent singular support condition is expected to be automatic.
\end{conjecture}

Conjecture~\ref{conj:FS-cat-geom} is stated in \cite[\S 6]{FSReview} (in a slightly different notation).
It is the local analogue of the identification of Whittaker categories with coherent sheaves with nilpotent
singular support in geometric Langlands.

\subsection{Plan of the paper and proof architecture}
This paper is a working blueprint.  Each section isolates the relevant constructions, states the
intermediate claims needed for a proof of Conjecture~\ref{conj:FS-cat-geom}, and provides a proof
sketch or a reduction to known results.

Our intended proof follows the architecture of Gaitsgory's approach to geometric Langlands
\cite{AGSingSupp,GRProofI,GRProofIV,GRProofV}:

\begin{enumerate}[label=\textbf{Step \arabic*.}, leftmargin=2.8em]
  \item \textbf{Set up automorphic and spectral categories.}
  Define $D_{\mathrm{lis}}(\Bun_G,\Lambda)$, the Whittaker object $\mathcal{W}_\psi$, and the spectral stack
  $\LocSys_{\checkG}$, together with $\Coh_{\mathrm{Nilp}}(\LocSys_{\checkG})$ via singular support.

  \item \textbf{Construct the comparison functor.}
  Use the spectral action of $\Perf(\LocSys_{\checkG})$ on $D_{\mathrm{lis}}(\Bun_G,\Lambda)$ to define
  $\Phi(\mathcal{F})=\mathcal{F}\ast \mathcal{W}_\psi$.

  \item \textbf{Compactness and finiteness.}
  Prove that $\Phi$ sends $\Perf$ with quasi-compact support to compact objects, and extend $\Phi$
  to $\Coh_{\mathrm{Nilp}}$.

  \item \textbf{Full faithfulness via centers and monadicity.}
  Identify $\End(\mathcal{W}_\psi)$ with functions on $\LocSys_{\checkG}$ (or the stable center).
  Use Barr--Beck type arguments to deduce full faithfulness of $\Phi$.

  \item \textbf{Essential surjectivity by parabolic gluing.}
  Use geometric Eisenstein series and constant term functors on $\Bun_G$ (in the strong form developed
  by Hamann--Hansen--Scholze \cite{HHSGeometricEis}) to glue from Levi subgroups and reduce
  to cuspidal blocks.

  \item \textbf{Cuspidal block analysis and multiplicity one.}
  Establish the cuspidal analogue of ``multiplicity one'' and complete the equivalence.
\end{enumerate}

\subsection*{Status}
Many foundational inputs are theorems: the construction of $\Bun_G$, $\ell$-adic sheaves, geometric Satake,
the spectral action, and the stack of parameters are in \cite{FSGeometrization}, and the parabolic formalism
needed for gluing is developed in \cite{HHSGeometricEis}.  The new work required for a full proof is the
systematic assembly of these inputs into a Gaitsgory-style argument with singular support and monadicity.

% ============================================================
% PATCH 1 (add a "Known results / checkpoints" paragraph in the Introduction)
% Put this near the end of the Introduction, e.g. after "Status".
% ============================================================

\subsection*{Checkpoints and known comparisons (evidence)}
While the categorical equivalence of Conjecture~\ref{conj:FS-cat-geom} is open in general, several pieces of the
package are already theorems and provide concrete checkpoints.

\begin{itemize}[leftmargin=2em]
  \item \textbf{Compatibility with the classical LLC.}
  The construction of Hecke eigensheaves on $\Bun_G$ and the spectral action are designed to recover the usual
  local Langlands correspondence after passing to appropriate fibers and taking traces; see
  \cite{FSGeometrization,FSReview}.
  For specific groups (including $\mathrm{GSp}_4$), compatibility with known LLC normalizations is now proved;
  see \cite{HamannGanTakeda}.
  \item \textbf{Tori.}
  When $G=T$ is a torus, the geometry of $\Bun_T$ and the spectral stack are abelian, and the expected
  equivalence is compatible with local class field theory (no proper parabolics, no nilpotent correction).
  This case serves as a sanity check for normalizations and spectral functoriality.
  \item \textbf{Stable Bernstein center and endomorphisms.}
  The map $\Gamma(\LocSys_{\checkG},\OO)\to \End(W_{\psi})$ is a categorical form of the \emph{stable Bernstein
  center} and is closely related to classical work on stable center conjectures
  \cite{HainesStableCenter,BKVStableCenter,VarmaStableCenterNotes}.
  For $G=\GL_n$, integral Bernstein center and Whittaker/co-Whittaker refinements are available \cite{HelmWhittakerBernstein},
  and one expects to upgrade several “Task~1/Task~4” steps in this paper to theorems in that case using
  local Langlands in families \cite{EmertonHelmFamilies,HelmBernsteinCenterWk,HelmMossFamilies}.
\end{itemize}

\paragraph{What this paper proves unconditionally.}
At the current stage, the fully written unconditional results in the body are:
(i) the parabolic calculation $\mathrm{CT}_{P^{-}}(W_{\psi})\simeq W_{\psi_M}$ (Section~\ref{sec:CT-Whittaker}),
and (ii) the reduction of $\End(W_{\psi})$ to the endomorphisms of the universal Gelfand--Graev representation
(Section~\ref{sec:endo-Whittaker}), together with the comparison to generic Bernstein centers and the $\GL_n$
identification \emph{up to} the remaining stable-center comparison input.



\section{Notation and standing assumptions}\label{sec:notation}

Throughout:
\begin{itemize}[leftmargin=2em]
  \item $E$ is a non-archimedean local field of residue characteristic $p$ and residue field $\mathbb{F}_q$.
  \item $\ell$ is a fixed prime with $\ell\neq p$.
  \item $\Lambda$ denotes one of $\Zl$, $\Zl[\sqrt{q}]$, $\Ql$, or $\Ql[\sqrt{q}]$, chosen so that Satake normalizations
  are available (as in \cite{FSReview}).
  \item $G/E$ is a connected reductive group.  When discussing Whittaker objects, we assume $G$ is quasi-split
  and fix a Whittaker datum $(B,\psi)$.
  \item $\checkG$ denotes the Langlands dual group over $\Lambda$, and ${}^LG$ denotes the $L$-group.
\end{itemize}

\section{The stack \texorpdfstring{$\Bun_G$}{BunG} and its stratification}\label{sec:BunG}

\subsection{Definition of \texorpdfstring{$\Bun_G$}{BunG}}
Let $X_{S,E}$ denote the relative Fargues--Fontaine curve over a perfectoid base $S$.
Following \cite{FSGeometrization}, the v-stack
\[
  \Bun_G := \Bun_G(X_{-/E})
\]
classifies $G$-bundles on $X_{S,E}$ in families.

\subsection{Harder--Narasimhan strata and local groups}
The topological space $|\Bun_G|$ is stratified by the Kottwitz set $B(G)$.
For $b\in B(G)$, the stratum $\Bun_G^b$ is controlled by a locally profinite group $J_b(E)$.

\begin{proposition}[Strata and automorphism groups]\label{prop:strata}
For each $b\in B(G)$ there is a natural morphism $\Bun_G^b\to B\underline{J_b(E)}$ which is expected
to be an equivalence on semistable points, and for basic $b$ one has
\[
  \Bun_G^b \simeq [*/J_b(E)]
\]
in the sense of Artin v-stacks.
\end{proposition}

\paragraph{Sketch of proof.}
This is part of the geometric structure theory of $\Bun_G$ developed in \cite[\S III]{FSGeometrization}
and summarized in \cite[\S 2]{FSReview}.  The key input is the classification of $G$-bundles on the
Fargues--Fontaine curve by $B(G)$ and the description of automorphism groups in terms of $J_b$.

\subsection{Gluing along strata}
The geometry of the inclusions $\Bun_G^{\leq b}\hookrightarrow \Bun_G$ and of the local charts $M_b\to \Bun_G$
constructed in \cite{FSGeometrization} provides a basis for recollement arguments.  For torsion coefficients,
a detailed gluing formalism along Harder--Narasimhan strata is developed by Miles \cite{MilesGluingHN}.

\section{Automorphic categories: $\ell$-adic sheaves on \texorpdfstring{$\Bun_G$}{BunG}}\label{sec:automorphic}

\subsection{The category \texorpdfstring{$D_{\mathrm{lis}}(\Bun_G,\Lambda)$}{Dlis(BunG)}}
Following \cite{FSGeometrization}, one defines the derived category
$D_{\mathrm{lis}}(\Bun_G,\Lambda)$ of lisse $\Lambda$-sheaves on $\Bun_G$ (and variants with constructibility or torsion
coefficients).  This category admits:
\begin{itemize}[leftmargin=2em]
  \item pullbacks and pushforwards along morphisms of v-stacks in the range needed for Hecke correspondences,
  \item Verdier duality, and
  \item compact objects corresponding to ``constructible with quasi-compact support'' subcategories.
\end{itemize}

\begin{axiom}[Compact generation package]\label{ax:compactgen}
The full subcategory of objects supported on a quasi-compact open substack is compactly generated, and
the six operation formalism preserves compactness under the finiteness hypotheses proved in \cite{FSGeometrization}
and \cite{HHSGeometricEis}.
\end{axiom}

\paragraph{Comment.}
In the paper itself, Axiom~\ref{ax:compactgen} will be replaced by precise statements, extracted from
\cite[\S IV--V]{FSGeometrization} and \cite{HHSGeometricEis}.

\section{The spectral stack of parameters and nilpotent singular support}\label{sec:spectral}

\subsection{Algebraic models for the stack of parameters}
Fargues and Scholze construct a stack $\LocSys_{\checkG}$ of Langlands parameters and establish
its basic functoriality \cite{FSGeometrization,FSReview}.
An algebraic model for the moduli of Langlands parameters is constructed by Dat--Helm--Kurinczuk--Moss
\cite{DHKMParameters}, producing a stack locally of finite type over $\Z[1/p]$.

\begin{definition}[Spectral parameter stack]\label{def:LocSys}
Fix $\Lambda$ as in Section~\ref{sec:notation}.  We write $\LocSys_{\checkG}$ for the base change of the
DHKM parameter stack to $\Spec(\Lambda)$, equipped with its derived enhancement.
\end{definition}

\subsection{Derived geometry and singular support}
To formulate nilpotent singular support, one needs $\LocSys_{\checkG}$ to be quasi-smooth in the derived
sense (a local complete intersection).  The DHKM result that the parameter stack is a reduced local complete
intersection provides the expected input for this.

\begin{axiom}[Quasi-smoothness]\label{ax:quasismooth}
The derived enhancement of $\LocSys_{\checkG}$ is quasi-smooth, so that singular support for ind-coherent
sheaves is defined in the sense of Arinkin--Gaitsgory \cite{AGSingSupp}.
\end{axiom}

\subsection{Nilpotent cone and the category \texorpdfstring{$\Coh_{\mathrm{Nilp}}$}{CohNilp}}
Let $\mathcal{N}_{\checkG}\subset \Lie(\checkG)^\ast$ denote the nilpotent cone, and consider the corresponding
conical subset of the singular support space of $\LocSys_{\checkG}$.

\begin{definition}[Nilpotent support category]\label{def:CohNilp}
Let $\Coh_{\mathrm{Nilp}}(\LocSys_{\checkG})$ denote the full subcategory of $\IndCoh(\LocSys_{\checkG})$
consisting of objects with quasi-compact support and singular support contained in the nilpotent cone.
\end{definition}

\begin{remark}
In the rational coefficient case $\Lambda=\Ql$ (or $\Ql[\sqrt{q}]$), \cite{FSReview} explains that the nilpotent
support condition is expected to be automatic.
In the integral case, it is a genuine restriction and is the local analogue of the nilpotent singular support
condition in geometric Langlands \cite{AGSingSupp}.
\end{remark}

\section{Hecke correspondences, geometric Satake, and the spectral action}\label{sec:hecke-satake}

\subsection{Hecke stacks and Hecke operators}
The Hecke correspondence on $\Bun_G$ is defined using modifications of $G$-bundles at the untilt point.
This yields for each $V\in\Rep(\checkG)$ a Hecke functor
\[
  T_V:\ D_{\mathrm{lis}}(\Bun_G,\Lambda)\ \longrightarrow\ D_{\mathrm{lis}}(\Bun_G,\Lambda).
\]

\paragraph{Sketch of construction.}
This is \cite[\S I.6]{FSGeometrization}.  The Hecke stack is built from the $B_{\mathrm{dR}}$-affine Grassmannian
and the Beauville--Laszlo uniformization, and the Hecke functors are defined by pullback, tensor, and pushforward
along this correspondence.

\subsection{Geometric Satake}
Fargues and Scholze prove a geometric Satake equivalence over the Fargues--Fontaine curve, identifying the
spherical Hecke category with $\Rep(\checkG)$ (with coefficients in $\Lambda$), compatibly with convolution.

\begin{theorem}[Geometric Satake over the Fargues--Fontaine curve]\label{thm:satake}
There is an equivalence of monoidal categories between the spherical Hecke category on the
$B_{\mathrm{dR}}$-affine Grassmannian and $\Rep(\checkG)$.
\end{theorem}

\paragraph{Sketch of proof.}
This is established in \cite[\S I.6]{FSGeometrization}.  The argument parallels geometric Satake for
algebraic curves, with the untilt point playing the role of the point of modification.

\subsection{The spectral action}
The geometric Satake equivalence and the parameter stack $\LocSys_{\checkG}$ together yield a canonical monoidal
action
\[
  \Perf(\LocSys_{\checkG})\ \curvearrowright\ D_{\mathrm{lis}}(\Bun_G,\Lambda)
\]
called the spectral action in \cite[\S I.10]{FSGeometrization}.

\begin{proposition}[Compatibility of spectral and Hecke actions]\label{prop:hecke-spectral}
For $V\in\Rep(\checkG)\subset \Perf(\LocSys_{\checkG})$, the induced endofunctor of
$D_{\mathrm{lis}}(\Bun_G,\Lambda)$ coincides with the Hecke operator $T_V$.
\end{proposition}

\paragraph{Sketch of proof.}
This is a formal consequence of geometric Satake and the definition of the spectral action
\cite[\S I.10]{FSGeometrization}.

\section{Whittaker sheaves and the Whittaker-generated subcategory}\label{sec:whittaker}

\subsection{The Whittaker object}
Assume $G$ is quasi-split and fix a Whittaker datum $(B,\psi)$ with unipotent radical $U$.
Fargues and Scholze define a Whittaker object $\mathcal{W}_\psi\in D_{\mathrm{lis}}(\Bun_G,\Lambda)$
as a geometric incarnation of compact induction from $U(E)$ with character $\psi$ \cite{FSReview}.

\begin{definition}[Whittaker category]\label{def:whittaker-cat}
Let $D_{\mathrm{lis}}(\Bun_G,\Lambda)_\omega$ denote the smallest stable full subcategory of
$D_{\mathrm{lis}}(\Bun_G,\Lambda)$ that contains $\mathcal{W}_\psi$, is closed under colimits, and is stable under the
spectral action of $\Perf(\LocSys_{\checkG})$.
\end{definition}

\subsection{The comparison functor}
Define a functor
\[
  \Phi:\ \Perf(\LocSys_{\checkG})\ \longrightarrow\ D_{\mathrm{lis}}(\Bun_G,\Lambda)_\omega,
  \qquad \mathcal{F}\ \longmapsto\ \mathcal{F}\ast \mathcal{W}_\psi.
\]
Conjecture~\ref{conj:FS-cat-geom} asserts that $\Phi$ extends to an equivalence from
$\Coh_{\mathrm{Nilp}}(\LocSys_{\checkG})$.

\section{Parabolic functors and gluing from Levi subgroups}\label{sec:parabolic}

\subsection{Geometric Eisenstein series and constant term}
Let $P\subset G$ be a parabolic subgroup with Levi quotient $M$.
One has maps of v-stacks
\[
  \Bun_M \xleftarrow{q} \Bun_P \xrightarrow{p} \Bun_G.
\]
Hamann--Hansen--Scholze construct geometric Eisenstein and constant term functors
\cite{HHSGeometricEis}
\[
  \mathrm{Eis}_P := p_\ast q^!:\ D_{\mathrm{lis}}(\Bun_M,\Lambda)\to D_{\mathrm{lis}}(\Bun_G,\Lambda),
  \qquad
  \mathrm{CT}_P := q_\ast p^!:\ D_{\mathrm{lis}}(\Bun_G,\Lambda)\to D_{\mathrm{lis}}(\Bun_M,\Lambda),
\]
prove finiteness theorems for these functors, and establish a geometric form of Bernstein's second adjointness.

\begin{theorem}[Finiteness and adjointness for parabolic functors]\label{thm:HHS}
Geometric Eisenstein and constant term functors satisfy finiteness properties analogous to those of parabolic
induction and Jacquet modules, and enjoy adjunction and second adjointness statements that are sufficiently
strong to support gluing arguments.
\end{theorem}

\paragraph{Sketch of proof.}
This is the main content of \cite{HHSGeometricEis}.  The proof uses the geometry of moduli of parabolic bundles,
continuity of gluing functors between Harder--Narasimhan strata, and a detailed analysis of the relevant
correspondences.

\subsection{Cuspidal and Eisenstein decompositions}
A crucial structural output of \cite{HHSGeometricEis} is a decomposition of $D_{\mathrm{lis}}(\Bun_G,\Lambda)$ into
cuspidal and Eisenstein parts, analogous to the classical Bernstein decomposition in representation theory.

\begin{proposition}[Cuspidal subcategory]\label{prop:cuspidal}
Define $D_{\mathrm{cusp}}(\Bun_G,\Lambda)$ as the intersection of kernels of all constant term functors for proper
parabolics.  Then $D_{\mathrm{lis}}(\Bun_G,\Lambda)$ is generated under colimits by $D_{\mathrm{cusp}}(\Bun_G,\Lambda)$
and the essential images of Eisenstein series functors from Levi subgroups.
\end{proposition}

\paragraph{Sketch of proof.}
This is proved in \cite{HHSGeometricEis} as an application of their finiteness results and adjointness.

\section{Proof strategy for the categorical geometrization conjecture}\label{sec:proof-strategy}

This section is the core blueprint: it explains how Conjecture~\ref{conj:FS-cat-geom} should follow from a sequence
of intermediate steps that parallel Gaitsgory's proof strategy.

\subsection{Compactness and extension from \texorpdfstring{$\Perf$}{Perf} to \texorpdfstring{$\Coh_{\mathrm{Nilp}}$}{CohNilp}}
\begin{proposition}[Compactness of Whittaker translates]\label{prop:compactness}
Let $\mathcal{F}\in\Perf(\LocSys_{\checkG})$ have quasi-compact support.  Then $\mathcal{F}\ast \mathcal{W}_\psi$
is compact in $D_{\mathrm{lis}}(\Bun_G,\Zl)$.
\end{proposition}

\paragraph{Sketch of proof.}
The compactness statement is the first non-formal obstacle.
A proposed approach is:
\begin{enumerate}[label=(\alph*), leftmargin=2em]
  \item Use the compatibility of the spectral action with Hecke operators
  (Proposition~\ref{prop:hecke-spectral}) to reduce compactness to finiteness properties of Hecke correspondences.
  \item Use stratifications of $\Bun_G$ and the continuity and finiteness results of \cite{HHSGeometricEis} to
  show that Hecke operators preserve compactness on objects supported on a bounded range of Harder--Narasimhan
  strata.
  \item Use that $\mathcal{F}$ has quasi-compact support on $\LocSys_{\checkG}$ to obtain uniform bounds on the
  complexity of the corresponding Hecke operators.
\end{enumerate}
Once this is established, one can extend $\Phi$ from $\Perf$ to $\Coh_{\mathrm{Nilp}}$ by
compact generation and devissage.

\subsection{Full faithfulness via endomorphisms and monadicity}
\begin{proposition}[Endomorphisms of the Whittaker generator]\label{prop:endo}
There is a canonical identification
\[
  \End_{D_{\mathrm{lis}}(\Bun_G,\Lambda)}(\mathcal{W}_\psi)\ \cong\ \Gamma(\LocSys_{\checkG},\mathcal{O}),
\]
compatible with the action of $\Perf(\LocSys_{\checkG})$.
\end{proposition}

\paragraph{Sketch of proof.}
Fargues and Scholze construct a map from the spectral Bernstein center to the Bernstein center using the spectral
action \cite{FSGeometrization,FSReview}.  The endomorphisms of $\mathcal{W}_\psi$ should recover the same center map.
The identification is expected to follow by:
\begin{enumerate}[label=(\alph*), leftmargin=2em]
  \item relating $\End(\mathcal{W}_\psi)$ to the center of the compactly generated subcategory
  $D_{\mathrm{lis}}(\Bun_G,\Lambda)_\omega$,
  \item identifying the latter with functions on $\LocSys_{\checkG}$ by the spectral action, and
  \item using algebraicity properties of $\LocSys_{\checkG}$ from \cite{DHKMParameters}.
\end{enumerate}

\begin{proposition}[Barr--Beck reduction]\label{prop:bb}
Assume Propositions~\ref{prop:compactness} and \ref{prop:endo}.
Then the functor $\Phi:\Coh_{\mathrm{Nilp}}(\LocSys_{\checkG})\to D_{\mathrm{lis}}(\Bun_G,\Lambda)_\omega$
is fully faithful if the right adjoint of $\Phi$ is conservative.
\end{proposition}

\paragraph{Sketch of proof.}
Once $\Phi$ preserves compact objects and admits a continuous right adjoint, one can apply a Barr--Beck type theorem
to compare $D_{\mathrm{lis}}(\Bun_G,\Lambda)_\omega$ with modules for the monad $\Phi^R\circ \Phi$.
The identification in Proposition~\ref{prop:endo} is expected to identify this monad with the tautological monad
on the spectral side, giving full faithfulness.

\subsection{Essential surjectivity by parabolic gluing}
\begin{proposition}[Generation by the Whittaker object]\label{prop:generation}
The category $D_{\mathrm{lis}}(\Bun_G,\Lambda)_\omega$ is generated under colimits by the essential image of
$\Phi$.
\end{proposition}

\paragraph{Sketch of proof.}
The key point is that $D_{\mathrm{lis}}(\Bun_G,\Lambda)_\omega$ is by definition the smallest subcategory
containing $\mathcal{W}_\psi$ and stable under the spectral action.  Thus essential surjectivity reduces to
showing that the spectral action of $\Perf(\LocSys_{\checkG})$ on $\mathcal{W}_\psi$ is already large enough
to generate the whole Whittaker subcategory.  Parabolic gluing should be used to reduce this to:
\begin{enumerate}[label=(\alph*), leftmargin=2em]
  \item the analogous statement for Levi subgroups (induction on semisimple rank),
  \item the cuspidal case, where Whittaker objects are expected to detect blocks,
  \item compatibility of $\Phi$ with constant term and Eisenstein functors (Section~\ref{sec:parabolic}).
\end{enumerate}

\subsection{Cuspidal blocks and multiplicity one}
A final step in a Gaitsgory-style proof is a cuspidal uniqueness statement, serving as a local analogue of
multiplicity one in geometric Langlands.

\begin{conjecture}[Cuspidal multiplicity one for the Whittaker category]\label{conj:multone-local}
For a cuspidal Langlands parameter $\phi$, the fiber of $D_{\mathrm{lis}}(\Bun_G,\Ql)_\omega$ over $\phi$ is
equivalent to $\Perf(\Rep_{\Ql}(S_\phi))$, and the generic representation in the corresponding packet is unique
with respect to the Whittaker datum.
\end{conjecture}

\paragraph{Comment.}
This is stated in \cite[\S 5]{FSReview} as a ``toy model'' that the full categorical conjecture refines.
In a proof of Conjecture~\ref{conj:FS-cat-geom}, Conjecture~\ref{conj:multone-local} is the natural place where
one expects to use the geometry of local shtukas and the known construction of Hecke eigensheaves
\cite{FSGeometrization}.

\section{Consequences and further directions}\label{sec:consequences}

Assuming Conjecture~\ref{conj:FS-cat-geom}, one obtains:
\begin{itemize}[leftmargin=2em]
  \item identification of stable centers and the spectral Bernstein center map \cite{FSReview},
  \item functoriality for $L$-morphisms realized by geometric kernels on $\Bun_H\times \Bun_G$ \cite{FSReview},
  \item structural decompositions of $D_{\mathrm{lis}}(\Bun_G,\Lambda)$ compatible with parabolic induction
  \cite{HHSGeometricEis},
  \item a conceptual categorical formulation of local Langlands parameters in families, compatible with the
  moduli-theoretic construction of Langlands parameters \cite{DHKMParameters}.
\end{itemize}

\section{Checklist of inputs and where they enter}\label{app:checklist}

This appendix will be expanded to a detailed dependency chart.
At the current stage, the intended inputs are:
\begin{itemize}[leftmargin=2em]
  \item \cite{FSGeometrization}: definition of $\Bun_G$, $\ell$-adic sheaves, geometric Satake, spectral action,
  Whittaker object, and the parameter stack.
  \item \cite{FSReview}: formulation of the categorical conjecture and its consequences.
  \item \cite{DHKMParameters}: algebraicity and local complete intersection properties of the parameter stack.
  \item \cite{HHSGeometricEis}: finiteness and adjointness for Eisenstein series and constant term functors.
  \item \cite{AGSingSupp}: singular support and the definition of nilpotent support categories.
  \item \cite{GRProofI,GRProofIV,GRProofV}: the proof architecture and monadicity patterns to be adapted.
  \item \cite{MilesGluingHN}: detailed gluing along Harder--Narasimhan strata for prime-to-$p$ torsion coefficients.
\end{itemize}






\section{The stack $\Bun_G$ and its stratification}\label{sec:BunG}

This section recalls the geometric object on the automorphic side: the v-stack $\Bun_G$ of $G$-bundles on the
Fargues--Fontaine curve, together with its Harder--Narasimhan stratification by the Kottwitz set $B(G)$.
We will use three kinds of input from this geometry throughout the paper:
\begin{enumerate}[label=(\roman*), leftmargin=2em]
  \item the classification of geometric points of $\Bun_G$ by $B(G)$ and the description of strata in terms of
  groups $J_b(E)$;
  \item boundedness statements ensuring that natural unions of strata are quasi-compact (needed for compactness
  and finiteness arguments);
  \item recollement and gluing along Harder--Narasimhan strata (needed for inductive and parabolic gluing
  arguments).
\end{enumerate}
All of these statements are proved or explained in \cite{FSGeometrization,FSReview}; we include them here in a
form convenient for later use.

\subsection{The relative Fargues--Fontaine curve and $G$-bundles}

Let $E$ be a non-archimedean local field of residue characteristic $p$.
For a perfectoid space $S$ of characteristic $p$, there is a relative Fargues--Fontaine curve $X_{S,E}$
(and a relative adic space or diamond version), functorial in $S$.
We will not reproduce the construction, but we emphasize two properties:
\begin{itemize}[leftmargin=2em]
  \item for a geometric point $S=\Spa(C,C^+)$ with $C$ an algebraically closed perfectoid field of characteristic
  $p$, the category of vector bundles on $X_{C,E}$ (and more generally $G$-bundles for reductive $G$) is
  \emph{discrete up to isomorphism}, in the sense that it is classified by isocrystal-type invariants;
  \item the construction is functorial enough to define moduli of bundles in families over arbitrary bases $S$,
  producing a v-stack.
\end{itemize}

\begin{definition}[The v-stack $\Bun_G$]\label{def:BunG}
Let $G/E$ be a connected reductive group.
The v-stack $\Bun_G$ is the functor on perfectoid spaces $S$ (in characteristic $p$) given by the groupoid
of $G$-bundles on $X_{S,E}$:
\[
  \Bun_G(S)\ :=\ \{\text{$G$-bundles on $X_{S,E}$}\}.
\]
\end{definition}

\begin{remark}
All geometric constructions in \cite{FSGeometrization} are formulated at the level of diamonds and v-stacks.
In particular, $\Bun_G$ is an Artin v-stack locally of diamond type, and it admits a well-behaved theory of
$\ell$-adic lisse sheaves for $\ell\neq p$.
\end{remark}

\subsection{The Kottwitz set $B(G)$ and its structure}

Fix an algebraic closure $\overline{E}$ of $E$ and let $\breve{E}$ be the completion of the maximal unramified
extension of $E$ inside $\overline{E}$. Let $\sigma$ denote the Frobenius automorphism of $\breve{E}/E$.
Recall that $B(G)$ is the set of $\sigma$-conjugacy classes in $G(\breve{E})$:
\[
  B(G)\ :=\ G(\breve{E})/\!\sim,\qquad b\sim b' \Longleftrightarrow \exists g\in G(\breve{E})\text{ with }
  b'=g^{-1}b\sigma(g).
\]

The set $B(G)$ carries two fundamental invariants:
\begin{itemize}[leftmargin=2em]
  \item the \emph{Kottwitz invariant} $\kappa_G: B(G)\to \pi_1(G)_{\Gamma}$ (where $\Gamma=\Gal(\overline{E}/E)$);
  \item the \emph{Newton point} $\nu: B(G)\to (X_\ast(T)\otimes_\Z \Q)^{+}$ for a choice of maximal torus $T$,
  landing in dominant rational cocharacters.
\end{itemize}
These maps are compatible with the partial order on $B(G)$ defined by $\nu$ (and equality of $\kappa_G$).
We recall the terminology:

\begin{definition}[Basic elements]\label{def:basic}
An element $b\in B(G)$ is \emph{basic} if its Newton point is central (equivalently, if the associated $G$-bundle
is semistable).  We write $B(G)_{\mathrm{basic}}\subset B(G)$ for the subset of basic elements.
\end{definition}

\begin{remark}[Boundedness and finiteness]
A recurring principle is that imposing bounds on $\nu$ and fixing $\kappa_G$ cuts $B(G)$ down to a finite set.
We will use this to reduce statements on $\Bun_G$ to statements on finite unions of strata.
\end{remark}

\subsection{Harder--Narasimhan stratification of $\Bun_G$}

Fargues and Scholze define a Harder--Narasimhan map from the topological space underlying $\Bun_G$ to $B(G)$,
which on geometric points recovers the isomorphism class of the associated $G$-bundle \cite{FSGeometrization}.

\begin{definition}[Harder--Narasimhan strata]\label{def:HN-strata}
For $b\in B(G)$, let $\Bun_G^b\subset \Bun_G$ denote the locally closed sub-v-stack consisting of those
$G$-bundles whose Harder--Narasimhan invariant equals $b$.
We write $\Bun_G^{\leq b}$ for the open sub-v-stack consisting of bundles whose Harder--Narasimhan invariant is
$\leq b$ in the partial order on $B(G)$.
\end{definition}

The key structural input is that each stratum is a classifying stack, controlled by a locally profinite group.

\begin{definition}[The group $J_b$]\label{def:Jb}
Fix $b\in G(\breve{E})$ representing a class in $B(G)$.
Define an affine group scheme $J_b$ over $E$ by the functor
\[
  J_b(R)\ :=\ \{ g\in G(R\otimes_E \breve{E}) \mid g^{-1}b\sigma(g)=b\},
\]
for $E$-algebras $R$.
Its group of $E$-points $J_b(E)$ is a locally profinite group.
\end{definition}

\begin{proposition}[Description of strata as classifying stacks]\label{prop:BunG-stratum-BJ}
For each $b\in B(G)$ there is a canonical equivalence of v-stacks
\[
  \Bun_G^b\ \simeq\ B\underline{J_b(E)},
\]
where $B\underline{J_b(E)}$ denotes the classifying v-stack of the constant sheaf of groups $J_b(E)$.
\end{proposition}

\begin{proof}[Sketch of proof]
Choose a representative $b\in G(\breve{E})$.
Fargues associates to $b$ a $G$-bundle $\mathcal{E}_b$ on the Fargues--Fontaine curve, and the classification
theorem asserts that every geometric $G$-bundle is isomorphic to a unique $\mathcal{E}_b$.
The automorphism group of $\mathcal{E}_b$ is identified with $J_b(E)$.
Twisting $\mathcal{E}_b$ by a $J_b(E)$-torsor produces a map $B\underline{J_b(E)}\to \Bun_G$, and one checks that
its essential image is exactly the stratum $\Bun_G^b$ and that it induces an equivalence onto that stratum.
All details are in \cite[\S III]{FSGeometrization} and \cite[\S 2]{FSReview}.
\end{proof}

\begin{remark}[Semistable locus]
The semistable locus of $\Bun_G$ is the union of the basic strata:
\[
  \Bun_G^{\mathrm{ss}}\ :=\ \coprod_{b\in B(G)_{\mathrm{basic}}}\Bun_G^b.
\]
In particular, if $b$ is basic then $J_b$ is an inner form of $G$ and $J_b(E)$ is the group that governs the
corresponding inner form block of representation theory.
\end{remark}

\subsection{Bounded open substacks and quasi-compactness}

Most geometric arguments on $\Bun_G$ proceed by restricting to a quasi-compact open substack on which the
relevant correspondences are proper or of finite cohomological dimension.  The Harder--Narasimhan stratification
provides a supply of such opens.

\begin{definition}[Bounded opens]\label{def:bounded-opens}
Fix a dominant rational cocharacter $\mu$ (or, more generally, a finite subset of $B(G)$ stable under the
partial order).  Let $\Bun_G^{\leq \mu}\subset \Bun_G$ denote the open sub-v-stack consisting of bundles whose
Newton point is $\leq \mu$ and whose Kottwitz invariant lies in the corresponding finite set.
Equivalently, $\Bun_G^{\leq \mu}$ is a finite union of strata $\Bun_G^b$.
\end{definition}

\begin{proposition}[Quasi-compactness of bounded opens]\label{prop:bounded-qc}
The open sub-v-stack $\Bun_G^{\leq \mu}$ is quasi-compact, and it is a finite disjoint union of classifying
stacks $B\underline{J_b(E)}$.
\end{proposition}

\begin{proof}[Sketch of proof]
By the finiteness principle for $B(G)$ under bounded Newton point and fixed Kottwitz invariant, only finitely
many $b$ occur in $\Bun_G^{\leq \mu}$.
Each stratum $\Bun_G^b\simeq B\underline{J_b(E)}$ is quasi-compact (as a classifying v-stack of a locally profinite
group), hence their finite union is quasi-compact.
See \cite[\S 2]{FSReview}.
\end{proof}

\subsection{Recollement along Harder--Narasimhan strata}

The stratification by $B(G)$ induces recollement patterns for categories of $\ell$-adic sheaves.
This is the geometric mechanism behind ``gluing from strata'' arguments on the automorphic side.

\begin{proposition}[Recollement for prime-to-$p$ torsion coefficients]\label{prop:recollement}
Let $\Lambda$ be a torsion ring of characteristic $\ell\neq p$.
Let $j:\Bun_G^{\leq \mu}\hookrightarrow \Bun_G$ be a bounded open immersion and let
$i:\Bun_G\setminus \Bun_G^{\leq \mu}\hookrightarrow \Bun_G$ be its complementary closed immersion.
Then the derived category of lisse $\Lambda$-sheaves on $\Bun_G$ admits a recollement diagram with respect to
$(i,j)$.
Moreover, on bounded opens $\Bun_G^{\leq \mu}$ this recollement can be described by gluing sheaves along the
Harder--Narasimhan strata.
\end{proposition}

\begin{proof}[Sketch of proof]
The gluing formalism along Harder--Narasimhan strata is developed in detail by Miles for prime-to-$p$ torsion
coefficients \cite{MilesGluingHN}, building on the general sheaf theory on v-stacks in \cite{FSGeometrization}.
\end{proof}

\begin{remark}
For $\Lambda=\Z_\ell$ or $\Q_\ell$, one expects the same recollement statements after a suitable finiteness and
completeness discussion.  In the blueprint approach of this paper, we will systematically reduce compactness
questions to bounded opens where the torsion theory and finiteness statements are available.
\end{remark}

\subsection{Strata as representation theory}

The description $\Bun_G^b\simeq B\underline{J_b(E)}$ has an immediate representation-theoretic consequence:
lisse $\ell$-adic sheaves on a classifying stack are the same as smooth representations of the corresponding
locally profinite group.

\begin{proposition}[Sheaves on strata and smooth representations]\label{prop:sheaves-on-strata}
Let $H$ be a locally profinite group and let $\Lambda$ be a ring in which $p$ is invertible.
Then the derived category of lisse $\Lambda$-sheaves on $B\underline{H}$ is equivalent to the derived category
of smooth $\Lambda$-representations of $H$.
In particular, for each $b\in B(G)$ one has a canonical equivalence
\[
  D_{\mathrm{lis}}(\Bun_G^b,\Lambda)\ \simeq\ D\big(\Rep^{\infty}_\Lambda(J_b(E))\big).
\]
\end{proposition}

\begin{proof}[Sketch of proof]
A lisse sheaf on $B\underline{H}$ is a locally constant sheaf on the classifying topos, which amounts to a
continuous action of $H$ on a discrete $\Lambda$-module; this is equivalent to a smooth representation.
The derived equivalence is obtained by passing to complexes.  This is the dictionary used throughout
\cite{FSGeometrization}.
\end{proof}

\begin{remark}[Why $\Bun_G$ is still interesting]
Even though $\Bun_G$ is ``discrete up to automorphisms'' (a disjoint union of classifying stacks), the Hecke
correspondences between different strata, and the functoriality of sheaves with respect to these correspondences,
encode deep representation theory.  All categorical phenomena in the conjecture are driven by these
correspondences, not by moduli within a fixed stratum.
\end{remark}




\section{Automorphic categories: $\ell$-adic sheaves on $\Bun_G$}\label{sec:automorphic}

In this section we fix the automorphic category on the Fargues--Fontaine side:
the derived category of lisse $\ell$-adic sheaves on the v-stack $\Bun_G$.
We record the formal properties that will be used in later sections:
\begin{enumerate}[label=(\roman*), leftmargin=2em]
  \item a precise definition of $D_{\mathrm{lis}}(X,\Lambda)$ for Artin v-stacks $X$ (in particular $X=\Bun_G$),
  \item the functoriality needed for correspondences (pullback, pushforward, exceptional functors),
  \item compact generation and compact objects with bounded support,
  \item restriction to Harder--Narasimhan strata and the resulting representation-theoretic dictionary.
\end{enumerate}
All of these structures are developed in \cite{FSGeometrization} and used systematically in
\cite{FSReview,HHSGeometricEis}.  We present them here in a way that is adapted to the Gaitsgory-style argument
in Section~\ref{sec:proof-strategy}.

\subsection{Sheaves on diamonds and $\ell$-adic coefficients}

Let $X$ be a locally spatial diamond.  For a coefficient ring $\Lambda$ with $\ell$ invertible on $X$
(for us $\ell\neq p$), one has the derived $\infty$-category $D(X,\Lambda)$ of sheaves of $\Lambda$-modules on the
pro-\'etale site of $X$ (equivalently, on the v-site; see \cite[\S I.2--I.3]{FSGeometrization}).

\begin{definition}[Lisse sheaves on a diamond]\label{def:lisse-diamond}
A sheaf $\mathcal{F}$ of $\Lambda$-modules on $X$ is \emph{lisse} if it is locally constant on the pro-\'etale
site and its stalks are finite projective $\Lambda$-modules (for torsion $\Lambda$, ``finite projective'' means
finite of constant rank).
Write $D_{\mathrm{lis}}(X,\Lambda)\subset D(X,\Lambda)$ for the full subcategory consisting of complexes whose
cohomology sheaves are lisse.
\end{definition}

\begin{remark}[Torsion, integral, and rational coefficients]
When $\Lambda=\mathbb{Z}/\ell^n\mathbb{Z}$, the category $D_{\mathrm{lis}}(X,\Lambda)$ is defined directly.
For $\Lambda=\mathbb{Z}_\ell$ we adopt the usual derived $\ell$-adic formalism:
\[
  D_{\mathrm{lis}}(X,\mathbb{Z}_\ell)\ :=\ \varprojlim_{n} D_{\mathrm{lis}}(X,\mathbb{Z}/\ell^n\mathbb{Z}),
\]
interpreted as an $\ell$-adically complete derived category (as in \cite[\S I.2]{FSGeometrization}).
For $\Lambda=\mathbb{Q}_\ell$ one may define
$D_{\mathrm{lis}}(X,\mathbb{Q}_\ell):=D_{\mathrm{lis}}(X,\mathbb{Z}_\ell)\otimes_{\mathbb{Z}_\ell}\mathbb{Q}_\ell$.
All later constructions are compatible with these changes of coefficients.
\end{remark}

\subsection{Six functors for maps of diamonds}

Let $f:X\to Y$ be a morphism of locally spatial diamonds.  The sheaf theory provides:
\begin{itemize}[leftmargin=2em]
  \item pullback $f^\ast:D(Y,\Lambda)\to D(X,\Lambda)$ and pushforward $f_\ast:D(X,\Lambda)\to D(Y,\Lambda)$;
  \item if $f$ is compactifiable (for example partially proper in the sense of \cite[\S I.1]{FSGeometrization}),
  an exceptional pushforward $f_!:D(X,\Lambda)\to D(Y,\Lambda)$;
  \item if $f$ is representable in locally spatial diamonds and locally of finite cohomological dimension,
  an exceptional pullback $f^!:D(Y,\Lambda)\to D(X,\Lambda)$;
  \item tensor products, internal Hom, and Verdier duality on suitable subcategories.
\end{itemize}
We will only use these functors in the range where they are constructed and shown to satisfy the usual
formal properties (base change, projection formula, adjunctions), as developed in
\cite[\S I.2--I.5]{FSGeometrization} and refined for parabolic correspondences in \cite{HHSGeometricEis}.

\begin{proposition}[Functoriality needed for correspondences]\label{prop:six-functors}
Let $f:X\to Y$ be a morphism of locally spatial diamonds.
\begin{enumerate}[label=(\alph*), leftmargin=2em]
  \item The functors $f^\ast$ and $f_\ast$ preserve $D_{\mathrm{lis}}$.
  \item If $f$ is compactifiable, then $f_!$ is defined on $D_{\mathrm{lis}}(X,\Lambda)$ and preserves
  $D_{\mathrm{lis}}$.
  \item If $f$ is representable in locally spatial diamonds and cohomologically smooth of relative dimension $d$,
  then $f^!$ is defined and satisfies
  \[
    f^!(\mathcal{G}) \simeq f^\ast(\mathcal{G})(d)[2d]
  \]
  on $D_{\mathrm{lis}}(Y,\Lambda)$ (with the usual Tate twist and shift convention).
\end{enumerate}
\end{proposition}

\begin{proof}[Sketch of proof]
These statements are standard in the $\ell$-adic sheaf theory on diamonds for $\ell\neq p$ and are proved in the
setup of \cite{FSGeometrization}.  Cohomological smoothness is the diamond analogue of smoothness that guarantees
the existence and explicit form of $f^!$; it is discussed systematically in \cite[\S I.2]{FSGeometrization}.
\end{proof}

\subsection{From diamonds to Artin v-stacks}

The stack $\Bun_G$ is an Artin v-stack locally of diamond type.  One defines sheaf categories on such stacks by
descent along a smooth atlas.

\begin{definition}[Lisse sheaves on an Artin v-stack]\label{def:lisse-vstack}
Let $X$ be an Artin v-stack locally of diamond type.
Choose a smooth surjective atlas $p:U\to X$ by a locally spatial diamond $U$.
Let $U_\bullet$ denote the associated Cech nerve, a simplicial diamond.
Define
\[
  D_{\mathrm{lis}}(X,\Lambda)\ :=\ \varprojlim_{\Delta} D_{\mathrm{lis}}(U_\bullet,\Lambda),
\]
the limit taken in stable presentable $\infty$-categories.
This category is independent of the choice of atlas and has the expected functoriality with respect to morphisms
of Artin v-stacks (see \cite[\S I.3--I.4]{FSGeometrization}).
\end{definition}

\begin{remark}
All categories in this paper are stable presentable $\infty$-categories.
When we write an ``equivalence of categories'', we always mean an equivalence in this sense.
\end{remark}

\subsection{The automorphic category on \texorpdfstring{$\Bun_G$}{BunG}}

Applying Definition~\ref{def:lisse-vstack} to $X=\Bun_G$ yields the automorphic category
$D_{\mathrm{lis}}(\Bun_G,\Lambda)$.
We will frequently work with objects supported on bounded opens in the sense of
Definition~\ref{def:bounded-opens}.

\begin{definition}[Subcategories with bounded support]\label{def:bounded-support-cat}
Let $j:\Bun_G^{\leq\mu}\hookrightarrow \Bun_G$ be a bounded open immersion.
Define $D_{\mathrm{lis}}(\Bun_G,\Lambda)_{\leq \mu}$ to be the full subcategory of
$D_{\mathrm{lis}}(\Bun_G,\Lambda)$ consisting of objects whose restriction to
$\Bun_G\setminus \Bun_G^{\leq\mu}$ vanishes.
Equivalently,
\[
  D_{\mathrm{lis}}(\Bun_G,\Lambda)_{\leq \mu}\ \simeq\ D_{\mathrm{lis}}(\Bun_G^{\leq\mu},\Lambda)
\]
via the restriction functor $j^\ast$ with inverse given by extension by zero $j_!$.
\end{definition}

\begin{proposition}[Exhaustion by bounded opens]\label{prop:exhaustion}
The v-stack $\Bun_G$ is the increasing union of bounded opens $\Bun_G^{\leq \mu}$, and the category
$D_{\mathrm{lis}}(\Bun_G,\Lambda)$ is the filtered colimit of the subcategories with bounded support:
\[
  D_{\mathrm{lis}}(\Bun_G,\Lambda)\ \simeq\ \varinjlim_{\mu}\, D_{\mathrm{lis}}(\Bun_G,\Lambda)_{\leq \mu}.
\]
\end{proposition}

\begin{proof}[Sketch of proof]
The union statement is a reformulation of the Harder--Narasimhan stratification:
every point lies in some finite union of strata, hence in some bounded open.
The colimit statement follows because lisse sheaves satisfy v-descent and extension by zero along open immersions
is compatible with filtered unions of opens; see \cite[\S I.4]{FSGeometrization}.
\end{proof}

\subsection{Compact objects and compact generation}

To run Barr--Beck type arguments later, we need a manageable supply of compact objects.
For $\ell\neq p$ and for bounded opens, compact objects behave as expected.

\begin{definition}[Compact objects with bounded support]\label{def:compact-bounded}
Let $\mu$ be a bound.  An object $\mathcal{F}\in D_{\mathrm{lis}}(\Bun_G,\Lambda)_{\leq \mu}$ is called
\emph{compact} if $\Hom(\mathcal{F},-)$ commutes with filtered colimits in the subcategory
$D_{\mathrm{lis}}(\Bun_G,\Lambda)_{\leq \mu}$.
Write $D_{\mathrm{lis}}(\Bun_G,\Lambda)_{\leq \mu}^{\mathrm{c}}$ for the full subcategory of compact objects.
\end{definition}

\begin{proposition}[Compact generation on bounded opens]\label{prop:compactgen-bounded}
For each bound $\mu$, the category $D_{\mathrm{lis}}(\Bun_G,\Lambda)_{\leq \mu}$ is compactly generated.
Moreover, for torsion coefficients $\Lambda=\mathbb{Z}/\ell^n\mathbb{Z}$, the compact objects are precisely the
objects whose cohomology sheaves are lisse and constructible and whose support is contained in a finite union of
Harder--Narasimhan strata inside $\Bun_G^{\leq \mu}$.
\end{proposition}

\begin{proof}[Sketch of proof]
For torsion coefficients, compact generation for lisse sheaves on quasi-compact Artin v-stacks follows from the
general formalism in \cite[\S I.4--I.5]{FSGeometrization}.  One reduces to the case of a finite union of strata,
where each stratum is a classifying stack of a locally profinite group (Proposition~\ref{prop:BunG-stratum-BJ})
and compactness can be checked on the corresponding representation category.
The constructibility characterization is the diamond analogue of the usual fact that, for a quasi-compact and
quasi-separated space, compact objects in the derived category of sheaves are the constructible ones.
\end{proof}

\begin{remark}[Integral and rational coefficients]
For $\Lambda=\mathbb{Z}_\ell$ or $\mathbb{Q}_\ell$, the correct notion of compactness involves derived
$\ell$-adic completeness and boundedness conditions.  In practice, we will reduce compactness statements to the
torsion level and then pass to limits; this is the strategy used in \cite{HHSGeometricEis}.
\end{remark}

\subsection{Restriction to strata and the representation-theoretic dictionary}

Let $i_b:\Bun_G^b\hookrightarrow \Bun_G$ be the locally closed immersion of a Harder--Narasimhan stratum.
By Proposition~\ref{prop:BunG-stratum-BJ} we may identify $\Bun_G^b$ with the classifying v-stack
$B\underline{J_b(E)}$.

\begin{proposition}[Restriction to strata]\label{prop:restrict-strata}
For each $b\in B(G)$, restriction along $i_b$ defines functors
\[
  i_b^\ast,\ i_b^!:\ D_{\mathrm{lis}}(\Bun_G,\Lambda)\ \longrightarrow\ D_{\mathrm{lis}}(\Bun_G^b,\Lambda),
\]
and there is a canonical identification
\[
  D_{\mathrm{lis}}(\Bun_G^b,\Lambda)\ \simeq\ D\big(\Rep^\infty_\Lambda(J_b(E))\big),
\]
where $\Rep^\infty_\Lambda(J_b(E))$ denotes the abelian category of smooth $\Lambda$-representations of $J_b(E)$.
\end{proposition}

\begin{proof}[Sketch of proof]
The existence of $i_b^\ast$ and $i_b^!$ follows from the six functor formalism on Artin v-stacks
(Definition~\ref{def:lisse-vstack} and Proposition~\ref{prop:six-functors}).
The identification with smooth representations is Proposition~\ref{prop:sheaves-on-strata}.
\end{proof}

\begin{remark}[Why global geometry matters]
If $\Bun_G$ were merely a disjoint union of its strata with no additional structure, then
$D_{\mathrm{lis}}(\Bun_G,\Lambda)$ would be a product of representation categories of the groups $J_b(E)$.
The geometric content of the Fargues--Scholze program lies in the \emph{correspondences} between strata
(Hecke operators, Eisenstein series, constant term), which couple these representation-theoretic fibers.
This coupling is what makes a geometric Langlands-type spectral description possible.
\end{remark}

\subsection{Recollement and gluing along Harder--Narasimhan strata}

For later inductive arguments we will need to glue objects from their restrictions to bounded unions of strata.

\begin{proposition}[Gluing on bounded opens for torsion coefficients]\label{prop:gluing-bounded}
Assume $\Lambda$ is torsion of characteristic $\ell\neq p$.
Let $\Bun_G^{\leq\mu}$ be a bounded open.  Then objects of $D_{\mathrm{lis}}(\Bun_G^{\leq\mu},\Lambda)$ can be
reconstructed from their restrictions to the finitely many strata contained in $\Bun_G^{\leq\mu}$, together with
the natural gluing data along the closure relations.
In particular, there is a recollement formalism for the filtration by closed unions of strata.
\end{proposition}

\begin{proof}[Sketch of proof]
This is proved in detail by Miles \cite{MilesGluingHN}, building on the stratified geometry of $\Bun_G$ and the
torsion $\ell$-adic formalism for v-stacks.  The main point is that the closure relations among strata inside a
bounded open are finite and admit cohomological control that allows gluing in the derived category.
\end{proof}

\begin{remark}
Later sections will use Proposition~\ref{prop:gluing-bounded} to reduce compactness and generation statements
to calculations on individual strata, where representation theory can be applied.
\end{remark}

\section{The spectral stack of parameters and nilpotent singular support}\label{sec:spectral}

This section fixes the spectral geometric object that will control the automorphic category via the spectral action:
the stack $\LocSys_{\checkG}$ of $\ell$-adic Langlands parameters, together with the spectral category
$\Coh_{\mathrm{Nilp}}(\LocSys_{\checkG})$ of coherent sheaves with nilpotent singular support.

There are two complementary viewpoints on $\LocSys_{\checkG}$:
\begin{enumerate}[label=(\roman*), leftmargin=2em]
  \item the v-stack of parameters constructed by Fargues and Scholze in the course of defining the spectral action
  \cite{FSGeometrization,FSReview};
  \item an algebraic model for the moduli of parameters constructed by Dat, Helm, Kurinczuk, and Moss
  \cite{DHKMParameters}.
\end{enumerate}
For the blueprint of this paper, it is convenient to work with the algebraic model of \cite{DHKMParameters}
(because it supports the use of derived singular support), while keeping the link to the spectral action of
\cite{FSGeometrization} in mind.

\subsection{Weil--Deligne parameters and Langlands parameters}\label{subsec:WD}

Let $W_E$ denote the Weil group of $E$, and let $I_E\subset W_E$ be the inertia subgroup.
Write ${}^LG$ for the $L$-group of $G$, defined over $\Lambda$:
\[
  {}^LG \;=\; \checkG \rtimes W_E.
\]
We recall one standard model for Langlands parameters at $\ell\neq p$.

\begin{definition}[Weil--Deligne parameter]\label{def:WD-parameter}
A \emph{Weil--Deligne parameter} (with coefficients in a $\Lambda$-algebra $A$) is a pair $(r,N)$ where
\begin{itemize}[leftmargin=2em]
  \item $r:W_E\to {}^LG(A)$ is a continuous homomorphism such that the composite $W_E\xrightarrow{r}{}^LG(A)\to W_E$
  is the identity and $r(I_E)$ has finite image;
  \item $N\in \Lie(\checkG)\otimes_\Lambda A$ is nilpotent and satisfies the usual equivariance relation
  $r(w)Nr(w)^{-1}=|w|\,N$ for $w\in W_E$.
\end{itemize}
Two parameters are identified if they are conjugate by $\checkG(A)$.
\end{definition}

\begin{remark}
There are equivalent formulations (for example, via homomorphisms $W_E\times \mathrm{SL}_2\to {}^LG$).
For the purposes of this paper, the particular model is not essential, provided one has an algebraic moduli
problem with good finiteness properties.
\end{remark}

\subsection{The Dat--Helm--Kurinczuk--Moss moduli stack}\label{subsec:DHKM}

The foundational input on the spectral side is that Weil--Deligne parameters form an algebraic stack with good
geometric properties.

\begin{theorem}[Dat--Helm--Kurinczuk--Moss]\label{thm:DHKM}
There exists an algebraic stack $\mathcal{X}_{\checkG}$ locally of finite type over $\Spec(\Z[1/p])$
whose $A$-points classify $\checkG$-valued Langlands parameters (equivalently, Weil--Deligne parameters)
with coefficients in $A$.
Moreover, $\mathcal{X}_{\checkG}$ has strong finiteness properties (in particular, it is a reduced
local complete intersection stack in the sense of \cite{DHKMParameters}).
\end{theorem}

\begin{proof}[Proof sketch]
This is the main result of \cite{DHKMParameters}.  The construction is by a careful algebraization of the
representation-theoretic moduli problem, together with a deformation-theoretic analysis that produces the
local complete intersection structure.
\end{proof}

\begin{definition}[Spectral parameter stack]\label{def:LocSys-spectral}
Fix $\Lambda$ as in Section~\ref{sec:notation}.  We define
\[
  \LocSys_{\checkG}\ :=\ \mathcal{X}_{\checkG}\times_{\Spec(\Z[1/p])}\Spec(\Lambda).
\]
We will freely use the same notation for a chosen derived enhancement of this stack
(Definition~\ref{def:LocSys-derived} below).
\end{definition}

\begin{remark}[Comparison with Fargues--Scholze]
Fargues and Scholze construct a stack of $\ell$-adic parameters and show that $\Perf(\LocSys_{\checkG})$ acts on
$D_{\mathrm{lis}}(\Bun_G,\Lambda)$ \cite{FSGeometrization}.  In this paper we view
Definition~\ref{def:LocSys-spectral} as an algebraic incarnation of the same spectral stack, suitable for the
use of singular support.
Making the comparison between these constructions precise is an important foundational point, but it is not
needed for the formal argument template presented here.
\end{remark}

\subsection{Derived enhancement and deformation theory}\label{subsec:derived-enhancement}

Singular support for ind-coherent sheaves requires a quasi-smooth derived stack.  We therefore fix a derived
enhancement of $\LocSys_{\checkG}$ compatible with the local complete intersection structure.

\begin{definition}[Derived enhancement]\label{def:LocSys-derived}
Let $\LocSys_{\checkG}^{\mathrm{cl}}$ denote the classical stack of Definition~\ref{def:LocSys-spectral}.
A \emph{derived enhancement} of $\LocSys_{\checkG}^{\mathrm{cl}}$ is a derived stack
$\LocSys_{\checkG}$ whose truncation is $\LocSys_{\checkG}^{\mathrm{cl}}$ and whose cotangent complex is
perfect of Tor-amplitude in $[-1,0]$ (equivalently, $\LocSys_{\checkG}$ is quasi-smooth).
\end{definition}

\begin{axiom}[Quasi-smoothness]\label{ax:LocSys-quasi-smooth}
We fix a quasi-smooth derived enhancement of $\LocSys_{\checkG}^{\mathrm{cl}}$.
\end{axiom}

\begin{remark}[Why this is reasonable]
Theorem~\ref{thm:DHKM} asserts that $\LocSys_{\checkG}^{\mathrm{cl}}$ is a local complete intersection stack.
In derived algebraic geometry, local complete intersection classical stacks admit natural quasi-smooth derived
enhancements, and the deformation theory at a point is governed by a two-term complex.
In later versions of this paper, Axiom~\ref{ax:LocSys-quasi-smooth} will be replaced by a precise reference
to the derived enhancement implicit in the local complete intersection statement of \cite{DHKMParameters}.
\end{remark}

At a point $\phi\in \LocSys_{\checkG}$ represented by a parameter, the deformation theory is controlled by
group cohomology with coefficients in the adjoint local system $\ad(\phi)$.

\begin{proposition}[Tangent complex at a parameter]\label{prop:tangent-complex}
Let $\phi\in \LocSys_{\checkG}$ be a geometric point corresponding to a Langlands parameter.
Then the shifted tangent complex of $\LocSys_{\checkG}$ at $\phi$ is canonically identified with a complex
computing continuous group cohomology of $W_E$ with coefficients in $\ad(\phi)$:
\[
  \mathbb{T}_{\LocSys_{\checkG},\phi}\ \simeq\ R\Gamma\big(W_E,\ad(\phi)\big)[1],
\]
and similarly the cotangent complex is given (up to the same shift) by $R\Gamma(W_E,\ad(\phi)^\vee)$.
In particular:
\begin{itemize}[leftmargin=2em]
  \item $H^{-1}(\mathbb{T}_{\LocSys_{\checkG},\phi})$ is the Lie algebra of the infinitesimal automorphism group,
  that is, the Lie algebra of the centralizer of $\phi$;
  \item $H^{0}(\mathbb{T}_{\LocSys_{\checkG},\phi})$ controls first-order deformations of $\phi$;
  \item $H^{1}(\mathbb{T}_{\LocSys_{\checkG},\phi})$ controls obstruction classes.
\end{itemize}
\end{proposition}

\begin{proof}[Proof sketch]
This is the standard deformation theory of representation stacks: the deformation complex is the
continuous cochain complex of $W_E$ with coefficients in the adjoint representation associated to $\phi$,
shifted by one.  In the algebraic setting of \cite{DHKMParameters}, this deformation theory is built into the
construction of the moduli stack and its local complete intersection structure.
\end{proof}

\subsection{Ind-coherent sheaves and singular support}\label{subsec:singular-support}

Let $X$ be a quasi-smooth derived stack locally of finite type over $\Spec(\Lambda)$.
Arinkin and Gaitsgory attach to $X$ a classical stack $\mathrm{Sing}(X)$, called the \emph{singularity stack}
(or the stack of singularities), and define for each $\mathcal{F}\in \IndCoh(X)$ its \emph{singular support}
\[
  \mathrm{SS}(\mathcal{F})\ \subset\ \mathrm{Sing}(X),
\]
a conical Zariski-closed subset \cite{AGSingSupp}.

\begin{definition}[Singularity stack]\label{def:SingX}
Let $X$ be quasi-smooth.  Write $X^{\mathrm{cl}}$ for its classical truncation.
The singularity stack $\mathrm{Sing}(X)$ is the relative spectrum
\[
  \mathrm{Sing}(X)\ :=\ \underline{\Spec}_{X^{\mathrm{cl}}}\Big(\mathrm{Sym}_{\OO_{X^{\mathrm{cl}}}}
  \big(H^{1}(\mathbb{T}_X)\big)\Big),
\]
which is a classical conical stack over $X^{\mathrm{cl}}$.
\end{definition}

\begin{remark}
For $X=\LocSys_{\checkG}$, Proposition~\ref{prop:tangent-complex} identifies the fibers of $H^{1}(\mathbb{T}_X)$
with obstruction spaces for deformations of Langlands parameters.  Singular support measures in which
obstruction directions an ind-coherent sheaf is allowed to have singularities.
\end{remark}

\subsection{The nilpotent cone and nilpotent singular support}\label{subsec:nilpotent}

Let $\mathcal{N}\subset \Lie(\checkG)^\ast$ denote the nilpotent cone, viewed as a conical $\checkG$-stable closed
subset.  Following \cite{AGSingSupp} and the discussion in \cite[\S 6]{FSReview}, the nilpotent singular support
condition for $\LocSys_{\checkG}$ is defined by pulling back $\mathcal{N}$ along the natural map from the
singularity stack to the adjoint quotient.

\begin{axiom}[Nilpotent cone map]\label{ax:nilpotent-map}
There exists a canonical morphism of conical stacks
\[
  \chi:\ \mathrm{Sing}(\LocSys_{\checkG})\ \longrightarrow\ [\Lie(\checkG)^\ast/\checkG],
\]
functorial in $G$, such that the nilpotent cone condition is defined by the inverse image
\[
  \mathrm{Nilp}(\LocSys_{\checkG})\ :=\ \chi^{-1}\big([\mathcal{N}/\checkG]\big)\ \subset\ \mathrm{Sing}(\LocSys_{\checkG}).
\]
\end{axiom}

\begin{remark}
For global geometric Langlands on a curve, the morphism $\chi$ is the (global) Hitchin map on the singularity
stack, and nilpotent singular support is the condition that the corresponding Higgs field is nilpotent.
In the present local setting, Axiom~\ref{ax:nilpotent-map} should be viewed as the direct analogue: it is the
mechanism that cuts the spectral category down to the ``Arthur-nilpotent'' range required by Whittaker and
Eisenstein compatibility \cite{AGSingSupp,FSReview}.
\end{remark}

\subsection{The category $\Coh_{\mathrm{Nilp}}(\LocSys_{\checkG})$}\label{subsec:CohNilp}

We now define the spectral category that is predicted to match the Whittaker-generated automorphic category.

\begin{definition}[Ind-coherent sheaves with nilpotent singular support]\label{def:IndCohNilp}
Let $X$ be quasi-smooth, and let $\mathcal{Y}\subset \mathrm{Sing}(X)$ be a closed conical substack.
Define $\IndCoh_{\mathcal{Y}}(X)\subset \IndCoh(X)$ to be the full subcategory of objects whose singular support
is contained in $\mathcal{Y}$:
\[
  \IndCoh_{\mathcal{Y}}(X)\ :=\ \{\mathcal{F}\in \IndCoh(X)\mid \mathrm{SS}(\mathcal{F})\subset \mathcal{Y}\}.
\]
For $X=\LocSys_{\checkG}$ and $\mathcal{Y}=\mathrm{Nilp}(\LocSys_{\checkG})$ we write
$\IndCoh_{\mathrm{Nilp}}(\LocSys_{\checkG})$.
\end{definition}

\begin{definition}[Coherent objects with nilpotent singular support]\label{def:CohNilp}
Let $\Coh_{\mathrm{Nilp}}(\LocSys_{\checkG})$ be the full subcategory of compact objects in
$\IndCoh_{\mathrm{Nilp}}(\LocSys_{\checkG})$ (equivalently, the coherent objects of ind-coherent sheaves with
nilpotent singular support).
\end{definition}

\begin{proposition}[Basic properties]\label{prop:CohNilp-basic}
Assume Axioms~\ref{ax:LocSys-quasi-smooth} and \ref{ax:nilpotent-map}.
\begin{enumerate}[label=(\alph*), leftmargin=2em]
  \item The subcategory $\IndCoh_{\mathrm{Nilp}}(\LocSys_{\checkG})$ is stable under colimits and under the action
  of $\QCoh(\LocSys_{\checkG})$ by tensor product.
  \item Every perfect complex on $\LocSys_{\checkG}$ has singular support contained in the zero section, hence
  belongs to $\Coh_{\mathrm{Nilp}}(\LocSys_{\checkG})$.
  \item If $\Lambda=\Ql$ (or $\Ql[\sqrt{q}]$), then the nilpotent singular support condition is expected to be
  automatic, in the sense that $\Coh_{\mathrm{Nilp}}(\LocSys_{\checkG})=\Coh(\LocSys_{\checkG})$.
\end{enumerate}
\end{proposition}

\begin{proof}[Proof sketch]
Parts (a) and (b) are general properties of singular support for ind-coherent sheaves on quasi-smooth stacks
\cite{AGSingSupp}.  The expectation in (c) is explained in \cite[\S 6]{FSReview}: over $\Ql$ the integral
correction imposed by nilpotent singular support should disappear.
\end{proof}

\begin{remark}[Why the integral nilpotent condition matters]
Even if (c) holds, the integral category $\Coh_{\mathrm{Nilp}}(\LocSys_{\checkG})$ is the natural object for
categorical local Langlands: it refines the spectral action integrally and keeps track of congruences and
integral structures that are invisible over $\Ql$.
This parallels the role of nilpotent singular support in integral formulations of geometric Langlands.
\end{remark}


\section{Hecke correspondences, geometric Satake, and the spectral action}\label{sec:hecke-satake}

In this section we recall the geometric Satake formalism over the Fargues--Fontaine curve and the resulting
Hecke and spectral actions on the automorphic category $D_{\mathrm{lis}}(\Bun_G,\Lambda)$.
This is the principal mechanism that couples the different Harder--Narasimhan strata of $\Bun_G$ and makes the
category sensitive to Langlands parameters.

There are three layers of structure:
\begin{enumerate}[label=(\roman*), leftmargin=2em]
  \item the \emph{Hecke correspondence} on $\Bun_G$ defined using modifications at the untilt point;
  \item the \emph{spherical Hecke category} (the Satake category) living on the $B_{\mathrm{dR}}^+$-affine
  Grassmannian, together with the geometric Satake equivalence;
  \item the \emph{spectral action} of $\Perf(\LocSys_{\checkG})$ on $D_{\mathrm{lis}}(\Bun_G,\Lambda)$, which
  extends the Hecke action and will be used to define the comparison functor in
  Section~\ref{sec:whittaker}.
\end{enumerate}

\subsection{The $B_{\mathrm{dR}}^+$-affine Grassmannian}\label{subsec:BdR-Grassmannian}

Let $C$ be an algebraically closed complete extension of $\breve{E}$.
Write $B_{\mathrm{dR}}^+(C)$ and $B_{\mathrm{dR}}(C)$ for Fontaine's de Rham period rings, and let
$B_{\mathrm{dR}}^+(C)\subset B_{\mathrm{dR}}(C)$ denote the canonical inclusion.
The $B_{\mathrm{dR}}^+$-affine Grassmannian of $G$ is the v-sheaf (in fact a diamond) whose $C$-points are
the double coset space
\[
  \mathrm{Gr}_G(C)\ :=\ G\big(B_{\mathrm{dR}}(C)\big)\big/ G\big(B_{\mathrm{dR}}^+(C)\big),
\]
and which is functorial in $C$.
Equivalently, $\mathrm{Gr}_G$ classifies modifications of the trivial $G$-bundle on the punctured untilt
disc with a trivialization on the formal disc.

\begin{definition}[$B_{\mathrm{dR}}^+$-affine Grassmannian]\label{def:GrG}
Let $\mathrm{Gr}_G$ denote the $B_{\mathrm{dR}}^+$-affine Grassmannian diamond of $G$ over $\Spa(\breve{E})$
defined in \cite[\S I.3--I.6]{FSGeometrization}.
\end{definition}

\begin{remark}
The diamond $\mathrm{Gr}_G$ admits a stratification by Schubert cells indexed by dominant cocharacters.
Many geometric features formally parallel those of the usual affine Grassmannian over a curve, but the
ambient geometry is mixed characteristic and the stratification is described in the language of diamonds.
\end{remark}

\subsection{The Hecke correspondence on \texorpdfstring{$\Bun_G$}{BunG}}\label{subsec:Hecke-correspondence}

Fix a untilt point on the relative Fargues--Fontaine curve (equivalently, work with the global version of the
Hecke correspondence defined in \cite[\S I.6]{FSGeometrization}).
Informally, the Hecke correspondence parameterizes pairs of $G$-bundles together with an isomorphism away from
the untilt point.  Concretely:

\begin{definition}[Hecke stack]\label{def:Hecke-stack}
Let $\Hecke_G$ be the v-stack classifying triples $(\mathcal{E}_1,\mathcal{E}_2,\beta)$ where
$\mathcal{E}_1,\mathcal{E}_2\in \Bun_G$ and $\beta$ is an isomorphism between $\mathcal{E}_1$ and $\mathcal{E}_2$
over the complement of the untilt point (equivalently, over the punctured formal disc).
There are natural morphisms
\[
  p_1,p_2:\Hecke_G\to \Bun_G,\qquad p_i(\mathcal{E}_1,\mathcal{E}_2,\beta)=\mathcal{E}_i.
\]
\end{definition}

The morphism $(p_1,p_2):\Hecke_G\to \Bun_G\times \Bun_G$ is representable in diamonds and is locally modeled on
the correspondence induced by $\mathrm{Gr}_G$.

\begin{remark}[Bounded Hecke correspondences]
For a dominant cocharacter $\mu$, one defines the closed sub-v-stack $\Hecke_G^{\leq \mu}\subset \Hecke_G$
cut out by the condition that the relative position of the modification lies in the Schubert closure of $\mu$.
These bounded Hecke correspondences are the geometric inputs used to define Hecke operators attached to
representations of $\checkG$.
\end{remark}

\subsection{The spherical Hecke category and convolution}\label{subsec:Satake-category}

Let $\Lambda$ be as in Section~\ref{sec:notation}, with $\ell\neq p$.
The geometric Satake formalism uses an $\ell$-adic sheaf category on $\mathrm{Gr}_G$ that is equivariant under the
loop group $L^+G$ (defined using $B_{\mathrm{dR}}^+$).  Fargues and Scholze define a stable monoidal category
$\Sat_{G,\Lambda}$ (the spherical Hecke category) whose monoidal structure is given by convolution.

\begin{definition}[Spherical Hecke category]\label{def:Satake}
Let $\Sat_{G,\Lambda}$ denote the full subcategory of $D_{\mathrm{lis}}(\mathrm{Gr}_G,\Lambda)$ consisting of
$L^+G$-equivariant objects satisfying the finiteness conditions that make convolution well-defined.
The convolution product $\star$ is defined using the usual correspondence
\[
  \mathrm{Gr}_G \times \mathrm{Gr}_G \xleftarrow{\ \mathrm{pr}\ } \widetilde{\mathrm{Gr}}_G
  \xrightarrow{\ m\ } \mathrm{Gr}_G
\]
and is associative up to canonical equivalence.
\end{definition}

\begin{remark}
For the blueprint of this paper it is not important whether one formulates $\Sat_{G,\Lambda}$ using perverse
sheaves, constructible complexes, or the larger lisse derived category with suitable compactness constraints.
What matters is that:
\begin{itemize}[leftmargin=2em]
  \item convolution is defined on a monoidal subcategory,
  \item the unit is the delta sheaf at the base point,
  \item the geometric Satake equivalence identifies this monoidal category with $\Rep(\checkG)$.
\end{itemize}
All of this is established in \cite[\S I.6]{FSGeometrization}.
\end{remark}

\subsection{Geometric Satake over the Fargues--Fontaine curve}\label{subsec:Geometric-Satake}

\begin{theorem}[Fargues--Scholze geometric Satake]\label{thm:FS-satake}
There is a canonical equivalence of symmetric monoidal categories
\[
  \Sat_{G,\Lambda}\ \xrightarrow{\ \sim\ }\ \Rep_{\Lambda}(\checkG),
\]
compatible with standard normalizations (Tate twists and cohomological shifts) as in
\cite[\S I.6]{FSGeometrization} and \cite[\S 4]{FSReview}.
\end{theorem}

\begin{proof}[Proof sketch]
This is proved in \cite[\S I.6]{FSGeometrization}.
The argument follows the geometric Satake strategy for a curve, replacing the usual affine Grassmannian by the
$B_{\mathrm{dR}}^+$-affine Grassmannian and using the untilt point to define the local geometry of
modifications.  The key inputs are:
\begin{itemize}[leftmargin=2em]
  \item the stratification of $\mathrm{Gr}_G$ by Schubert cells and the geometry of their closures,
  \item semismallness and purity properties needed for the Tannakian reconstruction,
  \item compatibility with convolution and duality.
\end{itemize}
\end{proof}

\subsection{Hecke operators on \texorpdfstring{$D_{\mathrm{lis}}(\Bun_G,\Lambda)$}{Dlis(BunG)}}\label{subsec:Hecke-functors}

The Hecke stack $\Hecke_G$ produces an action of the spherical Hecke category on $\Bun_G$ by integral transforms.

\begin{definition}[Hecke action of the Satake category]\label{def:Hecke-action}
Let $\mathcal{S}\in \Sat_{G,\Lambda}$ and let $\mathcal{F}\in D_{\mathrm{lis}}(\Bun_G,\Lambda)$.
Define an object $T_{\mathcal{S}}(\mathcal{F})\in D_{\mathrm{lis}}(\Bun_G,\Lambda)$ by the correspondence
\[
  T_{\mathcal{S}}(\mathcal{F})
  \ :=\
  (p_2)_!\Big(p_1^\ast(\mathcal{F}) \otimes \mathcal{S}_{\Hecke}\Big),
\]
where $\mathcal{S}_{\Hecke}$ is the pullback of $\mathcal{S}$ to $\Hecke_G$ along the local model map
$\Hecke_G\to \mathrm{Gr}_G$ (as in \cite[\S I.6]{FSGeometrization}).
\end{definition}

\begin{proposition}[Convolution compatibility]\label{prop:Hecke-convolution}
The assignment $\mathcal{S}\mapsto T_{\mathcal{S}}$ defines a monoidal functor
\[
  \Sat_{G,\Lambda}\ \longrightarrow\ \mathrm{End}\big(D_{\mathrm{lis}}(\Bun_G,\Lambda)\big),
\]
where the target is the monoidal category of colimit-preserving endofunctors under composition.
Under the geometric Satake equivalence (Theorem~\ref{thm:FS-satake}), this recovers Hecke operators indexed by
representations of $\checkG$.
\end{proposition}

\begin{proof}[Proof sketch]
This is the standard convolution-of-kernels formalism, once one knows the relevant correspondences are
compactifiable and satisfy base change and projection formula statements.
The necessary finiteness for these correspondences is established in \cite[\S I.6]{FSGeometrization}.
\end{proof}

\begin{definition}[Hecke operators indexed by $\Rep(\checkG)$]\label{def:Hecke-V}
For $V\in \Rep_{\Lambda}(\checkG)$, let $\mathcal{S}_V\in \Sat_{G,\Lambda}$ be the corresponding Satake object.
Define the Hecke operator
\[
  T_V\ :=\ T_{\mathcal{S}_V}:\ D_{\mathrm{lis}}(\Bun_G,\Lambda)\to D_{\mathrm{lis}}(\Bun_G,\Lambda).
\]
\end{definition}

\subsection{The universal parameter and perfect complexes on \texorpdfstring{$\LocSys_{\checkG}$}{LocSys}}\label{subsec:universal-parameter}

On the spectral stack $\LocSys_{\checkG}$ there is a tautological ``universal Langlands parameter'' in the sense
that the moduli interpretation produces a universal $W_E$-representation with values in ${}^LG$.
Applying a representation $V\in \Rep(\checkG)$ produces a vector bundle (perfect complex) on $\LocSys_{\checkG}$.

\begin{definition}[The Satake-to-spectral functor]\label{def:Satake-to-spectral}
Let $\mathcal{V}$ denote the vector bundle on $\LocSys_{\checkG}$ associated to
$V\in \Rep_{\Lambda}(\checkG)$ by applying $V$ to the universal parameter.
This gives a symmetric monoidal functor
\[
  \Rep_{\Lambda}(\checkG)\ \longrightarrow\ \Perf(\LocSys_{\checkG}),
  \qquad V\longmapsto \mathcal{V}.
\]
\end{definition}

\begin{remark}
The precise construction depends on which model of $\LocSys_{\checkG}$ one uses (the v-stack model in
\cite{FSGeometrization} or the algebraic model in \cite{DHKMParameters}).  In either case, the point is that
$\LocSys_{\checkG}$ is a moduli stack of parameters, hence carries the tautological parameter.
\end{remark}

\subsection{The spectral action}\label{subsec:spectral-action}

Fargues and Scholze construct a canonical monoidal action
\[
  \Perf(\LocSys_{\checkG})\ \curvearrowright\ D_{\mathrm{lis}}(\Bun_G,\Lambda),
\]
called the \emph{spectral action}.  It is uniquely characterized (up to contractible choices) by the requirement
that it extends the Hecke action in the sense of Definition~\ref{def:Satake-to-spectral} and
Definition~\ref{def:Hecke-V}.

\begin{theorem}[Existence of the spectral action]\label{thm:spectral-action}
There exists a canonical symmetric monoidal functor
\[
  \Perf(\LocSys_{\checkG})\ \longrightarrow\ \mathrm{End}\big(D_{\mathrm{lis}}(\Bun_G,\Lambda)\big)
\]
such that, for every $V\in \Rep_{\Lambda}(\checkG)\subset \Perf(\LocSys_{\checkG})$, the induced endofunctor is
canonically equivalent to the Hecke operator $T_V$.
\end{theorem}

\begin{proof}[Proof sketch]
This is constructed in \cite[\S I.10]{FSGeometrization}.
The key input is that Hecke operators satisfy strong compatibilities (factorization and excursion relations)
which allow one to extend the assignment $V\mapsto T_V$ from representations to perfect complexes on the parameter
stack.  One may view this as a categorified form of the fact that excursion operators generate the spectral
Bernstein center.
\end{proof}

\begin{proposition}[Compatibility with Hecke operators]\label{prop:spectral-vs-hecke}
Let $\mathcal{V}\in \Perf(\LocSys_{\checkG})$ be the vector bundle attached to $V\in \Rep_{\Lambda}(\checkG)$.
Then the action of $\mathcal{V}$ on $D_{\mathrm{lis}}(\Bun_G,\Lambda)$ is canonically equivalent to $T_V$.
\end{proposition}

\begin{proof}[Proof sketch]
This is part of the construction of the spectral action and is stated explicitly in
\cite[\S I.10]{FSGeometrization}.
\end{proof}

\subsection{Restriction to strata and recovery of the usual spherical Hecke action}\label{subsec:Hecke-on-strata}

Let $b\in B(G)$ and identify $\Bun_G^b\simeq B\underline{J_b(E)}$ (Proposition~\ref{prop:BunG-stratum-BJ}).
Restriction to the stratum identifies $D_{\mathrm{lis}}(\Bun_G^b,\Lambda)$ with smooth representations of $J_b(E)$
(Proposition~\ref{prop:restrict-strata}).
Under this identification, bounded Hecke correspondences recover the usual spherical Hecke operators acting on
representation categories.

\begin{proposition}[Hecke operators on basic strata]\label{prop:Hecke-basic-stratum}
Assume $b$ is basic.  Under the equivalence
$D_{\mathrm{lis}}(\Bun_G^b,\Lambda)\simeq D(\Rep^\infty_\Lambda(J_b(E)))$,
the restriction of $T_V$ to $\Bun_G^b$ coincides with the usual (categorified) Hecke operator associated to $V$
acting on smooth representations of the inner form $J_b(E)$.
\end{proposition}

\begin{proof}[Proof sketch]
For basic $b$, the stratum is a classifying stack and the Hecke correspondence restricts to the usual Hecke
correspondence for the locally profinite group $J_b(E)$.
This is explained in \cite[\S I.7--I.8]{FSGeometrization}, where the restriction of the geometrization functor
to basic strata is identified with compact induction and the usual convolution action.
\end{proof}

\subsection{A first finiteness principle for later use}\label{subsec:Hecke-finiteness}

Later sections will require that Hecke operators preserve compactness on bounded opens, and that they interact well
with the Harder--Narasimhan stratification.

\begin{proposition}[Boundedness of Hecke operators on bounded opens]\label{prop:Hecke-boundedness}
Fix a bound $\mu$ on $\Bun_G$ and fix $V\in \Rep_{\Lambda}(\checkG)$.
Then there exists a bound $\mu'$, depending only on $\mu$ and $V$, such that
\[
  T_V\big(D_{\mathrm{lis}}(\Bun_G,\Lambda)_{\leq \mu}\big)\ \subset\ D_{\mathrm{lis}}(\Bun_G,\Lambda)_{\leq \mu'}.
\]
In particular, Hecke operators preserve the filtered colimit presentation of
$D_{\mathrm{lis}}(\Bun_G,\Lambda)$ by bounded support subcategories
(Proposition~\ref{prop:exhaustion}).
\end{proposition}

\begin{proof}[Proof sketch]
The modification type corresponding to $V$ is bounded by a dominant cocharacter $\nu(V)$.
Applying a bounded modification to a bundle with Harder--Narasimhan polygon bounded by $\mu$ yields a bundle with
Harder--Narasimhan polygon bounded by a computable function of $\mu$ and $\nu(V)$.
This is the geometric analogue of the fact that spherical Hecke operators move one between finitely many
Bernstein components.  The required boundedness statements are discussed in
\cite[\S III]{FSGeometrization} and used in \cite{HHSGeometricEis}.
\end{proof}

\begin{remark}
Proposition~\ref{prop:Hecke-boundedness} is one of the places where the discreteness of $B(G)$ and the finiteness
of bounded subsets is essential: it ensures that Hecke operators cannot ``run off to infinity'' in $\Bun_G$ when
restricted to objects with bounded support.
\end{remark}

\section{The Whittaker sheaf and the Whittaker-generated subcategory}\label{sec:whittaker}

The role of this section is to isolate the \emph{Whittaker generator} on the automorphic side and to define the
Whittaker-generated subcategory
\[
  D_{\mathrm{lis}}(\Bun_G,\Lambda)_{\omega}\ \subset\ D_{\mathrm{lis}}(\Bun_G,\Lambda)
\]
that appears in the categorical geometrization conjecture of Fargues and Scholze.
This is the precise analogue of the use of the Whittaker category in geometric Langlands, and it is the entry
point for the Gaitsgory-style strategy: one uses a Whittaker object to define a comparison functor from the
spectral category and then proves that it is fully faithful and essentially surjective by combining
Hecke theory, constant term and Eisenstein functors, and gluing.

Throughout this section we assume that $G$ is \emph{quasi-split} over $E$.

\subsection{Whittaker data}

Fix a Borel subgroup $B\subset G$ defined over $E$ and let $U\subset B$ be its unipotent radical.

\begin{definition}[Whittaker datum]\label{def:whittaker-datum}
A \emph{Whittaker datum} for $G$ (over $E$) is a pair $(B,\psi)$ where $B\subset G$ is a Borel subgroup and
\[
  \psi:U(E)\longrightarrow \Lambda^{\times}
\]
is a \emph{generic} smooth character of the locally profinite group $U(E)$.
\end{definition}

\begin{definition}[Generic character]\label{def:generic-character}
Let $\Delta$ be the set of simple roots determined by $B$ (and a maximal torus $T\subset B$).
A smooth character $\psi:U(E)\to \Lambda^\times$ is called \emph{generic} if, for every simple root
$\alpha\in \Delta$, the restriction of $\psi$ to the corresponding root subgroup $U_{\alpha}(E)\subset U(E)$
is nontrivial.
\end{definition}

\begin{remark}[Coefficient ring for $\psi$]
In practice, to arrange the existence of a generic character with values in $\Lambda^\times$ one often enlarges
$\Lambda$ (for example, by passing to the ring of integers of a sufficiently large algebraic extension of
$\mathbb{Q}_{\ell}$, as in \cite[Conjecture I.10.2]{FSGeometrization}).
For the purposes of this blueprint, we fix such a coefficient ring once and for all and suppress it from the
notation.
\end{remark}

\subsection{The basic stratum and the Whittaker representation}

Let $1\in B(G)$ denote the neutral $\sigma$-conjugacy class; it corresponds to the trivial isocrystal and, under
Fargues' classification, to the \emph{trivial} $G$-bundle on the Fargues--Fontaine curve.
Let $\Bun_G^{1}\subset \Bun_G$ be the corresponding Harder--Narasimhan stratum and let
\[
  i_{1}:\Bun_G^{1}\hookrightarrow \Bun_G
\]
denote the locally closed immersion.

\begin{proposition}[The neutral stratum is a classifying stack]\label{prop:neutral-stratum}
There is a canonical equivalence of v-stacks
\[
  \Bun_G^{1}\ \simeq\ B\underline{G(E)}.
\]
Consequently, there is a canonical identification
\[
  D_{\mathrm{lis}}(\Bun_G^{1},\Lambda)\ \simeq\ D\big(\Rep_{\Lambda}^{\infty}(G(E))\big),
\]
the derived category of smooth $\Lambda$-representations of $G(E)$.
\end{proposition}

\begin{proof}[Sketch of proof]
The equivalence $\Bun_G^{1}\simeq B\underline{G(E)}$ is proved in \cite[\S III.4]{FSGeometrization}.
The identification of lisse sheaves on a classifying v-stack with smooth representations is part of the general
formalism of \cite[\S V.1]{FSGeometrization} and was recalled in
Proposition~\ref{prop:sheaves-on-strata}.
\end{proof}

Now fix Whittaker data $(B,\psi)$ as above.
Let $\mathrm{c}\text{-}\Ind_{U(E)}^{G(E)}\psi$ denote the usual compact induction of smooth representations.

\begin{definition}[The Whittaker representation]\label{def:whittaker-representation}
Let $\mathcal{W}_{\psi}^{\mathrm{rep}}$ be the smooth $\Lambda$-representation of $G(E)$ given by
\[
  \mathcal{W}_{\psi}^{\mathrm{rep}}\ :=\ \mathrm{c}\text{-}\Ind_{U(E)}^{G(E)}\psi.
\]
Via Proposition~\ref{prop:neutral-stratum}, we view $\mathcal{W}_{\psi}^{\mathrm{rep}}$ also as an object
\[
  \mathcal{W}_{\psi}^{\mathrm{str}}\ \in\ D_{\mathrm{lis}}(\Bun_G^{1},\Lambda).
\]
\end{definition}

\subsection{Definition of the Whittaker sheaf on $\Bun_G$}

The Whittaker sheaf is defined by extension by zero from the neutral stratum.

\begin{definition}[Whittaker sheaf]\label{def:whittaker-sheaf}
The \emph{Whittaker sheaf} is
\[
  W_{\psi}\ :=\ (i_{1})_{!}\big(\mathcal{W}_{\psi}^{\mathrm{str}}\big)\ \in\ D_{\mathrm{lis}}(\Bun_G,\Lambda).
\]
Equivalently, $W_{\psi}$ is the object of $D_{\mathrm{lis}}(\Bun_G,\Lambda)$ supported on $\Bun_G^{1}$ whose
restriction to $\Bun_G^{1}\simeq B\underline{G(E)}$ corresponds to the Whittaker representation
$\mathrm{c}\text{-}\Ind_{U(E)}^{G(E)}\psi$.
\end{definition}

\begin{remark}[Relation to the literature]
This is the definition used by Fargues and Scholze in both the survey \cite[\S 6.2]{FSReview} and the main
preprint \cite[\S I.10]{FSGeometrization}.
\end{remark}

\subsection{Non-compactness and the meaning of \texorpdfstring{$F\ast W_{\psi}$}{F * Wpsi}}

The object $W_{\psi}$ is typically \emph{not} compact in $D_{\mathrm{lis}}(\Bun_G,\Lambda)$.
Nevertheless, the spectral action of $\Perf(\LocSys_{\checkG})$ on $D_{\mathrm{lis}}(\Bun_G,\Lambda)$ is
colimit-preserving, and one can define $F\ast W_{\psi}$ for perfect complexes $F$ with quasi-compact support by
writing $W_{\psi}$ as a filtered colimit of finite-type objects.

\begin{lemma}[Filtered colimit presentation of the Whittaker sheaf]\label{lem:Whittaker-colimit}
There exists a filtered system $\{W_{\psi,\alpha}\}_{\alpha}$ in $D_{\mathrm{lis}}(\Bun_G,\Lambda)$ such that:
\begin{enumerate}[label=(\alph*), leftmargin=2em]
  \item each $W_{\psi,\alpha}$ is supported on $\Bun_G^{1}$ and corresponds to a smooth representation of
  $G(E)$ of finite type;
  \item there is a canonical identification
  \[
    W_{\psi}\ \simeq\ \varinjlim_{\alpha} W_{\psi,\alpha}
  \]
  in $D_{\mathrm{lis}}(\Bun_G,\Lambda)$.
\end{enumerate}
\end{lemma}

\begin{proof}[Sketch of proof]
As a smooth $G(E)$-representation, $\mathrm{c}\text{-}\Ind_{U(E)}^{G(E)}\psi$ is the filtered union of its
$G(E)$-subrepresentations of finite type (for example, the subrepresentations generated by vectors fixed by
a chosen compact open subgroup).
Transporting this filtration through the equivalence
$D_{\mathrm{lis}}(\Bun_G^{1},\Lambda)\simeq D(\Rep^{\infty}_{\Lambda}(G(E)))$ and applying $(i_{1})_{!}$ gives the
required presentation.
This is exactly the device used in \cite[\S 6.2]{FSReview}.
\end{proof}

\begin{definition}[Acting on $W_{\psi}$]\label{def:act-on-Wpsi}
Let $F\in \Perf(\LocSys_{\checkG})$ be a perfect complex with quasi-compact support.
Define
\[
  F\ast W_{\psi}\ :=\ \varinjlim_{\alpha}\, \big(F\ast W_{\psi,\alpha}\big),
\]
where $F\ast(-)$ denotes the spectral action of $F$ on $D_{\mathrm{lis}}(\Bun_G,\Lambda)$
(Theorem~\ref{thm:spectral-action}).
This is well-defined because the spectral action preserves filtered colimits.
\end{definition}

\begin{remark}
One should think of $F\ast W_{\psi}$ as a \emph{non-abelian Fourier transform} of $F$ into the automorphic
category, with kernel $W_{\psi}$.
This is exactly the form of the functor that appears in \cite[Conjecture I.10.2]{FSGeometrization} and
\cite[Conjecture 6.2]{FSReview}.
\end{remark}

\subsection{The Whittaker-generated subcategory}

We now define the automorphic subcategory that is expected to match coherent sheaves on the spectral stack.

\begin{definition}[The Whittaker-generated subcategory]\label{def:omega-subcategory}
Let $D_{\mathrm{lis}}(\Bun_G,\Lambda)_{\omega}$ be the smallest full stable subcategory of
$D_{\mathrm{lis}}(\Bun_G,\Lambda)$ satisfying:
\begin{enumerate}[label=(\alph*), leftmargin=2em]
  \item it is closed under all (small) colimits;
  \item it contains the Whittaker sheaf $W_{\psi}$;
  \item it is stable under the spectral action of $\Perf(\LocSys_{\checkG})$
  (equivalently, if $A\in D_{\mathrm{lis}}(\Bun_G,\Lambda)_{\omega}$ and
  $F\in \Perf(\LocSys_{\checkG})$, then $F\ast A\in D_{\mathrm{lis}}(\Bun_G,\Lambda)_{\omega}$).
\end{enumerate}
\end{definition}

\begin{proposition}[Generation by spectral translates]\label{prop:omega-generated}
The subcategory $D_{\mathrm{lis}}(\Bun_G,\Lambda)_{\omega}$ is generated under colimits by the objects
$F\ast W_{\psi}$, where $F$ ranges over perfect complexes on $\LocSys_{\checkG}$ with quasi-compact support.
\end{proposition}

\begin{proof}[Sketch of proof]
By definition, $D_{\mathrm{lis}}(\Bun_G,\Lambda)_{\omega}$ is the closure of $W_{\psi}$ under colimits and under
the spectral action.  Any object obtained from $W_{\psi}$ by iterating colimits and the action is a colimit of
objects of the form $F\ast W_{\psi}$ for some $F$; conversely, all such objects lie in
$D_{\mathrm{lis}}(\Bun_G,\Lambda)_{\omega}$.
\end{proof}

\begin{remark}[Dependence on the Whittaker datum]
If $(B,\psi)$ and $(B',\psi')$ are two Whittaker data, they are conjugate under $G(E)$ when $G$ is quasi-split.
Conjugation induces an autoequivalence of $D_{\mathrm{lis}}(\Bun_G,\Lambda)$ that identifies the corresponding
subcategories $D_{\mathrm{lis}}(\Bun_G,\Lambda)_{\omega}$.
For the rest of the paper we fix one Whittaker datum once and for all.
\end{remark}

\subsection{The categorical geometrization conjecture in Whittaker form}

The point of introducing $W_{\psi}$ and $D_{\mathrm{lis}}(\Bun_G,\Lambda)_{\omega}$ is that they allow one to
formulate the expected equivalence with the spectral category.

\begin{conjecture}[Categorical geometrization conjecture, Whittaker form]\label{conj:categorical-geom-whittaker}
Assume $G$ is quasi-split and fix Whittaker data $(B,\psi)$.
\begin{enumerate}[label=(\alph*), leftmargin=2em]
  \item For every perfect complex $F\in \Perf(\LocSys_{\checkG})$ with quasi-compact support, the object
  $F\ast W_{\psi}$ is compact in $D_{\mathrm{lis}}(\Bun_G,\Lambda)$.
  \item The functor
  \[
    \Perf(\LocSys_{\checkG})\ \longrightarrow\ D_{\mathrm{lis}}(\Bun_G,\Lambda),\qquad
    F\longmapsto F\ast W_{\psi}
  \]
  extends (uniquely) to an equivalence of small stable $\infty$-categories
  \[
    \Coh_{\mathrm{Nilp}}\big(\LocSys_{\checkG}\big)\ \xrightarrow{\ \sim\ }\ D_{\mathrm{lis}}(\Bun_G,\Lambda)_{\omega},
  \]
  compatible with the spectral action of $\Perf(\LocSys_{\checkG})$.
\end{enumerate}
\end{conjecture}

\begin{remark}
Conjecture~\ref{conj:categorical-geom-whittaker} is a slightly repackaged form of
\cite[Conjecture I.10.2]{FSGeometrization} and \cite[Conjecture 6.2]{FSReview} (with our notation
$\LocSys_{\checkG}$ for the stack of parameters).
In characteristic zero coefficients (for example over $\mathbb{Q}_{\ell}$ at banal primes), the nilpotent
singular support condition is expected to be automatic, and the conjecture should simplify by replacing
$\Coh_{\mathrm{Nilp}}(\LocSys_{\checkG})$ with $\Coh(\LocSys_{\checkG})$.
\end{remark}

\begin{remark}[A first consequence: the stable Bernstein center]
If Conjecture~\ref{conj:categorical-geom-whittaker} holds, then applying it to the unit object
$\OO_{\LocSys_{\checkG}}\in \Coh_{\mathrm{Nilp}}(\LocSys_{\checkG})$ gives
\[
  \End_{D_{\mathrm{lis}}(\Bun_G,\Lambda)}(W_{\psi})
  \ \simeq\
  \End_{\Coh_{\mathrm{Nilp}}(\LocSys_{\checkG})}(\OO_{\LocSys_{\checkG}})
  \ \simeq\
  \Gamma(\LocSys_{\checkG},\OO),
\]
which is the spectral (stable) Bernstein center acting on Whittaker models.
This is the categorical form of the identification of centers explained in \cite[\S 6.3.1]{FSReview}.
\end{remark}

\section{Parabolic functors and gluing from Levi subgroups}\label{sec:parabolic}

This section records the parabolic functoriality on $\Bun_G$ that will be used to implement the
``gluing from Levi subgroups'' step in the proof strategy of
Section~\ref{sec:proof-strategy}.
Concretely, for each parabolic subgroup $P\subset G$ with Levi quotient $M$, Hamann--Hansen--Scholze construct
geometric Eisenstein series and constant term functors
\[
  \mathrm{Eis}_P:\ D_{\mathrm{lis}}(\Bun_M,\Lambda)\to D_{\mathrm{lis}}(\Bun_G,\Lambda),
  \qquad
  \mathrm{CT}_P:\ D_{\mathrm{lis}}(\Bun_G,\Lambda)\to D_{\mathrm{lis}}(\Bun_M,\Lambda),
\]
prove strong finiteness properties, and establish a geometric analogue of Bernstein's second adjointness
\cite{HHSGeometricEis}.
These functors are the local geometric counterparts of parabolic induction and Jacquet modules.

We divide the discussion into five parts:
\begin{enumerate}[label=(\roman*), leftmargin=2em]
  \item the geometry of $\Bun_P$ and the basic correspondence $\Bun_M \leftarrow \Bun_P \to \Bun_G$;
  \item definition of $\mathrm{Eis}_P$ and $\mathrm{CT}_P$ (including normalization conventions);
  \item finiteness and boundedness properties needed for compactness arguments;
  \item adjunction and second adjointness;
  \item the resulting ``Eisenstein plus cuspidal'' generation statements.
\end{enumerate}

\subsection{Parabolic substacks and the correspondence $\Bun_M \leftarrow \Bun_P \to \Bun_G$}\label{subsec:BunP}

Fix a parabolic subgroup $P\subset G$ defined over $E$, and let $M$ be a Levi quotient of $P$.
Let $N\subset P$ denote the unipotent radical, so that $P=MN$.

\begin{definition}[Stacks of $P$- and $M$-bundles]\label{def:BunP-BunM}
Let $\Bun_P$ (respectively $\Bun_M$) denote the v-stack of $P$-bundles (respectively $M$-bundles) on the
Fargues--Fontaine curve.
There are natural morphisms of v-stacks
\[
  \Bun_M \xleftarrow{\,q\,} \Bun_P \xrightarrow{\,p\,} \Bun_G,
\]
where:
\begin{itemize}[leftmargin=2em]
  \item $p$ is extension of structure group along $P\hookrightarrow G$ (forgetting the reduction to $P$);
  \item $q$ is extension of structure group along $P\twoheadrightarrow M$ (quotienting by the unipotent radical).
\end{itemize}
\end{definition}

\begin{remark}
In analogy with the classical case, the stack $\Bun_P$ may be viewed as parameterizing pairs
$(\mathcal{E},\mathcal{E}_P)$ consisting of a $G$-bundle $\mathcal{E}$ together with a reduction
$\mathcal{E}_P$ to $P$; then $p(\mathcal{E},\mathcal{E}_P)=\mathcal{E}$ and $q(\mathcal{E},\mathcal{E}_P)$ is the
induced $M$-bundle $\mathcal{E}_P/N$.
\end{remark}

\subsection{Definition of geometric Eisenstein series and constant term}\label{subsec:EisCT-def}

We next recall the definitions of the parabolic functors on derived categories of lisse $\ell$-adic sheaves.
The precise definition requires choosing which of the four operations is appropriate in the correspondence
\[
  \Bun_M \xleftarrow{q} \Bun_P \xrightarrow{p} \Bun_G.
\]
For $\ell\neq p$, the relevant functors exist in the range needed here and satisfy the usual base change and
projection formula statements.

\begin{definition}[Unnormalized parabolic functors]\label{def:unnormalized-parabolic}
Define the \emph{unnormalized} Eisenstein and constant term functors by
\[
  \mathrm{Eis}_P^{\mathrm{un}} \ :=\ p_\ast\, q^!:\ D_{\mathrm{lis}}(\Bun_M,\Lambda)\longrightarrow
  D_{\mathrm{lis}}(\Bun_G,\Lambda),
\]
\[
  \mathrm{CT}_P^{\mathrm{un}} \ :=\ q_\ast\, p^!:\ D_{\mathrm{lis}}(\Bun_G,\Lambda)\longrightarrow
  D_{\mathrm{lis}}(\Bun_M,\Lambda).
\]
\end{definition}

\begin{remark}[Normalization conventions]
In representation theory, one often normalizes parabolic induction and Jacquet functors by a power of the modulus
character $\delta_P^{1/2}$ so that adjunctions take a symmetric form and duality behaves well.
In the geometric setting, the modulus character is replaced by a combination of Tate twists and cohomological
shifts depending on relative dimensions of stacks and on the half-sum of roots of $N$.
Hamann--Hansen--Scholze implement this by defining normalized variants
$\mathrm{Eis}_P$ and $\mathrm{CT}_P$ that differ from the unnormalized ones by explicit twists and shifts and
that satisfy the cleanest adjunction statements \cite[\S 1.1 and \S 6]{HHSGeometricEis}.
In what follows, we work with their normalized functors and suppress the explicit normalizing twists.
\end{remark}

\begin{definition}[Normalized parabolic functors]\label{def:normalized-parabolic}
Let $\mathrm{Eis}_P$ and $\mathrm{CT}_P$ denote the normalized geometric Eisenstein and constant term functors
constructed in \cite{HHSGeometricEis}.
They are obtained from $\mathrm{Eis}_P^{\mathrm{un}}$ and $\mathrm{CT}_P^{\mathrm{un}}$ by tensoring with a fixed
Tate twist and shifting by a fixed cohomological degree that depends only on $P$.
\end{definition}

\subsection{Finiteness and boundedness properties}\label{subsec:EisCT-finiteness}

The key technical input for later sections is that $\mathrm{Eis}_P$ and $\mathrm{CT}_P$ behave like parabolic
induction and Jacquet modules: they preserve boundedness with respect to Harder--Narasimhan strata and satisfy
finiteness properties strong enough to control compactness.

\begin{theorem}[Finiteness theorems for parabolic functors]\label{thm:HHS-finiteness}
Let $\Lambda$ be a torsion ring of characteristic $\ell\neq p$, or let $\Lambda=\Z_\ell$.
Then the normalized parabolic functors of Definition~\ref{def:normalized-parabolic} satisfy:
\begin{enumerate}[label=(\alph*), leftmargin=2em]
  \item \textbf{Boundedness on bounded opens:}
  for every bound $\mu$ on $\Bun_M$ there exists a bound $\mu'$ on $\Bun_G$ such that
  \[
    \mathrm{Eis}_P\big(D_{\mathrm{lis}}(\Bun_M,\Lambda)_{\leq \mu}\big)\ \subset\
    D_{\mathrm{lis}}(\Bun_G,\Lambda)_{\leq \mu'}.
  \]
  Similarly, for every bound $\nu$ on $\Bun_G$ there exists a bound $\nu'$ on $\Bun_M$ such that
  \[
    \mathrm{CT}_P\big(D_{\mathrm{lis}}(\Bun_G,\Lambda)_{\leq \nu}\big)\ \subset\
    D_{\mathrm{lis}}(\Bun_M,\Lambda)_{\leq \nu'}.
  \]
  \item \textbf{Cohomological amplitude:}
  on each bounded open, $\mathrm{Eis}_P$ and $\mathrm{CT}_P$ have finite cohomological amplitude.
  \item \textbf{Compactness preservation:}
  on bounded opens, $\mathrm{Eis}_P$ and $\mathrm{CT}_P$ take compact objects to compact objects.
\end{enumerate}
\end{theorem}

\begin{proof}[Proof sketch]
These statements are the core finiteness results of \cite{HHSGeometricEis}.
The boundedness assertions ultimately come from:
\begin{itemize}[leftmargin=2em]
  \item the Harder--Narasimhan classification of bundles by $B(G)$ and the finiteness of bounded subsets;
  \item a geometric analysis of extensions of an $M$-bundle by $N$-torsors, which controls how instability can grow
  under extension of structure group $P\hookrightarrow G$ or under taking Levi quotients $P\twoheadrightarrow M$.
\end{itemize}
Finite cohomological amplitude and compactness preservation are proved by reducing to bounded opens (finite unions
of strata), and then proving that the relevant correspondences are ``cohomologically proper'' in a sense
appropriate for lisse $\ell$-adic sheaves on v-stacks.
\end{proof}

\begin{remark}[Why these finiteness statements matter later]
The compactness statement in Conjecture~\ref{conj:categorical-geom-whittaker}(a) will be reduced to compactness
calculations after applying constant term functors.  Theorem~\ref{thm:HHS-finiteness} is exactly what makes such
a reduction possible.
\end{remark}

\subsection{Adjunction and Bernstein's second adjointness}\label{subsec:second-adjointness}

In representation theory, parabolic induction and Jacquet modules satisfy:
\begin{itemize}[leftmargin=2em]
  \item Frobenius adjunction: parabolic induction is left adjoint to Jacquet;
  \item Bernstein's second adjointness: parabolic induction is also right adjoint to Jacquet for the opposite
  parabolic (after normalization).
\end{itemize}
Hamann--Hansen--Scholze prove geometric analogues of these statements.

Let $P^{-}\subset G$ denote a parabolic subgroup opposite to $P$, with the same Levi quotient $M$.

\begin{theorem}[Adjunction and second adjointness]\label{thm:HHS-adjointness}
With normalized conventions, the parabolic functors satisfy:
\begin{enumerate}[label=(\alph*), leftmargin=2em]
  \item \textbf{First adjointness:} $\mathrm{Eis}_P$ is left adjoint to $\mathrm{CT}_P$.
  \item \textbf{Second adjointness:} $\mathrm{Eis}_P$ is also right adjoint to $\mathrm{CT}_{P^{-}}$.
\end{enumerate}
Both adjunctions hold on the bounded-support subcategories, and hence extend to the filtered colimit
$D_{\mathrm{lis}}(\Bun_G,\Lambda)$.
\end{theorem}

\begin{proof}[Proof sketch]
This is proved in \cite{HHSGeometricEis}.
The first adjunction is a formal consequence of the correspondence definition of $\mathrm{Eis}_P$ and
$\mathrm{CT}_P$ together with Verdier duality.
The second adjunction is substantially deeper: it relies on a geometric analysis of the correspondence
$\Bun_P\times_{\Bun_G}\Bun_{P^{-}}$ and a vanishing statement that is the geometric analogue of the
classical second adjointness theorem.
\end{proof}

\begin{remark}
Theorem~\ref{thm:HHS-adjointness} is the local geometric input that plays the role of ``ambidexterity'' in the
global proof of geometric Langlands.  It is one of the main reasons the $\ell\neq p$ case on $\Bun_G$ is a
promising environment for a Gaitsgory-style proof.
\end{remark}

\subsection{Compatibility with Hecke operators and the spectral action}\label{subsec:parabolic-hecke}

The parabolic functors are expected to be compatible with Hecke operators and, more strongly, to be linear with
respect to the spectral action.  This is the categorical form of the statement that local Langlands parameters
are functorial for Levi inclusions.

Let ${}^LM\hookrightarrow {}^LG$ denote the natural morphism of $L$-groups induced by $M\hookrightarrow G$.
It induces a morphism of parameter stacks
\[
  \mathrm{res}_{M}^{G}:\ \LocSys_{\checkG}\longrightarrow \LocSys_{\checkM}.
\]

\begin{proposition}[Hecke compatibility]\label{prop:parabolic-hecke}
For $V\in \Rep(\checkG)$ and its restriction $V|_{\checkM}\in \Rep(\checkM)$, there are canonical equivalences
of functors:
\[
  \mathrm{CT}_P\circ T_V\ \simeq\ T_{V|_{\checkM}}\circ \mathrm{CT}_P,
  \qquad
  T_V\circ \mathrm{Eis}_P\ \simeq\ \mathrm{Eis}_P\circ T_{V|_{\checkM}}.
\]
\end{proposition}

\begin{proof}[Proof sketch]
This is a geometric form of the compatibility of parabolic induction and Jacquet modules with spherical Hecke
operators.  It is proved by comparing the corresponding correspondences on $\Bun_G$ and $\Bun_M$ and using
base change.
In the v-stack setting, the needed finiteness and base change statements are part of the main results of
\cite{HHSGeometricEis}.
\end{proof}

\begin{proposition}[Spectral linearity]\label{prop:parabolic-spectral-linearity}
The functors $\mathrm{Eis}_P$ and $\mathrm{CT}_P$ are linear with respect to the spectral action in the sense that
there are canonical equivalences
\[
  \mathrm{CT}_P\big(F\ast \mathcal{A}\big)\ \simeq\
  \big(\mathrm{res}_{M}^{G}\,F\big)\ast \mathrm{CT}_P(\mathcal{A}),
\]
\[
  \mathrm{Eis}_P\big(F'\ast \mathcal{B}\big)\ \simeq\
  \big((\mathrm{res}_{M}^{G})^{\ast}F'\big)\ast \mathrm{Eis}_P(\mathcal{B}),
\]
for $F\in \Perf(\LocSys_{\checkG})$, $F'\in \Perf(\LocSys_{\checkM})$ and objects
$\mathcal{A}\in D_{\mathrm{lis}}(\Bun_G,\Lambda)$, $\mathcal{B}\in D_{\mathrm{lis}}(\Bun_M,\Lambda)$.
\end{proposition}

\begin{proof}[Proof sketch]
By Theorem~\ref{thm:spectral-action}, the spectral action is generated (as a monoidal action) by the Hecke
operators attached to representations of $\checkG$.
Thus Proposition~\ref{prop:parabolic-spectral-linearity} is a formal consequence of
Proposition~\ref{prop:parabolic-hecke} and monoidality of the spectral action.
\end{proof}

\subsection{Cuspidal objects and Eisenstein generation}\label{subsec:cuspidal}

We now formulate the ``gluing from Levi subgroups'' principle on the automorphic side.

\begin{definition}[Cuspidal subcategory]\label{def:cuspidal}
Define the cuspidal subcategory $D_{\mathrm{cusp}}(\Bun_G,\Lambda)\subset D_{\mathrm{lis}}(\Bun_G,\Lambda)$ by
\[
  D_{\mathrm{cusp}}(\Bun_G,\Lambda)\ :=\ \bigcap_{P\subsetneq G}\ker(\mathrm{CT}_P),
\]
where $P$ runs over proper parabolic subgroups of $G$ (up to conjugacy).
\end{definition}

\begin{definition}[Eisenstein-generated subcategory]\label{def:eisenstein-generated}
Let $D_{\mathrm{Eis}}(\Bun_G,\Lambda)\subset D_{\mathrm{lis}}(\Bun_G,\Lambda)$ be the smallest full stable
subcategory closed under colimits that contains the essential images of $\mathrm{Eis}_P$ for all proper
parabolics $P\subsetneq G$ (with varying Levi quotients).
\end{definition}

\begin{theorem}[Eisenstein plus cuspidal generation]\label{thm:HHS-generation}
The category $D_{\mathrm{lis}}(\Bun_G,\Lambda)$ is generated under colimits by
$D_{\mathrm{cusp}}(\Bun_G,\Lambda)$ and $D_{\mathrm{Eis}}(\Bun_G,\Lambda)$.
Equivalently, the smallest colimit-closed stable subcategory containing the cuspidal objects and all Eisenstein
series objects is the whole category.
\end{theorem}

\begin{proof}[Proof sketch]
This is proved in \cite{HHSGeometricEis}.
The strategy is an induction on semisimple rank that uses:
\begin{itemize}[leftmargin=2em]
  \item the finiteness properties of Theorem~\ref{thm:HHS-finiteness} to ensure that constant term functors detect
  noncuspidal objects without losing control of cohomological degrees;
  \item adjunction and second adjointness (Theorem~\ref{thm:HHS-adjointness}) to construct, from an object with
  nonzero constant term, a map from an Eisenstein series object detecting it;
  \item gluing along Harder--Narasimhan strata to reduce global statements on $\Bun_G$ to finite unions of strata.
\end{itemize}
\end{proof}

\begin{remark}[Analogy with Bernstein decomposition]
In the representation theory of $p$-adic groups, one decomposes the category of smooth representations into
blocks generated from supercuspidal data by parabolic induction.
Theorem~\ref{thm:HHS-generation} is the geometric local analogue: it gives a structural decomposition of
$D_{\mathrm{lis}}(\Bun_G,\Lambda)$ into a cuspidal part and a part generated from Levi subgroups by Eisenstein
series.
\end{remark}

\subsection{Restriction to strata and comparison with parabolic induction}\label{subsec:parabolic-on-strata}

Finally, we record the basic compatibility with the representation-theoretic meaning of strata.  Fix $b\in B(G)$
and identify $\Bun_G^b\simeq B\underline{J_b(E)}$.  Let $b_M\in B(M)$ be the corresponding element induced by the
Levi quotient map (this is the element controlling the induced $M$-bundle on the stratum of $\Bun_M$).
Then one expects $J_{b_M}(E)$ to be a Levi subgroup of $J_b(E)$ (in a suitable inner form sense), and the
restriction of $\mathrm{Eis}_P$ and $\mathrm{CT}_P$ to strata recovers parabolic induction and Jacquet functors
for the groups $J_b(E)$.

\begin{proposition}[Parabolic functors on basic strata]\label{prop:parabolic-on-basic-strata}
Assume $b$ is basic.  Under the identifications
\[
  D_{\mathrm{lis}}(\Bun_G^b,\Lambda)\simeq D\big(\Rep^\infty_\Lambda(J_b(E))\big),
  \qquad
  D_{\mathrm{lis}}(\Bun_M^{b_M},\Lambda)\simeq D\big(\Rep^\infty_\Lambda(J_{b_M}(E))\big),
\]
the restrictions of $\mathrm{Eis}_P$ and $\mathrm{CT}_P$ to the basic strata coincide with the derived
normalized parabolic induction and normalized Jacquet functors between smooth representation categories of
$J_{b_M}(E)$ and $J_b(E)$.
\end{proposition}

\begin{proof}[Proof sketch]
This is the local meaning of the geometric definitions: on a basic stratum the moduli problem reduces to a
classifying stack, and the correspondence $\Bun_M\leftarrow \Bun_P\to \Bun_G$ restricts to the usual
correspondence defining parabolic induction at the level of groupoids.
The identification is discussed in \cite[\S III and \S V]{FSGeometrization} and is used implicitly in the
representation-theoretic interpretations of \cite{HHSGeometricEis}.
\end{proof}

\begin{remark}[How this will be used later]
Proposition~\ref{prop:parabolic-on-basic-strata} allows one to reduce certain statements about parabolic functors
on $\Bun_G$ to statements about classical parabolic induction and Jacquet modules on smooth representations.
In particular, it is the basic input for cuspidal block arguments in
Section~\ref{sec:proof-strategy}.
\end{remark}

\section{Proof strategy for the categorical geometrization conjecture}\label{sec:proof-strategy}

In this section we give a detailed blueprint for a proof of
Conjecture~\ref{conj:categorical-geom-whittaker}.
The argument is designed to parallel the structure of the proof of the geometric Langlands conjecture
in the work of Arinkin--Gaitsgory and Gaitsgory--Raskin: singular support on the spectral side,
a Whittaker generator on the automorphic side, monadicity and endomorphism calculations for full faithfulness,
and parabolic gluing (with second adjointness) for essential surjectivity
\cite{AGSingSupp,GRProofI,GRProofIV,GRProofV}.

Throughout we work in the case $\ell\neq p$ and keep the coefficient ring $\Lambda$ from
Section~\ref{sec:notation}.  We assume $G$ is quasi-split and Whittaker data $(B,\psi)$ are fixed, so that
$W_{\psi}\in D_{\mathrm{lis}}(\Bun_G,\Lambda)$ and
$D_{\mathrm{lis}}(\Bun_G,\Lambda)_{\omega}$ are defined as in Section~\ref{sec:whittaker}.

\subsection{Two equivalent formulations: compact objects and ind-completions}\label{subsec:two-levels}

A persistent technical point is that the spectral category in the conjecture is a \emph{small} category
$\Coh_{\mathrm{Nilp}}(\LocSys_{\checkG})$, whereas the Whittaker-generated automorphic category
$D_{\mathrm{lis}}(\Bun_G,\Lambda)_{\omega}$ is defined as a \emph{colimit-closed} subcategory and is therefore
presentable.  To keep the comparison honest, we separate the ``compact'' and the ``presentable'' formulations.

\begin{definition}[Compact objects in the Whittaker-generated category]\label{def:omega-compacts}
Let
\[
  D_{\mathrm{lis}}(\Bun_G,\Lambda)_{\omega}^{\mathrm{c}}
  \ :=\
  \Big(D_{\mathrm{lis}}(\Bun_G,\Lambda)_{\omega}\Big)^{\mathrm{c}}
\]
denote the full subcategory of compact objects in the presentable category
$D_{\mathrm{lis}}(\Bun_G,\Lambda)_{\omega}$.
\end{definition}

\begin{definition}[Ind-completion on the spectral side]\label{def:IndCohNilp}
Let $\IndCoh_{\mathrm{Nilp}}(\LocSys_{\checkG})$ be as in Definition~\ref{def:IndCohNilp}, and note that
\[
  \IndCoh_{\mathrm{Nilp}}(\LocSys_{\checkG})
  \ \simeq\
  \Ind\big(\Coh_{\mathrm{Nilp}}(\LocSys_{\checkG})\big)
\]
because $\Coh_{\mathrm{Nilp}}$ is, by definition, the full subcategory of compact objects.
\end{definition}

\begin{remark}
In later stages of the project it will be useful to prove the equivalence at the presentable level
\[
  \IndCoh_{\mathrm{Nilp}}(\LocSys_{\checkG})\ \simeq\ D_{\mathrm{lis}}(\Bun_G,\Lambda)_{\omega},
\]
and then recover the equivalence of compact objects by taking compacts.  Conversely, it is often easier
to prove the equivalence first on compact objects and then ind-complete.  We will allow ourselves to move
freely between these two formulations.
\end{remark}

\subsection{The comparison functor and the first main theorem}\label{subsec:comparison-functor}

We define the comparison functor on perfect complexes by the spectral action on the Whittaker sheaf.

\begin{definition}[The basic comparison functor on perfect complexes]\label{def:Phi-basic}
Let $\Phi_{\Perf}$ be the functor
\[
  \Phi_{\Perf}:\ \Perf(\LocSys_{\checkG})\ \longrightarrow\ D_{\mathrm{lis}}(\Bun_G,\Lambda)_{\omega},
  \qquad
  F\ \longmapsto\ F\ast W_{\psi},
\]
where the action on $W_{\psi}$ is defined as in Definition~\ref{def:act-on-Wpsi} when $F$ has quasi-compact
support.
\end{definition}

The first substantive input needed for the conjecture is the compactness of these translates.

\begin{theorem}[Compactness of Whittaker translates]\label{thm:compactness}
Let $F\in \Perf(\LocSys_{\checkG})$ have quasi-compact support.
Then $\Phi_{\Perf}(F)=F\ast W_{\psi}$ is compact in $D_{\mathrm{lis}}(\Bun_G,\Lambda)_{\omega}$.
Equivalently,
\[
  \Phi_{\Perf}:\ \Perf(\LocSys_{\checkG})_{\mathrm{q.c.}}\ \longrightarrow\ D_{\mathrm{lis}}(\Bun_G,\Lambda)_{\omega}^{\mathrm{c}}
\]
lands in the compact objects.
\end{theorem}

\begin{remark}
Theorem~\ref{thm:compactness} is exactly part (a) of Conjecture~\ref{conj:categorical-geom-whittaker}.
Once it is established, the equivalence of categories becomes a statement about compactly generated
module categories and can be attacked by monadicity and gluing.
\end{remark}

\subsection{Compactness: reduction to torsion coefficients and bounded opens}\label{subsec:compactness-reductions}

We record a structured reduction of Theorem~\ref{thm:compactness} to finiteness results already available in the
literature (notably \cite{HHSGeometricEis} and \cite{MilesGluingHN}).

\begin{lemma}[Reduction to torsion coefficients]\label{lem:compactness-torsion}
Assume $\Lambda=\Z_\ell$.
To prove Theorem~\ref{thm:compactness} for $\Z_\ell$-coefficients, it suffices to prove the torsion analogue:
for each $n\ge 1$, and each $F_n\in \Perf(\LocSys_{\checkG}\times_{\Spec(\Z_\ell)}\Spec(\Z/\ell^n\Z))$ of
quasi-compact support, the object $F_n\ast W_{\psi,n}$ is compact in
$D_{\mathrm{lis}}(\Bun_G,\Z/\ell^n\Z)_{\omega}$, where $W_{\psi,n}$ is the mod $\ell^n$ reduction of $W_{\psi}$.
\end{lemma}

\begin{proof}[Proof sketch]
The category $D_{\mathrm{lis}}(\Bun_G,\Z_\ell)$ is an $\ell$-adic limit of the torsion categories, and compactness
in the $\ell$-adic completion can be checked after reduction modulo $\ell^n$ provided one keeps track of derived
$\ell$-adic completeness.  This is the standard passage from torsion to $\Z_\ell$ in the formalism of
\cite[\S I.2]{FSGeometrization} and is used systematically in \cite{HHSGeometricEis}.
\end{proof}

\begin{lemma}[Reduction to bounded opens]\label{lem:compactness-bounded}
Assume $\Lambda$ is torsion of characteristic $\ell\neq p$.
To prove compactness of $F\ast W_{\psi}$, it suffices to show:
\begin{enumerate}[label=(\alph*), leftmargin=2em]
  \item there exists a bound $\mu$ such that $F\ast W_{\psi}$ is supported on $\Bun_G^{\le \mu}$, and
  \item viewed as an object of $D_{\mathrm{lis}}(\Bun_G^{\le \mu},\Lambda)$, it is compact.
\end{enumerate}
\end{lemma}

\begin{proof}[Proof sketch]
This is a formal consequence of the exhaustion
$D_{\mathrm{lis}}(\Bun_G,\Lambda)=\varinjlim_{\mu}D_{\mathrm{lis}}(\Bun_G,\Lambda)_{\le \mu}$
(Proposition~\ref{prop:exhaustion}) and the fact that compactness is local on a filtered union of open
substacks.  The main point is that if an object is supported on a quasi-compact open, compactness can be checked
in the quasi-compact subcategory.
\end{proof}

\begin{lemma}[Boundedness of Whittaker translates]\label{lem:boundedness-of-translates}
Let $F\in \Perf(\LocSys_{\checkG})$ have quasi-compact support.
Then $F\ast W_{\psi}$ is supported on a bounded open $\Bun_G^{\le \mu}$ for some $\mu$ depending on $F$.
\end{lemma}

\begin{proof}[Proof sketch]
By definition of the spectral action, the action of $F$ is built from finitely many Hecke operators
$T_V$ with $V\in\Rep(\checkG)$, together with finite colimits and shifts (because $F$ is perfect and has
quasi-compact support).
By Proposition~\ref{prop:Hecke-boundedness}, each $T_V$ sends bounded-support objects to bounded-support objects.
Since $W_{\psi}$ is supported on the neutral stratum $\Bun_G^{1}$, repeated application of finitely many
Hecke operators yields bounded support.
\end{proof}

\begin{remark}[Where Miles and Hamann--Hansen--Scholze enter]
After Lemmas~\ref{lem:compactness-torsion}--\ref{lem:boundedness-of-translates}, the remaining content of
Theorem~\ref{thm:compactness} is: compactness of certain explicit objects in
$D_{\mathrm{lis}}(\Bun_G^{\le\mu},\Lambda)$ for a bounded open $\Bun_G^{\le\mu}$.
On bounded opens, one can glue along Harder--Narasimhan strata \cite{MilesGluingHN}, and one has strong finiteness
properties for Hecke and parabolic correspondences \cite{HHSGeometricEis}.  These are the finiteness inputs that
the proof template will use repeatedly.
\end{remark}

\subsection{Extension from perfect complexes to coherent sheaves}\label{subsec:extend-to-coh}

Assuming Theorem~\ref{thm:compactness}, we explain how to extend $\Phi_{\Perf}$ to a functor on coherent objects
with nilpotent singular support.

\begin{proposition}[Spectral devissage on $\Coh_{\mathrm{Nilp}}$]\label{prop:spectral-devissage}
The category $\Coh_{\mathrm{Nilp}}(\LocSys_{\checkG})$ is the idempotent-complete stable subcategory generated by
perfect complexes with quasi-compact support under finite colimits and retracts.
\end{proposition}

\begin{remark}
Proposition~\ref{prop:spectral-devissage} is a purely spectral statement and should follow from the general
formalism of ind-coherent sheaves with singular support on quasi-smooth stacks \cite{AGSingSupp}.
In later versions of the paper we will extract an explicit reference or include a proof.
For the blueprint, we treat it as a standard devissage input.
\end{remark}

\begin{proposition}[Extension of $\Phi_{\Perf}$ to $\Coh_{\mathrm{Nilp}}$]\label{prop:extend-Phi}
Assume Theorem~\ref{thm:compactness} and Proposition~\ref{prop:spectral-devissage}.
Then there exists a unique exact functor
\[
  \Phi^{\mathrm{c}}:\ \Coh_{\mathrm{Nilp}}(\LocSys_{\checkG})\ \longrightarrow\ D_{\mathrm{lis}}(\Bun_G,\Lambda)_{\omega}^{\mathrm{c}}
\]
whose restriction to $\Perf(\LocSys_{\checkG})_{\mathrm{q.c.}}$ is $F\mapsto F\ast W_{\psi}$.
Moreover, $\Phi^{\mathrm{c}}$ admits an ind-extension
\[
  \widetilde{\Phi}:\ \IndCoh_{\mathrm{Nilp}}(\LocSys_{\checkG})\ \longrightarrow\ D_{\mathrm{lis}}(\Bun_G,\Lambda)_{\omega}
\]
which is colimit-preserving and whose restriction to compact objects is $\Phi^{\mathrm{c}}$.
\end{proposition}

\begin{proof}[Proof sketch]
By Theorem~\ref{thm:compactness}, the functor $\Phi_{\Perf}$ lands in compact objects, hence extends uniquely
to the idempotent-complete thick subcategory generated by $\Perf(\LocSys_{\checkG})_{\mathrm{q.c.}}$.
By Proposition~\ref{prop:spectral-devissage} this thick subcategory is $\Coh_{\mathrm{Nilp}}$.
Ind-completing yields a colimit-preserving functor on $\IndCoh_{\mathrm{Nilp}}$.
\end{proof}

\subsection{Compatibility with parabolic functors: the gluing interface}\label{subsec:Phi-parabolic}

To follow the geometric Langlands proof architecture, one needs compatibility of the comparison functor with
Eisenstein series and constant term functors on both sides.

\paragraph{Automorphic parabolic functors.}
On the automorphic side, the functors $\mathrm{Eis}_P$ and $\mathrm{CT}_P$ are constructed in
Section~\ref{sec:parabolic} and satisfy finiteness and adjunction properties by
Theorems~\ref{thm:HHS-finiteness} and \ref{thm:HHS-adjointness}.

\paragraph{Spectral parabolic functors.}
On the spectral side, the inclusion ${}^LM\hookrightarrow {}^LG$ suggests that parabolic induction should be
realized by the correspondence of parameter stacks associated with the dual parabolic $\checkP\subset \checkG$:
\[
  \LocSys_{\checkM}\ \xleftarrow{\,q_{\mathrm{spec}}\,}\ \LocSys_{\checkP}\ \xrightarrow{\,p_{\mathrm{spec}}\,}\ \LocSys_{\checkG}.
\]
We therefore define:

\begin{definition}[Spectral Eisenstein and constant term functors]\label{def:spectral-parabolic}
Assume the derived stacks $\LocSys_{\checkG}$, $\LocSys_{\checkM}$, and $\LocSys_{\checkP}$ are quasi-smooth, and the
correspondence maps are such that the relevant ind-coherent functors are defined.
Define
\[
  \mathrm{Eis}_P^{\mathrm{spec}} \ :=\ (p_{\mathrm{spec}})_{\ast}\,(q_{\mathrm{spec}})^{!}:\ 
  \IndCoh_{\mathrm{Nilp}}(\LocSys_{\checkM})\to \IndCoh_{\mathrm{Nilp}}(\LocSys_{\checkG}),
\]
\[
  \mathrm{CT}_P^{\mathrm{spec}} \ :=\ (q_{\mathrm{spec}})_{\ast}\,(p_{\mathrm{spec}})^{!}:\ 
  \IndCoh_{\mathrm{Nilp}}(\LocSys_{\checkG})\to \IndCoh_{\mathrm{Nilp}}(\LocSys_{\checkM}),
\]
with the same normalizing twists and shifts as on the automorphic side (suppressed from the notation).
\end{definition}

\begin{remark}
Definition~\ref{def:spectral-parabolic} is the direct local analogue of the spectral Eisenstein and constant term
functors in geometric Langlands.
The nilpotent singular support condition is expected to be exactly what ensures these functors preserve the
nilpotent subcategories and satisfy adjunction properties parallel to the automorphic ones
\cite{AGSingSupp,GRProofI}.
\end{remark}

\begin{conjecture}[Parabolic compatibility of $\widetilde{\Phi}$]\label{conj:Phi-parabolic}
Assume the spectral parabolic functors of Definition~\ref{def:spectral-parabolic} exist.
Then there are canonical equivalences of functors
\[
  \widetilde{\Phi}_{G}\circ \mathrm{Eis}_P^{\mathrm{spec}}\ \simeq\ \mathrm{Eis}_P\circ \widetilde{\Phi}_{M},
  \qquad
  \widetilde{\Phi}_{M}\circ \mathrm{CT}_P^{\mathrm{spec}}\ \simeq\ \mathrm{CT}_P\circ \widetilde{\Phi}_{G}.
\]
\end{conjecture}

\begin{remark}[Role in the proof]
Conjecture~\ref{conj:Phi-parabolic} is the mechanism by which one reduces the equivalence for $G$ to the
equivalence for Levi subgroups.  It is the local avatar of the compatibility between geometric Langlands
functors and Eisenstein series.
In this blueprint, we treat it as a required input to be proved by analyzing the universal Hecke eigensheaf
construction of the spectral action and the geometric definitions of Eisenstein series on $\Bun_G$.
\end{remark}

\subsection{Full faithfulness via monadicity and endomorphisms}\label{subsec:monadicity}

Assume we have constructed $\Phi^{\mathrm{c}}$ and $\widetilde{\Phi}$ as in Proposition~\ref{prop:extend-Phi}.
To prove that $\Phi^{\mathrm{c}}$ is fully faithful we aim to apply a Barr--Beck type theorem.
For this, we need:
\begin{enumerate}[label=(\roman*), leftmargin=2em]
  \item existence and good behavior of the right adjoint of $\widetilde{\Phi}$;
  \item conservativity of that right adjoint (on the Whittaker subcategory);
  \item identification of the induced monad with the tautological monad on the spectral side.
\end{enumerate}

\begin{proposition}[Existence of the right adjoint]\label{prop:Phi-right-adjoint}
The colimit-preserving functor
$\widetilde{\Phi}:\IndCoh_{\mathrm{Nilp}}(\LocSys_{\checkG})\to D_{\mathrm{lis}}(\Bun_G,\Lambda)_{\omega}$
admits a continuous right adjoint $\widetilde{\Phi}^{R}$.
\end{proposition}

\begin{proof}[Proof sketch]
Both categories are compactly generated presentable stable categories, and $\widetilde{\Phi}$ preserves colimits.
Under standard set-theoretic finiteness hypotheses (which hold in the settings considered by Fargues--Scholze),
the adjoint functor theorem produces a right adjoint.
In the later write-up, we will isolate a precise compact generation statement and cite the general existence of
adjoints in presentable stable $\infty$-categories.
\end{proof}

\begin{proposition}[Endomorphisms of the Whittaker generator]\label{prop:endo-Wpsi}
There is a canonical algebra map
\[
  \Gamma(\LocSys_{\checkG},\OO)\ \longrightarrow\ \End_{D_{\mathrm{lis}}(\Bun_G,\Lambda)}(W_{\psi})
\]
induced by the spectral action, and this map is expected to be an isomorphism.
\end{proposition}

\begin{remark}
The map in Proposition~\ref{prop:endo-Wpsi} is the categorical incarnation of the stable Bernstein center:
Fargues and Scholze construct a map from functions on the parameter stack to endomorphisms of the identity
functor on the relevant automorphic category \cite[\S 6.3]{FSReview}.
The Whittaker object is the distinguished test object where this map should be detected sharply, by analogy with
classical multiplicity one for Whittaker models.
\end{remark}

\begin{proposition}[Monadicity criterion for full faithfulness]\label{prop:monadicity-criterion}
Assume:
\begin{enumerate}[label=(\alph*), leftmargin=2em]
  \item $\widetilde{\Phi}$ admits a conservative right adjoint $\widetilde{\Phi}^R$;
  \item the endomorphism map of Proposition~\ref{prop:endo-Wpsi} is an isomorphism; and
  \item $\IndCoh_{\mathrm{Nilp}}(\LocSys_{\checkG})$ is generated under colimits by the unit object
  $\OO_{\LocSys_{\checkG}}$ under the action of $\Perf(\LocSys_{\checkG})$.
\end{enumerate}
Then $\widetilde{\Phi}$ is fully faithful, hence $\Phi^{\mathrm{c}}$ is fully faithful on compact objects.
\end{proposition}

\begin{proof}[Proof sketch]
The functor $\widetilde{\Phi}$ is a morphism of module categories for the monoidal category
$\Perf(\LocSys_{\checkG})$, sending the unit object $\OO_{\LocSys_{\checkG}}$ to $W_{\psi}$.
Under (c), the monad $\widetilde{\Phi}^R\widetilde{\Phi}$ is determined by its effect on the unit object.
By (b), this monad agrees with the tautological monad on $\IndCoh_{\mathrm{Nilp}}$ (tensoring by
$\Gamma(\LocSys_{\checkG},\OO)$), and by (a) Barr--Beck gives full faithfulness.
This is the local analogue of the monadic arguments used in \cite{GRProofIV}.
\end{proof}

\subsection{Essential surjectivity: parabolic gluing and induction on semisimple rank}\label{subsec:surjectivity}

We now explain how, once full faithfulness is established, one should prove essential surjectivity of
$\Phi^{\mathrm{c}}$ by an inductive argument on semisimple rank using parabolic functors.

\begin{proposition}[Automorphic generation]\label{prop:automorphic-generation}
The category $D_{\mathrm{lis}}(\Bun_G,\Lambda)_{\omega}$ is generated under colimits by the cuspidal objects
in $D_{\mathrm{lis}}(\Bun_G,\Lambda)_{\omega}$ together with the essential images of
$\mathrm{Eis}_P$ from Levi subgroups, where $P$ ranges over proper parabolic subgroups.
\end{proposition}

\begin{proof}[Proof sketch]
This is a direct refinement of Theorem~\ref{thm:HHS-generation} to the Whittaker-generated subcategory, using the
fact that $D_{\mathrm{lis}}(\Bun_G,\Lambda)_{\omega}$ is stable under parabolic functors (by spectral linearity and
Definition~\ref{def:omega-subcategory}).
The needed stability under $\mathrm{Eis}_P$ and $\mathrm{CT}_P$ is proved in \cite{HHSGeometricEis}.
\end{proof}

\begin{proposition}[Spectral generation]\label{prop:spectral-generation}
The category $\IndCoh_{\mathrm{Nilp}}(\LocSys_{\checkG})$ is generated under colimits by its cuspidal subcategory
together with the essential images of the spectral Eisenstein functors $\mathrm{Eis}_P^{\mathrm{spec}}$ from Levi
subgroups.
\end{proposition}

\begin{remark}
Proposition~\ref{prop:spectral-generation} is the spectral counterpart of Theorem~\ref{thm:HHS-generation}.
Its proof should use nilpotent singular support exactly as in global geometric Langlands:
compatibility of singular support with parabolic correspondences and a gluing argument along Levi strata
\cite{AGSingSupp,GRProofI}.
We include it here as a required spectral input for the inductive strategy.
\end{remark}

\begin{theorem}[Inductive essential surjectivity template]\label{thm:inductive-surjectivity}
Assume:
\begin{enumerate}[label=(\alph*), leftmargin=2em]
  \item $\widetilde{\Phi}$ is fully faithful (for $G$ and for all proper Levi subgroups of $G$);
  \item parabolic compatibility holds (Conjecture~\ref{conj:Phi-parabolic});
  \item automorphic and spectral generation statements
  (Propositions~\ref{prop:automorphic-generation} and \ref{prop:spectral-generation});
  \item $\widetilde{\Phi}$ induces an equivalence on cuspidal subcategories.
\end{enumerate}
Then $\widetilde{\Phi}$ is essentially surjective, hence an equivalence of presentable categories, and therefore
$\Phi^{\mathrm{c}}$ is an equivalence on compact objects.
\end{theorem}

\begin{proof}[Proof sketch]
Using (b), the essential images of Eisenstein series from Levi subgroups match under $\widetilde{\Phi}$, and by
induction on semisimple rank these Levi images are already in the essential image.
By (c), it remains to treat cuspidal objects; by (d) these are also in the essential image.
Thus $\widetilde{\Phi}$ is essentially surjective.
\end{proof}

\subsection{Cuspidal blocks and local multiplicity one}\label{subsec:cuspidal}

The remaining input in Theorem~\ref{thm:inductive-surjectivity} is the cuspidal equivalence.
Here one expects to use a local analogue of the multiplicity one theorem in the proof of geometric Langlands
\cite{GRProofV}, combined with the ``toy model'' description of cuspidal blocks described by
Fargues and Scholze \cite[\S 5]{FSReview}.

\begin{conjecture}[Cuspidal block description and Whittaker multiplicity one]\label{conj:cuspidal-multone}
Let $\phi$ be a cuspidal Langlands parameter for $G$ and let $S_{\phi}$ be the component group of the centralizer
of $\phi$ in $\checkG$.
Then the fiber of the Whittaker-generated category over $\phi$ is equivalent to
$\Perf(\Rep_{\Lambda}(S_{\phi}))$, and the Whittaker datum singles out a unique generic object in the packet.
\end{conjecture}

\begin{remark}
Conjecture~\ref{conj:cuspidal-multone} is the local categorical analogue of the description of cuspidal
Hecke eigensheaves and the uniqueness of Whittaker models.
It is the step at which one expects to use deeper geometry of local shtukas and the construction of Hecke
eigensheaves in \cite{FSGeometrization}, together with the explicit structure of the parameter stack in
\cite{DHKMParameters}.
\end{remark}

\subsection{Summary of the dependency chain}\label{subsec:summary-dependencies}

Collecting the discussion, a proof of Conjecture~\ref{conj:categorical-geom-whittaker} can be organized as follows:
\begin{enumerate}[label=\textbf{Task \arabic*.}, leftmargin=2.8em]
  \item Prove compactness of Whittaker translates (Theorem~\ref{thm:compactness}) using boundedness of Hecke
  correspondences (Proposition~\ref{prop:Hecke-boundedness}), gluing along Harder--Narasimhan strata
  \cite{MilesGluingHN}, and finiteness of parabolic functors \cite{HHSGeometricEis}.
  \item Establish spectral devissage (Proposition~\ref{prop:spectral-devissage}) and extend the functor to
  $\Coh_{\mathrm{Nilp}}$ (Proposition~\ref{prop:extend-Phi}).
  \item Construct spectral parabolic functors and prove parabolic compatibility (Conjecture~\ref{conj:Phi-parabolic}),
  using singular support as in \cite{AGSingSupp}.
  \item Prove full faithfulness by monadicity: show conservativity of the right adjoint and compute
  $\End(W_{\psi})$ (Proposition~\ref{prop:endo-Wpsi}), using the stable Bernstein center formalism
  \cite{FSReview} and the monadic patterns of \cite{GRProofIV}.
  \item Prove essential surjectivity by gluing from Levi subgroups (Theorem~\ref{thm:inductive-surjectivity}),
  using second adjointness on the automorphic side (Theorem~\ref{thm:HHS-adjointness}) and the spectral
  generation statement (Proposition~\ref{prop:spectral-generation}).
  \item Analyze the cuspidal case via Conjecture~\ref{conj:cuspidal-multone}, the local analogue of
  multiplicity one \cite{GRProofV}.
\end{enumerate}

This completes the proof template that will guide the remainder of the project.



%--------------------------------------------------------------------
\section{Compactness of Whittaker translates}\label{sec:compactness}
%--------------------------------------------------------------------

In this section we focus on \textbf{Task 1} from Section~\ref{subsec:summary-dependencies} and give a
proof strategy that reduces compactness of Whittaker translates to a precise representation-theoretic
finite generation statement.
We then verify the required representation-theoretic input for $G=\GL_n$ using the integral theory of
co-Whittaker modules due to Helm.

\subsection{The compactness problem}

Recall that $G/E$ is quasi-split, $\ell\neq p$, and we fix Whittaker data $(B,\psi)$.
Let $W_{\psi}\in D_{\mathrm{lis}}(\Bun_G,\Lambda)$ be the Whittaker generator
(Definition~\ref{def:whittaker-sheaf}), and let
$D_{\mathrm{lis}}(\Bun_G,\Lambda)_{\omega}$ be the Whittaker-generated subcategory
(Definition~\ref{def:omega-subcategory}).

\begin{theorem}[Compactness of Whittaker translates]\label{thm:compactness-main}
Let $F\in \Perf(\LocSys_{\checkG})$ be a perfect complex whose (cohomological) support is quasi-compact.
Then the Whittaker translate
\[
  F\ast W_{\psi}\ \in\ D_{\mathrm{lis}}(\Bun_G,\Lambda)_{\omega}
\]
is compact.
\end{theorem}

The remainder of the section gives a reduction of Theorem~\ref{thm:compactness-main} to a representation-theoretic
input on the neutral stratum, and proves that input for $G=\GL_n$.

\subsection{Perfect actions preserve compact objects}\label{subsec:perfect-actions}

The spectral action gives, for each $F\in \Perf(\LocSys_{\checkG})$, an exact colimit-preserving endofunctor
\[
  T_F \;:\; D_{\mathrm{lis}}(\Bun_G,\Lambda)\longrightarrow D_{\mathrm{lis}}(\Bun_G,\Lambda),
  \qquad \mathcal{A}\longmapsto F\ast \mathcal{A}.
\]
A key formal point is that $T_F$ preserves compact objects whenever $F$ is perfect.
This uses only rigidity of $\Perf(\LocSys_{\checkG})$.

\begin{lemma}[Rigidity implies preservation of compact objects]\label{lem:rigid-preserves-compacts}
Let $\cC$ be a compactly generated presentable stable $\infty$-category.
Let $\cM$ be a rigid symmetric monoidal stable $\infty$-category acting on $\cC$ by colimit-preserving exact
endofunctors.
Then for every $m\in \cM$, the endofunctor
\[
  T_m:\cC\to \cC,\qquad c\mapsto m\ast c
\]
preserves compact objects.
\end{lemma}

\begin{proof}
Because $\cM$ is rigid, $m$ admits a dual $m^{\vee}$.
Rigidity of the action implies that $T_{m^{\vee}}$ is both left and right adjoint to $T_m$.
Since the action is by colimit-preserving functors, the right adjoint $T_{m^{\vee}}$ preserves filtered colimits.
By the standard adjointness criterion for compactness (a left adjoint preserves compact objects if and only if its
right adjoint preserves filtered colimits), it follows that $T_m$ preserves compact objects.
\end{proof}

\begin{remark}
Lemma~\ref{lem:rigid-preserves-compacts} reduces the compactness problem in
Theorem~\ref{thm:compactness-main} to constructing \emph{one} compact object in the Whittaker category on which
$\Perf(\LocSys_{\checkG})$ acts, and then observing that all perfect translates remain compact.
\end{remark}

\subsection{Quasi-compact support and reduction to finitely many connected components}

Write
\[
  \LocSys_{\checkG}\ =\ \coprod_{\alpha\in \pi_0(\LocSys_{\checkG})}\LocSys_{\checkG,\alpha}
\]
for the decomposition into connected components (equivalently, open and closed substacks).
By Theorem~\ref{thm:DHKM} these components are algebraic stacks locally of finite type; in particular each
$\LocSys_{\checkG,\alpha}$ is quasi-compact.

\begin{lemma}[Quasi-compact support meets finitely many components]\label{lem:qc-support-finite-components}
Let $F\in \Perf(\LocSys_{\checkG})$ have quasi-compact support.
Then there exists a finite subset $I\subset \pi_0(\LocSys_{\checkG})$ such that
\[
  \mathrm{Supp}(F)\ \subset\ \coprod_{\alpha\in I}\LocSys_{\checkG,\alpha}.
\]
Equivalently, $F$ is the extension by zero of its restriction to the quasi-compact open and closed substack
\[
  \LocSys_{\checkG,I}\ :=\ \coprod_{\alpha\in I}\LocSys_{\checkG,\alpha}.
\]
\end{lemma}

\begin{proof}
A quasi-compact subset of a disjoint union of open and closed quasi-compact components meets only finitely many
components.
\end{proof}

Let $\mathcal{O}_{\LocSys_{\checkG,I}}\in \Perf(\LocSys_{\checkG})$ denote the structure sheaf of the open and closed
substack $\LocSys_{\checkG,I}$, viewed as a direct summand of the unit object
$\mathcal{O}_{\LocSys_{\checkG}}$.
Since $\LocSys_{\checkG,I}$ is open and closed, $\mathcal{O}_{\LocSys_{\checkG,I}}$ is an idempotent algebra object.

\begin{lemma}[Projector reduction]\label{lem:projector-reduction}
Let $F\in \Perf(\LocSys_{\checkG})$ have support contained in $\LocSys_{\checkG,I}$.
Then there is a canonical identification
\[
  F\ast W_{\psi}\ \simeq\ F\ast\big(\mathcal{O}_{\LocSys_{\checkG,I}}\ast W_{\psi}\big).
\]
\end{lemma}

\begin{proof}
Since $F$ is supported on $\LocSys_{\checkG,I}$, one has a canonical isomorphism
$F\simeq \mathcal{O}_{\LocSys_{\checkG,I}}\otimes F$ in $\Perf(\LocSys_{\checkG})$.
Using monoidality of the action, this gives
\[
  F\ast W_{\psi}\ \simeq\ (\mathcal{O}_{\LocSys_{\checkG,I}}\otimes F)\ast W_{\psi}
  \ \simeq\ F\ast(\mathcal{O}_{\LocSys_{\checkG,I}}\ast W_{\psi}).
\]
\end{proof}

Lemma~\ref{lem:projector-reduction} shows that, to prove compactness of $F\ast W_{\psi}$, it suffices to prove
compactness of the \emph{localized Whittaker generator}
\[
  W_{\psi,I}\ :=\ \mathcal{O}_{\LocSys_{\checkG,I}}\ast W_{\psi}
\]
for finite sets of components $I$.

\subsection{A representation-theoretic finiteness hypothesis}

The localized object $W_{\psi,I}$ is still a priori defined inside
$D_{\mathrm{lis}}(\Bun_G,\Lambda)_{\omega}$.
To obtain a usable criterion for compactness, we impose a hypothesis that identifies $W_{\psi,I}$ with
an object supported on the neutral stratum and controlled by the usual Bernstein theory of smooth representations.

Let $i_1:\Bun_G^{1}\hookrightarrow \Bun_G$ be the neutral stratum, so $\Bun_G^{1}\simeq B\underline{G(E)}$.
Write $\Rep_{\Lambda}^{\infty}(G(E))$ for the abelian category of smooth $\Lambda$-representations.
Recall that $W_{\psi}=i_{1,!}(\mathrm{c}\text{-}\mathrm{ind}_{U(E)}^{G(E)}\psi)$.

\begin{axiom}[Blockwise Whittaker finiteness]\label{ax:blockwise-finiteness}
For every finite union of connected components $\LocSys_{\checkG,I}$, the object
$W_{\psi,I}=\mathcal{O}_{\LocSys_{\checkG,I}}\ast W_{\psi}$ satisfies:
\begin{enumerate}[label=(\alph*), leftmargin=2em]
  \item (\emph{Neutral support}) $W_{\psi,I}$ is supported on the neutral stratum $\Bun_G^{1}$, so
  $W_{\psi,I}\simeq i_{1,!}(V_I)$ for a smooth representation $V_I$ of $G(E)$.
  \item (\emph{Finite generation}) The representation $V_I$ belongs to the thick subcategory generated by compact
  inductions from compact open subgroups (equivalently, $V_I$ is a compact object of
  $D(\Rep_{\Lambda}^{\infty}(G(E)))$).
\end{enumerate}
\end{axiom}

\begin{remark}
Axiom~\ref{ax:blockwise-finiteness}(a) is a compatibility statement between the spectral action and restriction
to Harder--Narasimhan strata.  It is automatic if the action of the idempotent
$\mathcal{O}_{\LocSys_{\checkG,I}}$ is induced by the stable Bernstein center on the neutral stratum.
Axiom~\ref{ax:blockwise-finiteness}(b) is a representation-theoretic finiteness statement: it asks for the
blockwise Whittaker object selected by $\LocSys_{\checkG,I}$ to be of finite type in the sense relevant for
compact generation of smooth representation categories.
\end{remark}

Assuming Axiom~\ref{ax:blockwise-finiteness}, Task~1 becomes formal.

\begin{proposition}[Compactness from blockwise finiteness]\label{prop:compactness-from-blockwise}
Assume Axiom~\ref{ax:blockwise-finiteness}.
Then Theorem~\ref{thm:compactness-main} holds.
\end{proposition}

\begin{proof}
Let $F\in \Perf(\LocSys_{\checkG})$ have quasi-compact support.
Choose a finite set of components $I$ as in Lemma~\ref{lem:qc-support-finite-components}.
By Lemma~\ref{lem:projector-reduction} we have
\[
  F\ast W_{\psi}\ \simeq\ F\ast W_{\psi,I}.
\]
By Axiom~\ref{ax:blockwise-finiteness}(b), $W_{\psi,I}$ is compact in
$D_{\mathrm{lis}}(\Bun_G,\Lambda)_{\omega}$.
By Lemma~\ref{lem:rigid-preserves-compacts}, the endofunctor $T_F$ preserves compact objects.
Therefore $F\ast W_{\psi,I}$ is compact, hence so is $F\ast W_{\psi}$.
\end{proof}


% ============================================================
% PATCH 2 (make the GL_n parts "as proved as possible": replace the “axiom”
% ax:blockwise-finiteness by a theorem+lemma package for GL_n)
%
% Suggested placement: inside \section{Compactness of Whittaker translates}
% right where Axiom~\ref{ax:blockwise-finiteness} currently appears.
% ============================================================

\subsection{Block projectors and the \texorpdfstring{$\GL_n$}{GLn} case}\label{subsec:block-projectors-gln}

We now specialize to $G=\GL_n$ and explain how the ``blockwise Whittaker finiteness'' input can be replaced
by a concrete theorem using the integral Bernstein center and local Langlands in families.

\begin{definition}[Bernstein block idempotents]\label{def:bernstein-idempotents}
Let $\mathfrak{Z}(G(E))$ be the Bernstein center of $\Rep^\infty_\Lambda(G(E))$ (with $\Lambda$ of
characteristic zero), and write
\[
  \Rep^\infty_\Lambda(G(E))\ \simeq\ \prod_{\mathfrak{s}}\Rep^\infty_\Lambda(G(E))_{\mathfrak{s}}
\]
for the Bernstein decomposition.
Let $e_{\mathfrak{s}}\in \mathfrak{Z}(G(E))$ be the central idempotent cutting out the block
$\Rep^\infty_\Lambda(G(E))_{\mathfrak{s}}$.
\end{definition}

\begin{lemma}[Components on the parameter stack and Bernstein blocks for $\GL_n$]\label{lem:components-vs-blocks}
Let $G=\GL_n$.
There is a natural finite-to-one assignment from connected components of the parameter stack
$\pi_0(\LocSys_{\checkG})$ to Bernstein blocks $\mathfrak{s}$ such that:
\begin{enumerate}[label=(\alph*), leftmargin=2em]
  \item the induced map on global functions factors the usual Bernstein center action through
  $\Gamma(\LocSys_{\checkG},\OO)$, and
  \item for any finite union of components $I\subset\pi_0(\LocSys_{\checkG})$, the idempotent
  $\OO_{\LocSys_{\checkG,I}}$ acts on the neutral stratum as the sum of the corresponding block idempotents
  $\sum_{\mathfrak{s}\in \mathfrak{S}(I)} e_{\mathfrak{s}}$.
\end{enumerate}
\end{lemma}

\begin{proof}[Proof sketch]
This is a reformulation of ``local Langlands in families'' for $\GL_n$:
one attaches to each Bernstein block a moduli space/stack of Weil--Deligne parameters (or its
$\ell$-adic avatar) carrying the universal family, and the Bernstein center identifies with functions on the
corresponding parameter space. See \cite{EmertonHelmFamilies,HelmBernsteinCenterWk,HelmMossFamilies} for the
construction and compatibility with Bernstein blocks; see also \cite{DHKMParameters} for a global algebraic
model of the parameter stack.
\end{proof}

\begin{theorem}[Blockwise Whittaker finiteness for $\GL_n$]\label{thm:blockwise-finiteness-gln}
Let $G=\GL_n$ and let $\Lambda$ be $\Z_\ell$ or a finite extension of $\Z_\ell$ (with $\ell\neq p$).
For every finite union of connected components $\LocSys_{\checkG,I}$, the localized Whittaker object
\[
  W_{\psi,I}\ :=\ \OO_{\LocSys_{\checkG,I}}\ast W_{\psi}
\]
is supported on the neutral stratum and corresponds there to a finite direct sum of Helm's universal
co-Whittaker modules in the associated generic Bernstein blocks. In particular $W_{\psi,I}$ is compact in
$D_{\mathrm{lis}}(\Bun_G,\Lambda)_{\omega}$.
\end{theorem}

\begin{proof}[Proof sketch]
By Lemma~\ref{lem:components-vs-blocks}, the projector $\OO_{\LocSys_{\checkG,I}}$ acts on the neutral stratum as
a finite sum of Bernstein block idempotents.  The Whittaker generator restricts to the universal Gelfand--Graev
representation, and in each generic block Helm constructs a universal co-Whittaker module which is a finitely
generated projective generator and whose endomorphism algebra is the integral block center
\cite{HelmWhittakerBernstein}. Thus the projected object is a finite sum of compact generators on the neutral
stratum, hence compact after extension by zero.
\end{proof}

\begin{corollary}[Compactness of Whittaker translates for $\GL_n$]\label{cor:compactness-gln}
For $G=\GL_n$, Theorem~\ref{thm:compactness-main} holds (for $\Lambda=\Z_\ell$ and hence after inverting $\ell$).
\end{corollary}

\begin{proof}
Combine Theorem~\ref{thm:blockwise-finiteness-gln} with Proposition~\ref{prop:compactness-from-blockwise}
(and the rigidity argument Lemma~\ref{lem:rigid-preserves-compacts}).
\end{proof}





\subsection{Verification for $G=\GL_n$}

We now verify Axiom~\ref{ax:blockwise-finiteness} for $G=\GL_n$ (and hence prove Task~1 in that case) using the
integral theory of co-Whittaker modules and the integral Bernstein center developed by Helm.

Fix $G=\GL_n$ over $E$ and assume $\Lambda=\Z_\ell$ (or a finite extension of $\Z_\ell$).
Let $\Rep_{\Lambda}^{\infty}(G(E))_{\mathfrak{s}}$ be a Bernstein block.
Helm constructs in each block a \emph{universal co-Whittaker module}
$\mathcal{W}_{\mathfrak{s}}$ which is a finitely generated projective generator of the generic subcategory of the
block and whose endomorphism ring is the integral Bernstein center of the block
\cite{HelmWhittakerBernstein}.

\begin{theorem}[Compactness for $\GL_n$]\label{thm:compactness-GLn}
Let $G=\GL_n$ and let $F\in \Perf(\LocSys_{\checkG})$ have quasi-compact support.
Then $F\ast W_{\psi}$ is compact in $D_{\mathrm{lis}}(\Bun_G,\Z_\ell)_{\omega}$.
\end{theorem}

\begin{proof}[Proof sketch]
By Lemma~\ref{lem:qc-support-finite-components}, the support of $F$ is contained in a finite union of connected
components $\LocSys_{\checkG,I}$.
The action of $\mathcal{O}_{\LocSys_{\checkG,I}}$ on $W_{\psi}$ produces an object $W_{\psi,I}$.
For $\GL_n$, the comparison between the connected components of the parameter stack and Bernstein blocks
identifies $W_{\psi,I}$ (on the neutral stratum) with the direct sum of the universal co-Whittaker modules
$\mathcal{W}_{\mathfrak{s}}$ for the finitely many generic blocks corresponding to $I$.
By \cite{HelmWhittakerBernstein}, each $\mathcal{W}_{\mathfrak{s}}$ is finitely generated projective in its block,
hence defines a compact object of the derived category of smooth representations, and therefore
$i_{1,!}(\mathcal{W}_{\mathfrak{s}})$ is compact on $\Bun_G$.
It follows that $W_{\psi,I}$ is compact.

Finally, by Lemma~\ref{lem:rigid-preserves-compacts}, the endofunctor $T_F$ preserves compact objects.
Thus $F\ast W_{\psi}\simeq F\ast W_{\psi,I}$ is compact.
\end{proof}

\begin{remark}
The only step that is not yet written in complete detail in this section is the explicit comparison between
$\mathcal{O}_{\LocSys_{\checkG,I}}\ast W_{\psi}$ and the direct sum of Helm's universal co-Whittaker modules on the
neutral stratum.
This comparison is precisely the expected compatibility between:
\begin{itemize}[leftmargin=2em]
  \item the spectral action of Fargues and Scholze on the neutral stratum, and
  \item the integral Bernstein center description of the generic representation theory of $\GL_n$.
\end{itemize}
Making this comparison completely formal will be one of the main ``local calculations'' to include in a full
write-up.
\end{remark}

\subsection{Compatibility with constant term and reduction to Levi subgroups}

We end by recording that the compactness statement is compatible with parabolic restriction, in a form that
uses the constant term computation of Section~\ref{sec:CT-Whittaker}.

Let $P\subset G$ be a parabolic with Levi quotient $M$, and let $P^{-}$ be the opposite parabolic.
Let
$f_M^G:\LocSys_{\checkG}\to \LocSys_{\checkM}$ be the induced morphism of parameter stacks.

\begin{proposition}[Constant term preserves compactness for Whittaker translates]\label{prop:CT-compactness}
Assume Theorem~\ref{thm:compactness-main} holds for the Levi $M$.
Then for every $F\in \Perf(\LocSys_{\checkG})$ of quasi-compact support, the object
$\mathrm{CT}_{P^{-}}(F\ast W_{\psi})$ is compact in $D_{\mathrm{lis}}(\Bun_M,\Lambda)_{\omega}$.
\end{proposition}

\begin{proof}
By Corollary~\ref{cor:CT-on-translates} one has
\[
  \mathrm{CT}_{P^{-}}(F\ast W_{\psi})
  \ \simeq\
  (f_M^G)^{\ast}(F)\ast W_{\psi_M}.
\]
The pullback $(f_M^G)^{\ast}(F)$ is again perfect with quasi-compact support.
By the compactness theorem for $M$, the right-hand side is compact.
\end{proof}

\begin{remark}
Proposition~\ref{prop:CT-compactness} is the first indication that compactness of Whittaker translates should be
accessible by induction on semisimple rank, once the representation-theoretic finiteness package is established
uniformly for Levi subgroups.
\end{remark}




%--------------------------------------------------------------------
\section{Spectral d\'evissage and extension of the Whittaker functor}\label{sec:task2}
%--------------------------------------------------------------------

In this section we address \textbf{Task 2} from Section~\ref{subsec:summary-dependencies}:
\begin{itemize}[leftmargin=2em]
  \item establish a workable d\'evissage principle on the spectral side (a description of
  $\IndCoh_{\mathrm{Nilp}}(\LocSys_{\checkG})$ and its compact objects in terms of explicit generators), and
  \item explain how, once Task~1 is known, the Whittaker comparison functor extends from perfect complexes to a
  functor on coherent objects with nilpotent singular support.
\end{itemize}
The main conceptual point is that the nilpotent singular support condition is formulated inside the category
$\IndCoh(\LocSys_{\checkG})$, whereas the spectral action constructed by Fargues and Scholze is initially an
action of $\Perf(\LocSys_{\checkG})$ on the automorphic category.
Thus, to extend the Whittaker functor beyond perfect complexes, one needs a spectral d\'evissage that
produces $\Coh_{\mathrm{Nilp}}(\LocSys_{\checkG})$ from explicit compact generators on which we can define images
geometrically.

\subsection{Ind-coherent sheaves, singular support, and the nilpotent subcategory}

We retain the notation and assumptions of Section~\ref{sec:spectral}.
In particular, we fix a quasi-smooth derived enhancement of the parameter stack $\LocSys_{\checkG}$
(Axiom~\ref{ax:LocSys-quasi-smooth}), so that singular support is defined in the sense of
Arinkin and Gaitsgory \cite{AGSingSupp}.

\begin{definition}[The nilpotent ind-coherent category]\label{def:IndCohNilp-local}
Let $\mathrm{Nilp}(\LocSys_{\checkG})\subset \mathrm{Sing}(\LocSys_{\checkG})$ be the nilpotent conical substack
defined in Axiom~\ref{ax:nilpotent-map}.  Define
\[
  \IndCoh_{\mathrm{Nilp}}(\LocSys_{\checkG})
  \ :=\
  \IndCoh_{\mathrm{Nilp}(\LocSys_{\checkG})}(\LocSys_{\checkG})
\]
to be the full subcategory of $\IndCoh(\LocSys_{\checkG})$ consisting of objects whose singular support is
contained in $\mathrm{Nilp}(\LocSys_{\checkG})$.

Define
\[
  \Coh_{\mathrm{Nilp}}(\LocSys_{\checkG})
  \ :=\
  \Big(\IndCoh_{\mathrm{Nilp}}(\LocSys_{\checkG})\Big)^{\mathrm{c}}
\]
to be the full subcategory of compact objects (equivalently, coherent objects with nilpotent singular support).
\end{definition}

\begin{proposition}[Compact generation and compacts]\label{prop:IndCohNilp-compactly-generated}
The presentable category $\IndCoh_{\mathrm{Nilp}}(\LocSys_{\checkG})$ is compactly generated, and its compact
objects are precisely $\Coh_{\mathrm{Nilp}}(\LocSys_{\checkG})$.
\end{proposition}

\begin{proof}[Proof sketch]
For a quasi-smooth derived stack $X$ locally of finite type, $\IndCoh(X)$ is compactly generated and its compact
objects identify with the bounded coherent complexes.
The singular support formalism of \cite{AGSingSupp} produces, for each conical closed substack
$\mathcal{Y}\subset \mathrm{Sing}(X)$, a reflective (colimit-closed) subcategory $\IndCoh_{\mathcal{Y}}(X)$, and
one proves that it is again compactly generated with compact objects those coherent complexes whose singular
support is contained in $\mathcal{Y}$.
Applying this with $X=\LocSys_{\checkG}$ and $\mathcal{Y}=\mathrm{Nilp}(\LocSys_{\checkG})$ gives the claim.
\end{proof}

\subsection{Perfect complexes lie in the nilpotent subcategory}

The category $\Perf(\LocSys_{\checkG})$ is the input for the spectral action.  We record that perfect complexes
have \emph{zero} singular support, hence are automatically nilpotent.

\begin{lemma}[Perfect implies nilpotent singular support]\label{lem:Perf-in-Nilp}
The natural functor $\Perf(\LocSys_{\checkG})\to \IndCoh(\LocSys_{\checkG})$ factors through
$\IndCoh_{\mathrm{Nilp}}(\LocSys_{\checkG})$.
Equivalently, every perfect complex has singular support contained in the zero section, hence in
$\mathrm{Nilp}(\LocSys_{\checkG})$.
\end{lemma}

\begin{proof}[Proof sketch]
This is a general property of singular support in the sense of \cite{AGSingSupp}:
objects coming from $\QCoh$ (and in particular perfect complexes) have singular support contained in the zero
section of $\mathrm{Sing}(X)$.
\end{proof}

\begin{remark}
Lemma~\ref{lem:Perf-in-Nilp} shows that the Whittaker functor defined on perfect complexes is automatically a
functor into the nilpotent ind-coherent category on the spectral side.
The difficulty of Task~2 is not to land in the nilpotent subcategory, but to extend beyond perfect complexes.
\end{remark}

\subsection{Ind-extension from perfect complexes to quasi-coherent sheaves}

The action of $\Perf(\LocSys_{\checkG})$ on the automorphic category is defined in
Theorem~\ref{thm:spectral-action}.  Since $\Perf(\LocSys_{\checkG})$ is a small rigid monoidal category and
$D_{\mathrm{lis}}(\Bun_G,\Lambda)$ is presentable, the action extends formally to an action of
$\QCoh(\LocSys_{\checkG})=\Ind(\Perf(\LocSys_{\checkG}))$ by left Kan extension.

\begin{proposition}[Extension of the spectral action to $\QCoh$]\label{prop:QCoh-action}
There is a canonical extension of the spectral action to a colimit-preserving monoidal action
\[
  \QCoh(\LocSys_{\checkG})\ \curvearrowright\ D_{\mathrm{lis}}(\Bun_G,\Lambda)
\]
whose restriction to $\Perf(\LocSys_{\checkG})$ agrees with the original spectral action.
\end{proposition}

\begin{proof}[Proof sketch]
The $\Perf(\LocSys_{\checkG})$-action is by exact colimit-preserving endofunctors.
Since $\QCoh(\LocSys_{\checkG})$ is the ind-completion of $\Perf(\LocSys_{\checkG})$ as a symmetric monoidal
category, the universal property of ind-completion gives a unique extension of the action to $\QCoh$.
\end{proof}

\begin{definition}[The Whittaker functor on quasi-coherent sheaves]\label{def:Phi-QCoh}
Let $\mathcal{W}_{\psi}$ be the Whittaker generator on $\Bun_G$ (Section~\ref{sec:whittaker}).
Define the colimit-preserving functor
\[
  \Phi_{\QCoh}\;:\; \QCoh(\LocSys_{\checkG})\longrightarrow D_{\mathrm{lis}}(\Bun_G,\Lambda)_{\omega},
  \qquad \mathcal{F}\longmapsto \mathcal{F}\ast \mathcal{W}_{\psi},
\]
using the $\QCoh$-action from Proposition~\ref{prop:QCoh-action}.
\end{definition}

\begin{remark}
The functor $\Phi_{\QCoh}$ is completely formal once the spectral action is available.
What is nontrivial, and was the content of Task~1, is compactness when one restricts from $\QCoh$ to perfect
complexes of quasi-compact support.
\end{remark}

\subsection{What Task 2 really requires: generators for $\Coh_{\mathrm{Nilp}}$}

Task~2 asks for an extension
\[
  \Phi^{\mathrm{c}}\;:\;\Coh_{\mathrm{Nilp}}(\LocSys_{\checkG})
  \longrightarrow D_{\mathrm{lis}}(\Bun_G,\Lambda)_{\omega}^{\mathrm{c}}
\]
whose restriction to perfect complexes is $F\mapsto F\ast \mathcal{W}_{\psi}$.

At this point it is important to emphasize that, in general, the inclusion
$\Perf(\LocSys_{\checkG})\subset \Coh_{\mathrm{Nilp}}(\LocSys_{\checkG})$ is \emph{not} essentially surjective:
coherent objects with nilpotent singular support can have nontrivial singularities and need not be perfect.
Thus, extending $\Phi$ is \emph{not} a tautological thick-closure argument.
Instead, one needs a d\'evissage that presents $\Coh_{\mathrm{Nilp}}$ by explicit compact generators whose
geometry can be related to the automorphic side.

The general singular support formalism provides such a d\'evissage, but we isolate it as an input to be proved
(or imported) in our local setting.

\begin{axiom}[Spectral d\'evissage by quasi-smooth maps]\label{ax:spectral-devissage}
The category $\IndCoh_{\mathrm{Nilp}}(\LocSys_{\checkG})$ is generated under colimits by the essential images of
pushforwards
\[
  f_\ast:\IndCoh(Y)\to \IndCoh(\LocSys_{\checkG}),
\]
where $f:Y\to \LocSys_{\checkG}$ ranges over quasi-smooth maps from quasi-compact quasi-smooth derived schemes $Y$
such that the induced map on singularity stacks satisfies
\[
  \mathrm{Sing}(Y)\ \longrightarrow\ \mathrm{Sing}(\LocSys_{\checkG})
\quad\text{has image contained in}\quad
  \mathrm{Nilp}(\LocSys_{\checkG}).
\]
Equivalently, $\Coh_{\mathrm{Nilp}}(\LocSys_{\checkG})$ is the smallest idempotent-complete stable subcategory of
$\IndCoh_{\mathrm{Nilp}}(\LocSys_{\checkG})$ containing all $f_\ast(\omega_Y)$ for such maps $f$ (where $\omega_Y$
denotes the dualizing object on $Y$).
\end{axiom}

\begin{remark}
Axiom~\ref{ax:spectral-devissage} is a local variant of standard generation results in the ind-coherent theory
with singular support (compare \cite{AGSingSupp} and the role of ``restricted variation'' in
\cite{AGKRRVRestrictedVariation}).
In later versions of this paper, we will replace it by an explicit theorem with precise hypotheses on
$\LocSys_{\checkG}$ (for example, quasi-compactness assumptions on the relevant substacks).
\end{remark}

\subsection{A general extension lemma for compactly generated targets}

We now record the purely categorical mechanism that converts Task~1 plus a spectral d\'evissage into an extension
of the Whittaker functor.

\begin{lemma}[Extension from compact generators]\label{lem:extend-from-generators}
Let $\cS$ be a compactly generated presentable stable $\infty$-category and let $\cA$ be a presentable stable
$\infty$-category.
Assume $\cS$ is generated under colimits by a set of compact objects $\mathcal{G}\subset \cS^{\mathrm{c}}$.
Then:
\begin{enumerate}[label=(\alph*), leftmargin=2em]
  \item any assignment $\mathcal{G}\to \cA$ extends uniquely (up to contractible choice) to an exact colimit-preserving
  functor $\cS\to \cA$ if and only if it is compatible with all finite colimits among objects of $\mathcal{G}$;
  \item the resulting functor carries compact objects of $\cS$ to compact objects of $\cA$ if and only if it
  carries every element of $\mathcal{G}$ to a compact object of $\cA$.
\end{enumerate}
\end{lemma}

\begin{proof}[Proof sketch]
This is the standard universal property of compact generation:
$\cS\simeq \Ind(\cS^{\mathrm{c}})$ and colimit-preserving functors out of $\cS$ are determined by their values on
compact generators.
\end{proof}

\subsection{Extension of the Whittaker functor to $\Coh_{\mathrm{Nilp}}$: reduction to images of generators}

We can now isolate the exact remaining content of Task~2.

\begin{proposition}[Task~2 reduction]\label{prop:task2-reduction}
Assume:
\begin{enumerate}[label=(\alph*), leftmargin=2em]
  \item Task~1: for every perfect complex $F$ on $\LocSys_{\checkG}$ with quasi-compact support,
  $F\ast \mathcal{W}_{\psi}$ is compact in $D_{\mathrm{lis}}(\Bun_G,\Lambda)_{\omega}$;
  \item the spectral d\'evissage axiom (Axiom~\ref{ax:spectral-devissage}); and
  \item for every generator $f_\ast(\omega_Y)$ appearing in Axiom~\ref{ax:spectral-devissage}, one has a
  canonically defined compact object
  \[
    \Phi^{\mathrm{c}}_{\mathrm{gen}}\big(f_\ast(\omega_Y)\big)\ \in\ D_{\mathrm{lis}}(\Bun_G,\Lambda)_{\omega}^{\mathrm{c}}
  \]
  compatible with pullback along maps of generators and with the $\Perf(\LocSys_{\checkG})$-action.
\end{enumerate}
Then there exists an exact functor
\[
  \Phi^{\mathrm{c}}\;:\;\Coh_{\mathrm{Nilp}}(\LocSys_{\checkG})\longrightarrow
  D_{\mathrm{lis}}(\Bun_G,\Lambda)_{\omega}^{\mathrm{c}}
\]
extending the prescription $F\mapsto F\ast \mathcal{W}_{\psi}$ on perfect complexes with quasi-compact support.
Moreover, $\Phi^{\mathrm{c}}$ is uniquely determined by its values on the compact generators
$f_\ast(\omega_Y)$.
\end{proposition}

\begin{proof}[Proof sketch]
By Proposition~\ref{prop:IndCohNilp-compactly-generated} and Axiom~\ref{ax:spectral-devissage}, the category
$\IndCoh_{\mathrm{Nilp}}(\LocSys_{\checkG})$ is compactly generated by the stated compact objects.
By Lemma~\ref{lem:extend-from-generators}, any compatible assignment on these generators extends uniquely to an
exact colimit-preserving functor
\[
  \widetilde{\Phi}\;:\;\IndCoh_{\mathrm{Nilp}}(\LocSys_{\checkG})\to D_{\mathrm{lis}}(\Bun_G,\Lambda)_{\omega}
\]
carrying generators to compact objects.
Restricting $\widetilde{\Phi}$ to compact objects yields the desired functor $\Phi^{\mathrm{c}}$ on
$\Coh_{\mathrm{Nilp}}$.
The agreement with $F\mapsto F\ast \mathcal{W}_{\psi}$ on perfect complexes follows from (a) and the required
compatibility in (c).
\end{proof}

\subsection{A concrete proposal for the generators: parabolic (spectral) objects}

Proposition~\ref{prop:task2-reduction} shifts the problem of Task~2 from ``extend from perfect complexes''
to ``define images of compact generators of $\IndCoh_{\mathrm{Nilp}}$''.
In geometric Langlands, a natural and representation-theoretically meaningful choice of such generators comes
from parabolic induction on the spectral side.

In our local setting, a corresponding supply of quasi-smooth maps $f:Y\to \LocSys_{\checkG}$ is expected to be
given by parameter stacks for dual parabolics:
for each parabolic $P\subset G$ with Levi quotient $M$, let $\checkP\subset \checkG$ be the dual parabolic with Levi
$\checkM$, and consider the natural map
\[
  f_P:\LocSys_{\checkP}\longrightarrow \LocSys_{\checkG}.
\]
One expects that the objects $f_{P,\ast}(\omega_{\LocSys_{\checkP}})$ have nilpotent singular support and that,
together with the cuspidal part, they generate $\IndCoh_{\mathrm{Nilp}}(\LocSys_{\checkG})$ under colimits
(compare the role of Eisenstein series and gluing in \cite{AGSingSupp}).

We will return to these generators in Task~3 (construction of spectral parabolic functors) and Task~5
(gluing and essential surjectivity).
For Task~2, the outcome is:

\begin{remark}[Conclusion of Task~2 as a reduction]
After Task~1 is proved, the remaining work in Task~2 is to pick a concrete generating family
$\{f_\ast(\omega_Y)\}$ for $\IndCoh_{\mathrm{Nilp}}(\LocSys_{\checkG})$ and to define their images in the
Whittaker category in a way compatible with the $\Perf(\LocSys_{\checkG})$-action.
The categorical extension then follows formally from Lemma~\ref{lem:extend-from-generators}.
\end{remark}



%--------------------------------------------------------------------
\section{Spectral parabolic functors and parabolic compatibility}\label{sec:task3}
%--------------------------------------------------------------------

In this section we address \textbf{Task 3} from Section~\ref{subsec:summary-dependencies}:
construct spectral parabolic functors and prove parabolic compatibility of the Whittaker functor.

There are three distinct issues:
\begin{enumerate}[label=(\roman*), leftmargin=2em]
  \item \textbf{Geometry:} define the spectral correspondence
  $\LocSys_{\checkM}\xleftarrow{\,q\,}\LocSys_{\checkP}\xrightarrow{\,p\,}\LocSys_{\checkG}$ and verify the
  finiteness and quasi-smoothness properties needed to define ind-coherent functors.
  \item \textbf{Microlocal control:} show that the functors preserve nilpotent singular support, so that they
  descend to $\IndCoh_{\mathrm{Nilp}}$.
  \item \textbf{Compatibility:} show that the (conjectural) Whittaker equivalence intertwines automorphic and
  spectral parabolic functors.  In practice, this is established first on perfect (or quasi-coherent) objects,
  using linearity and the explicit constant term computation of Section~\ref{sec:CT-Whittaker}, and then extended
  to $\IndCoh_{\mathrm{Nilp}}$ by the d\'evissage mechanism of Section~\ref{sec:task2}.
\end{enumerate}

We work throughout with $\ell\neq p$ and coefficients $\Lambda$ of characteristic zero (for simplicity in the
discussion of duality and adjunctions on the spectral side).

%--------------------------------------------------------------------
\subsection{The spectral parabolic correspondence}\label{subsec:spectral-parabolic-correspondence}
%--------------------------------------------------------------------

Fix a parabolic subgroup $P\subset G$ and let $M$ be a Levi quotient.
Write $\checkP\subset \checkG$ and $\checkM\subset \checkG$ for the dual parabolic and dual Levi.

\begin{definition}[Spectral stack with parabolic structure]\label{def:LocSysP}
Let $\LocSys_{\checkG}$ and $\LocSys_{\checkM}$ be the parameter stacks of Langlands parameters as in
Section~\ref{sec:spectral} (using the Dat--Helm--Kurinczuk--Moss algebraic model).
Define $\LocSys_{\checkP}$ to be the stack classifying \emph{pairs}
\[
  (\phi,\mathcal{P}),
\]
where $\phi$ is a $\checkG$-valued Langlands parameter and $\mathcal{P}$ is a reduction of $\phi$ to $\checkP$
(up to $\checkP$-conjugacy).
Equivalently, $\LocSys_{\checkP}$ is the stack of $\checkP$-valued parameters, modulo $\checkP$-conjugacy,
together with the natural map to $\LocSys_{\checkG}$ induced by $\checkP\hookrightarrow \checkG$.
\end{definition}

\begin{remark}[Two realizations of $\LocSys_{\checkP}$]
There are (at least) two natural realizations of $\LocSys_{\checkP}$.
\begin{enumerate}[label=(\alph*), leftmargin=2em]
  \item \textbf{Parameters valued in $\checkP$:} define $\LocSys_{\checkP}$ as the moduli of parameters with values
  in $\checkP$, and use $\checkP\hookrightarrow \checkG$ and $\checkP\twoheadrightarrow \checkM$ to get the maps
  $p$ and $q$ below.
  \item \textbf{Reductions of $\checkG$-parameters:} define $\LocSys_{\checkP}$ as the moduli of $\checkG$-parameters
  equipped with a $\checkP$-reduction.  This has the advantage that the map
  $p:\LocSys_{\checkP}\to \LocSys_{\checkG}$ is represented by a (relative) flag variety and is therefore
  quasi-compact and often proper-like, which is convenient for defining pushforwards.
\end{enumerate}
In a future version we will choose one construction and verify its properties in the DHKM framework.
For the blueprint, it suffices that \emph{some} reasonable $\LocSys_{\checkP}$ exists with the expected maps.
\end{remark}

\begin{definition}[The parabolic correspondence]\label{def:parabolic-correspondence}
Let
\[
  \LocSys_{\checkM}\xleftarrow{\,q\,}\LocSys_{\checkP}\xrightarrow{\,p\,}\LocSys_{\checkG}
\]
be the correspondence where:
\begin{itemize}[leftmargin=2em]
  \item $p$ forgets the $\checkP$-reduction (or composes with $\checkP\hookrightarrow \checkG$);
  \item $q$ remembers only the induced $\checkM$-parameter (or composes with $\checkP\twoheadrightarrow \checkM$).
\end{itemize}
\end{definition}

\begin{axiom}[Geometric hypotheses for spectral parabolic functors]\label{ax:spectral-parabolic-geometry}
The correspondence of Definition~\ref{def:parabolic-correspondence} admits derived enhancements making all three
stacks quasi-smooth, and the morphisms $p$ and $q$ satisfy the finiteness hypotheses required to define the
ind-coherent functors $p_\ast$, $q_\ast$, $p^!$, $q^!$ and to apply base change and projection formulas in the
range used below.
\end{axiom}

\begin{remark}
Axiom~\ref{ax:spectral-parabolic-geometry} is the local analogue of the algebro-geometric properties of
$\LocSys_{\checkP}$ in global geometric Langlands.
In the present local setting, the existence and finiteness properties should ultimately be extracted from the
DHKM description of parameter stacks and the properness of the flag variety of parabolics.
\end{remark}

%--------------------------------------------------------------------
\subsection{Definition of spectral constant term and Eisenstein functors}\label{subsec:define-spectral-parabolic}
%--------------------------------------------------------------------

Assuming Axiom~\ref{ax:spectral-parabolic-geometry}, we define spectral parabolic functors by the same formulas as
in (derived) geometric Langlands.

\begin{definition}[Spectral Eisenstein and constant term functors]\label{def:spectral-parabolic-functors}
Define functors
\[
  \mathrm{Eis}_{P}^{\mathrm{spec}}
  \ :=\
  p_\ast\circ q^!
  \;:\;
  \IndCoh(\LocSys_{\checkM})\longrightarrow \IndCoh(\LocSys_{\checkG}),
\]
\[
  \mathrm{CT}_{P}^{\mathrm{spec}}
  \ :=\
  q_\ast\circ p^!
  \;:\;
  \IndCoh(\LocSys_{\checkG})\longrightarrow \IndCoh(\LocSys_{\checkM}),
\]
with the usual normalization conventions (suppressed) matching those used on the automorphic side in
Section~\ref{sec:parabolic}.
\end{definition}

\begin{remark}[Adjunction on the spectral side]
Under standard finiteness conditions, $p_\ast$ is right adjoint to $p^!$ and $q_\ast$ is right adjoint to $q^!$.
It follows that $\mathrm{Eis}_P^{\mathrm{spec}}$ is left adjoint to $\mathrm{CT}_P^{\mathrm{spec}}$.
A second adjointness statement on the spectral side is expected to hold after replacing $P$ by the opposite
parabolic $P^{-}$, mirroring Theorem~\ref{thm:HHS-adjointness} on the automorphic side.
\end{remark}

%--------------------------------------------------------------------
\subsection{Nilpotent singular support is preserved}\label{subsec:nilpotent-preservation}
%--------------------------------------------------------------------

The key microlocal input is that $\mathrm{Eis}_P^{\mathrm{spec}}$ and $\mathrm{CT}_P^{\mathrm{spec}}$ preserve the
nilpotent singular support condition.  This is the local analogue of the corresponding statements in global
geometric Langlands \cite{AGSingSupp}.

We first recall the general functoriality of singular support for quasi-smooth maps.

\begin{theorem}[Functoriality of singular support, schematic form]\label{thm:SS-functoriality}
Let $f:X\to Y$ be a quasi-smooth morphism of quasi-smooth derived stacks.
Then there are canonical maps of conical stacks
\[
  \mathrm{Sing}(X)\ \xleftarrow{\,f^{\mathrm{Sing}}\,}\ \mathrm{Sing}(X\times_Y X)\ \xrightarrow{\,g^{\mathrm{Sing}}\,}\ \mathrm{Sing}(Y)
\]
and one has the following containment statements for singular support:
\begin{enumerate}[label=(\alph*), leftmargin=2em]
  \item (\emph{Pullback}) For $\mathcal{F}\in \IndCoh(Y)$,
  \[
    \mathrm{SS}(f^!\mathcal{F})\ \subset\ (f^{\mathrm{Sing}})^{-1}\big(\mathrm{SS}(\mathcal{F})\big).
  \]
  \item (\emph{Pushforward}) For $\mathcal{G}\in \IndCoh(X)$ for which $f_\ast$ is defined,
  \[
    \mathrm{SS}(f_\ast\mathcal{G})\ \subset\ \overline{g^{\mathrm{Sing}}\!\left(\mathrm{SS}(\mathcal{G})\right)}.
  \]
\end{enumerate}
\end{theorem}

\begin{remark}
Theorem~\ref{thm:SS-functoriality} summarizes standard properties in the Arinkin--Gaitsgory theory of singular
support for ind-coherent sheaves \cite{AGSingSupp}.  In a complete write-up, we will replace this schematic
statement by a precise citation (or by a tailored lemma) in the form needed here.
\end{remark}

We now apply this formalism to the parabolic correspondence.

\begin{definition}[Nilpotent cones for Levi and parabolic stacks]\label{def:nilp-Levi-parabolic}
Let $\mathrm{Nilp}(\LocSys_{\checkG})\subset \mathrm{Sing}(\LocSys_{\checkG})$ be the nilpotent cone defined in
Section~\ref{sec:spectral}.  Define $\mathrm{Nilp}(\LocSys_{\checkM})$ and $\mathrm{Nilp}(\LocSys_{\checkP})$
analogously, using the corresponding nilpotent cones in $\Lie(\checkM)^\ast$ and $\Lie(\checkP)^\ast$.
\end{definition}

\begin{proposition}[Parabolic maps respect nilpotent cones]\label{prop:nilp-map-respects}
Under the morphisms induced by $\checkM\hookrightarrow \checkG$ and $\checkP\hookrightarrow \checkG$, the
nilpotent cones map to nilpotent cones.
Equivalently, the pullbacks of $\mathrm{Nilp}(\LocSys_{\checkG})$ to $\mathrm{Sing}(\LocSys_{\checkM})$ and
$\mathrm{Sing}(\LocSys_{\checkP})$ contain $\mathrm{Nilp}(\LocSys_{\checkM})$ and $\mathrm{Nilp}(\LocSys_{\checkP})$,
respectively.
\end{proposition}

\begin{proof}[Proof sketch]
At the level of Lie algebras, the inclusion $\Lie(\checkM)\hookrightarrow \Lie(\checkG)$ sends nilpotent elements
to nilpotent elements.  The nilpotent singular support condition is defined by pulling back the nilpotent cone
along the map $\chi:\mathrm{Sing}(\LocSys_{\checkH})\to [\Lie(\checkH)^\ast/\checkH]$ (Axiom~\ref{ax:nilpotent-map}).
Compatibility of $\chi$ with homomorphisms of groups gives the result.
\end{proof}

\begin{theorem}[Spectral parabolic functors preserve nilpotent singular support]\label{thm:parabolic-preserve-nilp}
Assume Axiom~\ref{ax:spectral-parabolic-geometry}.
Then the functors of Definition~\ref{def:spectral-parabolic-functors} restrict to functors
\[
  \mathrm{Eis}_P^{\mathrm{spec}}:\IndCoh_{\mathrm{Nilp}}(\LocSys_{\checkM})\to \IndCoh_{\mathrm{Nilp}}(\LocSys_{\checkG}),
\]
\[
  \mathrm{CT}_P^{\mathrm{spec}}:\IndCoh_{\mathrm{Nilp}}(\LocSys_{\checkG})\to \IndCoh_{\mathrm{Nilp}}(\LocSys_{\checkM}).
\]
Moreover, these restricted functors preserve compact objects, hence induce functors on coherent subcategories:
\[
  \mathrm{Eis}_P^{\mathrm{spec}}:\Coh_{\mathrm{Nilp}}(\LocSys_{\checkM})\to \Coh_{\mathrm{Nilp}}(\LocSys_{\checkG}),
  \qquad
  \mathrm{CT}_P^{\mathrm{spec}}:\Coh_{\mathrm{Nilp}}(\LocSys_{\checkG})\to \Coh_{\mathrm{Nilp}}(\LocSys_{\checkM}).
\]
\end{theorem}

\begin{proof}[Proof sketch]
By Theorem~\ref{thm:SS-functoriality}, singular support behaves functorially for $p^!$, $q^!$, $p_\ast$, $q_\ast$.
The nilpotent condition is defined by containment of singular support in a conical closed subset.
Proposition~\ref{prop:nilp-map-respects} ensures that the inverse images and direct images of nilpotent cones under
the maps induced by $p$ and $q$ remain nilpotent.
Therefore the composition $p_\ast q^!$ and $q_\ast p^!$ preserve nilpotent support.
Compactness preservation follows from compact generation of ind-coherent categories and finiteness hypotheses on
$p$ and $q$ (part of Axiom~\ref{ax:spectral-parabolic-geometry}).
\end{proof}

\begin{remark}
Theorem~\ref{thm:parabolic-preserve-nilp} is the precise place where the nilpotent singular support condition is
used: without restricting to $\IndCoh_{\mathrm{Nilp}}$, the parabolic functors are not expected to satisfy the
required adjunction and gluing properties.
\end{remark}

%--------------------------------------------------------------------
\subsection{Parabolic compatibility on the Whittaker generator: constant term}\label{subsec:parabolic-compat-CT}
%--------------------------------------------------------------------

We now turn to compatibility with the automorphic parabolic functors.
We begin with constant term, because it is accessible on the automorphic side by the explicit computation of
Section~\ref{sec:CT-Whittaker} and because it is the input from which Eisenstein compatibility can be deduced by
adjunction.

Let $\mathrm{CT}_{P^{-}}$ be the normalized automorphic constant term functor
(Section~\ref{sec:parabolic}).
Let $\mathrm{CT}_{P^{-}}^{\mathrm{spec}}$ be the spectral constant term functor of
Definition~\ref{def:spectral-parabolic-functors}, restricted to $\IndCoh_{\mathrm{Nilp}}$ by
Theorem~\ref{thm:parabolic-preserve-nilp}.

\begin{proposition}[Compatibility on perfect objects]\label{prop:CT-compat-Perf}
Let $F\in \Perf(\LocSys_{\checkG})$ have quasi-compact support.
Then there is a canonical equivalence in $D_{\mathrm{lis}}(\Bun_M,\Lambda)$
\[
  \mathrm{CT}_{P^{-}}\big(F\ast W_{\psi}\big)
  \ \simeq\
  \big((f_M^G)^\ast F\big)\ast W_{\psi_M},
\]
where $f_M^G:\LocSys_{\checkG}\to \LocSys_{\checkM}$ is the morphism induced by $\checkM\hookrightarrow \checkG$.
\end{proposition}

\begin{proof}
This is exactly Corollary~\ref{cor:CT-on-translates} proved in Section~\ref{sec:CT-Whittaker}.
\end{proof}

\begin{remark}[Interpretation]
Proposition~\ref{prop:CT-compat-Perf} says that, on the objects that generate the Whittaker category on the
automorphic side, the effect of constant term is \emph{linear} for the spectral action and is governed on the
spectral side by pullback along $f_M^G$.
This is the local analogue of the $\QCoh$-linearity of constant term functors in geometric Langlands.
\end{remark}

%--------------------------------------------------------------------
\subsection{A parabolic compatibility conjecture for the full nilpotent category}\label{subsec:parabolic-compat-full}
%--------------------------------------------------------------------

We now state the desired compatibility in its most useful categorical form.

\begin{conjecture}[Parabolic compatibility of the Whittaker functor]\label{conj:parabolic-compat-full}
Let
\[
  \widetilde{\Phi}_G:\IndCoh_{\mathrm{Nilp}}(\LocSys_{\checkG})\to D_{\mathrm{lis}}(\Bun_G,\Lambda)_{\omega},
  \qquad
  \widetilde{\Phi}_M:\IndCoh_{\mathrm{Nilp}}(\LocSys_{\checkM})\to D_{\mathrm{lis}}(\Bun_M,\Lambda)_{\omega}
\]
be the ind-extended Whittaker functors from Section~\ref{sec:task2}.
Then there are canonical equivalences of functors
\[
  \mathrm{CT}_{P^{-}}\circ \widetilde{\Phi}_G
  \ \simeq\
  \widetilde{\Phi}_M\circ \mathrm{CT}_{P^{-}}^{\mathrm{spec}},
\]
\[
  \mathrm{Eis}_{P}\circ \widetilde{\Phi}_M
  \ \simeq\
  \widetilde{\Phi}_G\circ \mathrm{Eis}_{P}^{\mathrm{spec}}.
\]
\end{conjecture}

\subsection{Reduction of constant term compatibility to generators}\label{subsec:CT-compat-reduction}

Assume Task~2: the functors $\widetilde{\Phi}_G$ and $\widetilde{\Phi}_M$ exist and are defined on
$\IndCoh_{\mathrm{Nilp}}$ by specifying their values on a chosen family of compact generators
(Axiom~\ref{ax:spectral-devissage} and Proposition~\ref{prop:task2-reduction}).

\begin{proposition}[Generator reduction for constant term compatibility]\label{prop:CT-compat-generator-reduction}
Assume:
\begin{enumerate}[label=(\alph*), leftmargin=2em]
  \item Task~2 and Axiom~\ref{ax:spectral-devissage} for both $\LocSys_{\checkG}$ and $\LocSys_{\checkM}$;
  \item the functors $\mathrm{CT}_{P^{-}}^{\mathrm{spec}}$ and $\mathrm{CT}_{P^{-}}$ preserve compact objects
  (Theorem~\ref{thm:parabolic-preserve-nilp} and Theorem~\ref{thm:HHS-finiteness});
  \item the constant term compatibility holds on the chosen compact generators of
  $\Coh_{\mathrm{Nilp}}(\LocSys_{\checkG})$.
\end{enumerate}
Then constant term compatibility holds on all of $\IndCoh_{\mathrm{Nilp}}(\LocSys_{\checkG})$:
\[
  \mathrm{CT}_{P^{-}}\circ \widetilde{\Phi}_G
  \ \simeq\
  \widetilde{\Phi}_M\circ \mathrm{CT}_{P^{-}}^{\mathrm{spec}}.
\]
\end{proposition}

\begin{proof}[Proof sketch]
Both sides are colimit-preserving exact functors between compactly generated categories and preserve compact
objects.  By compact generation, it suffices to check an equivalence on a set of compact generators.
\end{proof}

\subsection{Eisenstein compatibility from constant term compatibility}\label{subsec:Eis-from-CT}

We now explain the standard Gaitsgory-style maneuver: once constant term compatibility is known,
Eisenstein compatibility follows by adjunction, provided we have the right adjoints on both sides.

\begin{proposition}[Eisenstein compatibility from constant term compatibility]\label{prop:Eis-from-CT}
Assume:
\begin{enumerate}[label=(\alph*), leftmargin=2em]
  \item constant term compatibility holds:
  $\mathrm{CT}_{P^{-}}\circ \widetilde{\Phi}_G \simeq \widetilde{\Phi}_M\circ \mathrm{CT}_{P^{-}}^{\mathrm{spec}}$;
  \item $\mathrm{Eis}_P$ is right adjoint to $\mathrm{CT}_{P^{-}}$ on the automorphic side
  (second adjointness, Theorem~\ref{thm:HHS-adjointness});
  \item $\mathrm{Eis}_P^{\mathrm{spec}}$ is right adjoint to $\mathrm{CT}_{P^{-}}^{\mathrm{spec}}$ on the spectral side
  (spectral second adjointness, a consequence of Axiom~\ref{ax:spectral-parabolic-geometry} plus duality).
\end{enumerate}
Then Eisenstein compatibility holds:
\[
  \mathrm{Eis}_{P}\circ \widetilde{\Phi}_M
  \ \simeq\
  \widetilde{\Phi}_G\circ \mathrm{Eis}_{P}^{\mathrm{spec}}.
\]
\end{proposition}

\begin{proof}[Proof sketch]
Let $\mathcal{F}\in \IndCoh_{\mathrm{Nilp}}(\LocSys_{\checkM})$ and $\mathcal{G}\in \IndCoh_{\mathrm{Nilp}}(\LocSys_{\checkG})$.
Using the adjunctions in (b) and (c) and constant term compatibility in (a), we obtain canonical equivalences:
\[
\Hom\big(\mathrm{Eis}_P(\widetilde{\Phi}_M(\mathcal{F})),\widetilde{\Phi}_G(\mathcal{G})\big)
\simeq
\Hom\big(\widetilde{\Phi}_M(\mathcal{F}),\mathrm{CT}_{P^{-}}(\widetilde{\Phi}_G(\mathcal{G}))\big)
\]
\[
\simeq
\Hom\big(\widetilde{\Phi}_M(\mathcal{F}),\widetilde{\Phi}_M(\mathrm{CT}_{P^{-}}^{\mathrm{spec}}(\mathcal{G}))\big)
\simeq
\Hom\big(\mathcal{F},\mathrm{CT}_{P^{-}}^{\mathrm{spec}}(\mathcal{G})\big)
\simeq
\Hom\big(\mathrm{Eis}_P^{\mathrm{spec}}(\mathcal{F}),\mathcal{G}\big).
\]
By Yoneda, this identifies $\mathrm{Eis}_P\circ \widetilde{\Phi}_M$ with
$\widetilde{\Phi}_G\circ \mathrm{Eis}_P^{\mathrm{spec}}$.
\end{proof}

\begin{remark}[What remains to complete Task 3]
The core missing ingredient in Task~3 is to establish spectral second adjointness and, more importantly, to
verify constant term compatibility on a set of spectral compact generators.
The constant term computation of Section~\ref{sec:CT-Whittaker} already verifies compatibility on the
\emph{perfect} generators coming from $\Perf(\LocSys_{\checkG})$.
To reach all of $\Coh_{\mathrm{Nilp}}$, one needs the explicit d\'evissage family of Task~2 and a geometric
definition of $\widetilde{\Phi}$ on those generators.
Once this is in place, Proposition~\ref{prop:CT-compat-generator-reduction} and
Proposition~\ref{prop:Eis-from-CT} make the remainder of Task~3 formal.
\end{remark}










%--------------------------------------------------------------------
\section{A parabolic calculation for the Whittaker generator}\label{sec:CT-Whittaker}
%--------------------------------------------------------------------

\subsection{Setup and statement}

Throughout this section we work with coefficients in a finite extension $\Lambda$ of $\mathbb{Q}_\ell$
for a prime $\ell\neq p$.

Fix a Borel subgroup $B\subset G$ over $E$, with unipotent radical $U\subset B$, and fix a
\emph{non-degenerate} (generic) smooth character
\[
  \psi \colon U(E)\longrightarrow \Lambda^\times.
\]
Write $i_1\colon \Bun_G^{1}\hookrightarrow \Bun_G$ for the basic (neutral) Harder--Narasimhan stratum,
so that $\Bun_G^{1}\simeq [*/G(E)]$.
Let $\cD_{\mathrm{sm}}(G(E),\Lambda)$ denote the (derived) $\infty$-category of smooth $\Lambda$-representations
of $G(E)$ (with the usual model in terms of $\ell$-adic sheaves on $[*/G(E)]$).

\begin{definition}[Whittaker generator on $\Bun_G$]\label{def:Whittaker-generator}
Let
\[
  \mathrm{GG}_{G,\psi}\ :=\ \mathrm{c}\text{-}\mathrm{ind}_{U(E)}^{G(E)}(\psi)
\]
be the (smooth) Gelfand--Graev representation.
We define the \emph{Whittaker generator} on $\Bun_G$ by
\[
  \mathcal{W}_{G,\psi}\ :=\ i_{1,!}\big(\mathrm{GG}_{G,\psi}\big)\ \in\ D_{\mathrm{lis}}(\Bun_G,\Lambda),
\]
where $D_{\mathrm{lis}}(\Bun_G,\Lambda)$ is the derived category of lisse $\Lambda$-adic sheaves on $\Bun_G$
in the sense of Fargues--Scholze.
\end{definition}

Now fix a standard parabolic subgroup $P\subset G$ containing $B$, with Levi factor $M$ and unipotent
radical $N$. Let $P^{-}=MN^{-}$ denote the parabolic opposite to $P$ with respect to $M$.
Set $U_M := U\cap M$ and $\psi_M := \psi|_{U_M(E)}$, which is again non-degenerate.

Let
\[
  \Bun_M \xleftarrow{\,q\,} \Bun_{P^{-}} \xrightarrow{\,p\,} \Bun_G
\]
be the natural correspondence. We use the \emph{normalized} constant term functor
\[
  \mathrm{CT}_{P^{-}} \ :=\ q_*\circ p^! \;:\; D_{\mathrm{lis}}(\Bun_G,\Lambda)\longrightarrow D_{\mathrm{lis}}(\Bun_M,\Lambda),
\]
with the normalizations fixed earlier so that $\mathrm{CT}_{P^{-}}$ is right adjoint to the corresponding
normalized Eisenstein series functor (as in the construction of Hamann--Hansen--Scholze).

\begin{theorem}[Constant term of the Whittaker generator]\label{thm:CT-Whittaker}
There is a canonical isomorphism in $D_{\mathrm{lis}}(\Bun_M,\Lambda)$
\[
  \mathrm{CT}_{P^{-}}\big(\mathcal{W}_{G,\psi}\big)\ \simeq\ \mathcal{W}_{M,\psi_M}.
\]
\end{theorem}

\subsection{Reduction to a representation-theoretic Jacquet module computation}

We first record a general fact: on objects supported on the neutral stratum, the geometric constant term
reduces to the usual normalized Jacquet module functor on smooth representations.

\begin{proposition}[Constant term on the neutral stratum]\label{prop:CT-neutral-stratum}
Let $V\in \cD_{\mathrm{sm}}(G(E),\Lambda)$ and view $i_{1,!}V\in D_{\mathrm{lis}}(\Bun_G,\Lambda)$.
Then $\mathrm{CT}_{P^{-}}(i_{1,!}V)$ is supported on $\Bun_M^{1}\simeq [*/M(E)]$, and there is a canonical
identification
\[
  \mathrm{CT}_{P^{-}}(i_{1,!}V)\ \simeq\ i_{1,!}^M\big(J_{P^{-}}(V)\big),
\]
where $i_1^M\colon \Bun_M^{1}\hookrightarrow \Bun_M$ is the neutral stratum and
$J_{P^{-}}$ denotes the \emph{normalized} Jacquet module functor for the parabolic $P^{-}$.
\end{proposition}

\begin{proof}[Proof sketch]
Because $V$ is supported on $\Bun_G^{1}$, the object $p^!(i_{1,!}V)$ is supported on the
fiber product
\[
  \Bun_{P^{-}}\times_{\Bun_G}\Bun_G^{1}.
\]
On the level of Harder--Narasimhan strata, the preimage of the neutral stratum in $\Bun_G$ is exactly
the neutral stratum in $\Bun_{P^{-}}$, hence
\[
  \Bun_{P^{-}}\times_{\Bun_G}\Bun_G^{1}\ \simeq\ \Bun_{P^{-}}^{1}\ \simeq\ [*/P^{-}(E)].
\]
Under this identification, the map $q\colon \Bun_{P^{-}}^{1}\to \Bun_M^{1}$ is induced by the quotient
homomorphism $P^{-}(E)\twoheadrightarrow M(E)$, and the functor $q_*$ is the usual derived pushforward
along classifying stacks, which identifies with derived (smooth) coinvariants for the subgroup $N^{-}(E)$.
With the chosen normalization of $\mathrm{CT}_{P^{-}}$, the resulting functor on the level of smooth
representations is precisely the normalized Jacquet module $J_{P^{-}}$.
\end{proof}

\subsection{Jacquet module of the Gelfand--Graev representation}

The remaining input is a classical representation-theoretic statement: the Jacquet module of the
Gelfand--Graev representation along a parabolic is again a Gelfand--Graev representation for the Levi.

\begin{theorem}[Bushnell--Henniart; see also Matringe]\label{thm:BH-GG}
There is a canonical isomorphism of smooth $M(E)$-representations
\[
  J_{P^{-}}\!\left(\mathrm{c}\text{-}\mathrm{ind}_{U(E)}^{G(E)}(\psi)\right)\ \simeq\
  \mathrm{c}\text{-}\mathrm{ind}_{U_M(E)}^{M(E)}(\psi_M).
\]
\end{theorem}

\begin{proof}[Reference]
This is proved in \cite[Theorem 2.2]{BushnellHenniartGW} in the language of compactly supported Whittaker
functions; an explicit formulation (including normalizations) is recalled and reproved in
\cite[Theorem 3.10]{MatringeGWJ}.
\end{proof}

\subsection{Proof of Theorem \ref{thm:CT-Whittaker}}

\begin{proof}
Apply Proposition~\ref{prop:CT-neutral-stratum} to $V=\mathrm{GG}_{G,\psi}$ and use
Theorem~\ref{thm:BH-GG}. By Definition~\ref{def:Whittaker-generator}, we obtain
\[
  \mathrm{CT}_{P^{-}}(\mathcal{W}_{G,\psi})
  \;\simeq\;
  i_{1,!}^M\!\left(J_{P^{-}}(\mathrm{GG}_{G,\psi})\right)
  \;\simeq\;
  i_{1,!}^M\!\left(\mathrm{GG}_{M,\psi_M}\right)
  \;=\;
  \mathcal{W}_{M,\psi_M},
\]
as claimed.
\end{proof}

\subsection{Consequences for the Whittaker subcategory and spectral functoriality}

Let $D_{\mathrm{lis}}(\Bun_G,\Lambda)_{\omega}$ denote the Whittaker-generated subcategory of
$D_{\mathrm{lis}}(\Bun_G,\Lambda)$, that is, the smallest full stable subcategory closed under colimits
and containing $\mathcal{W}_{G,\psi}$ and stable under the spectral action of
$\Perf(\LocSys_{\check{G}})$.

\begin{corollary}[Constant term preserves the Whittaker-generated subcategory]\label{cor:CT-preserves-omega}
The functor $\mathrm{CT}_{P^{-}}$ carries $D_{\mathrm{lis}}(\Bun_G,\Lambda)_{\omega}$ into
$D_{\mathrm{lis}}(\Bun_M,\Lambda)_{\omega}$ and sends the generator $\mathcal{W}_{G,\psi}$ to the generator
$\mathcal{W}_{M,\psi_M}$.
\end{corollary}

\begin{proof}[Proof sketch]
Theorem~\ref{thm:CT-Whittaker} gives the statement on generators.
Compatibility of constant term functors with Hecke operators (hence with the spectral action) implies
stability under the action, and closure under colimits finishes the argument.
\end{proof}

Let $f\colon \LocSys_{\check{M}}\to \LocSys_{\check{G}}$ be the natural map induced by $\check{M}\hookrightarrow \check{G}$.
Using the standard description of pullback on quasi-coherent sheaves as relative tensor product,
\[
  f^*(\mathcal{F})\ \simeq\ \mathcal{O}_{\LocSys_{\check{M}}}\otimes_{\mathcal{O}_{\LocSys_{\check{G}}}}\mathcal{F},
\]
we obtain the following concrete formula on the Whittaker-generated objects.

\begin{corollary}[Formula on Whittaker translates]\label{cor:CT-on-translates}
For any $\mathcal{F}\in \Perf(\LocSys_{\check{G}})$, there is a canonical isomorphism
\[
  \mathrm{CT}_{P^{-}}\big(\mathcal{F}\star \mathcal{W}_{G,\psi}\big)
  \;\simeq\;
  f^*(\mathcal{F}) \star \mathcal{W}_{M,\psi_M}.
\]
\end{corollary}

\begin{proof}[Proof sketch]
By the definition of the spectral action, the operation $\mathcal{F}\mapsto \mathcal{F}\star(-)$ is
$\Perf(\LocSys_{\check{G}})$-linear.
Compatibility of constant term functors with the Hecke category identifies the induced action on the Levi
side with pullback along $f$.
It therefore suffices to check the claim for $\mathcal{F}=\mathcal{O}_{\LocSys_{\check{G}}}$, which is exactly
Theorem~\ref{thm:CT-Whittaker}.
\end{proof}

\begin{remark}
Corollary~\ref{cor:CT-on-translates} is one of the points where the analogy with the characteristic zero
geometric Langlands proof strategy becomes visible: on the Whittaker-generated part, parabolic functors
are forced by linearity and the identification of the Whittaker generator, and the nontrivial content is
reduced to a concrete computation on a single distinguished object.
\end{remark}


%--------------------------------------------------------------------
\section{Endomorphisms of the Whittaker generator and the Bernstein center}\label{sec:endo-Whittaker}
%--------------------------------------------------------------------

The goal of this section is to make concrete progress on \textbf{Task 4} from
Section~\ref{subsec:summary-dependencies}: compute $\End(W_{\psi})$ and explain how this computation feeds
directly into the monadic (Barr--Beck) approach to full faithfulness.

The key point is that $W_{\psi}$ is supported on the neutral stratum $\Bun_G^{1}\simeq B\underline{G(E)}$.
Hence its endomorphism ring is the same as the endomorphism ring of the
universal Whittaker (Gelfand--Graev) representation of the locally profinite group $G(E)$.
This endomorphism ring is classical: it is controlled by the Bernstein center, by work of
Bushnell and Henniart, with integral refinements in the case $G=\GL_n$ due to Helm.

\subsection{Reduction to the neutral stratum}

Recall that $i_1:\Bun_G^{1}\hookrightarrow \Bun_G$ denotes the locally closed immersion of the neutral
Harder--Narasimhan stratum, and that $\Bun_G^{1}\simeq B\underline{G(E)}$
(Proposition~\ref{prop:neutral-stratum}).

\begin{lemma}[Extension by zero is fully faithful on endomorphisms]\label{lem:i1-fully-faithful-endo}
Let $\mathcal{F}\in D_{\mathrm{lis}}(\Bun_G^{1},\Lambda)$ and set $i_{1,!}\mathcal{F}\in D_{\mathrm{lis}}(\Bun_G,\Lambda)$.
Then the canonical map
\[
  \End_{D_{\mathrm{lis}}(\Bun_G^{1},\Lambda)}(\mathcal{F})
  \ \longrightarrow\
  \End_{D_{\mathrm{lis}}(\Bun_G,\Lambda)}(i_{1,!}\mathcal{F})
\]
is an isomorphism.
\end{lemma}

\begin{proof}[Proof sketch]
A locally closed immersion factors as an open immersion followed by a closed immersion.
For open immersions, extension by zero is fully faithful; for closed immersions, pushforward is fully faithful.
Composing yields full faithfulness of $i_{1,!}$ on mapping complexes and hence on endomorphisms.
\end{proof}

\begin{proposition}[Endomorphisms of $W_{\psi}$ as endomorphisms of the Gelfand--Graev representation]
\label{prop:EndWpsi-rep}
Let
\[
  \mathrm{GG}_{G,\psi}\ :=\ \mathrm{c}\text{-}\mathrm{ind}_{U(E)}^{G(E)}(\psi)
\]
and let $W_{\psi}=i_{1,!}(\mathrm{GG}_{G,\psi})$ be the Whittaker generator as in
Definition~\ref{def:whittaker-sheaf} (equivalently Definition~\ref{def:Whittaker-generator}).
Then there is a canonical identification
\[
  \End_{D_{\mathrm{lis}}(\Bun_G,\Lambda)}(W_{\psi})
  \ \simeq\
  \End_{G(E)}(\mathrm{GG}_{G,\psi}).
\]
\end{proposition}

\begin{proof}
Combine Lemma~\ref{lem:i1-fully-faithful-endo} with the equivalence
$D_{\mathrm{lis}}(\Bun_G^{1},\Lambda)\simeq D(\Rep_\Lambda^\infty(G(E)))$
(Proposition~\ref{prop:neutral-stratum}).
\end{proof}

\subsection{Bernstein center and the generic Bernstein summand}

Let $\mathfrak{Z}(G(E))$ denote the Bernstein center of the category of smooth $\Lambda$-representations
of $G(E)$ (for $\Lambda$ a field of characteristic zero, so that the classical Bernstein--Deligne theory applies).

Write
\[
  \Rep^\infty_\Lambda(G(E))\ \simeq\ \prod_{\mathfrak{s}}\ \Rep^\infty_\Lambda(G(E))_{\mathfrak{s}}
\]
for the Bernstein decomposition indexed by inertial equivalence classes $\mathfrak{s}$.
Let $e_{\mathfrak{s}}\in \mathfrak{Z}(G(E))$ denote the corresponding central idempotent.

\begin{definition}[Generic Bernstein summand]\label{def:generic-Bernstein}
We say that $\mathfrak{s}$ is \emph{generic} if the block $\Rep^\infty_\Lambda(G(E))_{\mathfrak{s}}$
contains at least one irreducible generic representation, equivalently if
\[
  e_{\mathfrak{s}}\cdot \mathrm{GG}_{G,\psi}\ \neq\ 0.
\]
Let $\mathfrak{Z}_{\mathrm{gen}}(G(E))\subset \mathfrak{Z}(G(E))$ be the direct product of the centers of the
generic blocks:
\[
  \mathfrak{Z}_{\mathrm{gen}}(G(E))\ :=\ \prod_{\mathfrak{s}\ \mathrm{generic}}\ \mathfrak{Z}(G(E))_{\mathfrak{s}}
  \qquad\text{where}\qquad
  \mathfrak{Z}(G(E))_{\mathfrak{s}}:=e_{\mathfrak{s}}\mathfrak{Z}(G(E)).
\]
\end{definition}

\begin{remark}
The relevance of $\mathfrak{Z}_{\mathrm{gen}}(G(E))$ is that the universal Whittaker representation
$\mathrm{GG}_{G,\psi}$ is supported precisely on the generic blocks.  In particular, its endomorphism ring is
expected to see exactly $\mathfrak{Z}_{\mathrm{gen}}(G(E))$, not necessarily the full Bernstein center.
\end{remark}

\subsection{Bushnell--Henniart and Helm: endomorphisms of the universal Whittaker representation}

The following theorem packages the representation-theoretic input we will use.

\begin{theorem}[Bushnell--Henniart; Helm]\label{thm:BH-Helm-EndGG}
Assume $G$ is quasi-split over $E$ and $\Lambda$ is a field of characteristic zero.
\begin{enumerate}[label=(\alph*), leftmargin=2em]
  \item For each Bernstein idempotent $e_{\mathfrak{s}}$, the block center
  $\mathfrak{Z}(G(E))_{\mathfrak{s}}$ acts on $e_{\mathfrak{s}}\mathrm{GG}_{G,\psi}$, and the induced map
  \[
    \mathfrak{Z}(G(E))_{\mathfrak{s}}\ \longrightarrow\ \End_{G(E)}(e_{\mathfrak{s}}\mathrm{GG}_{G,\psi})
  \]
  is injective (faithful action of the block center on Whittaker models).
  \item In many cases (in particular for $G=\GL_n$), the map in (a) is an isomorphism for every generic
  $\mathfrak{s}$.
  \item For $G=\GL_n$, Helm proves an integral refinement: after fixing integral coefficients
  (for example $\Lambda=W(k)$ with $k$ algebraically closed of characteristic $\ell\neq p$),
  the corresponding block center is the full endomorphism ring of the universal co-Whittaker
  object in that block.  In particular, after inverting $\ell$, one recovers the isomorphism in (b).
\end{enumerate}
\end{theorem}

\begin{proof}[References]
Part (a) and the characteristic zero form of (b) are contained in
\cite{BushnellHenniartGW} (see the discussion in the abstract and the results described there).
For $G=\GL_n$, the integral statement (c) and the explicit identification of the block center with the
endomorphism ring are proved in \cite[Theorem 5.2]{HelmWhittakerBernstein}.
\end{proof}

Combining Theorem~\ref{thm:BH-Helm-EndGG} with Proposition~\ref{prop:EndWpsi-rep} yields the geometric form:

\begin{corollary}[Endomorphisms of $W_{\psi}$]\label{cor:EndWpsi-Bernstein}
Assume $\Lambda$ is a field of characteristic zero.
Then there is a canonical injection of commutative algebras
\[
  \mathfrak{Z}_{\mathrm{gen}}(G(E))\ \hookrightarrow\ \End_{D_{\mathrm{lis}}(\Bun_G,\Lambda)}(W_{\psi}),
\]
and for $G=\GL_n$ this map is an isomorphism.
\end{corollary}

\begin{remark}
Corollary~\ref{cor:EndWpsi-Bernstein} makes \textbf{Task 4} precise:
computing $\End(W_{\psi})$ is equivalent to understanding how the Bernstein center acts on the universal
Whittaker representation, which is classical and explicit in important cases.
\end{remark}

\subsection{The stable center map on $W_{\psi}$}

By the spectral action of Theorem~\ref{thm:spectral-action} (Section~\ref{sec:spectral-action}),
the commutative algebra of global functions on the spectral stack acts on every object of the automorphic
category.  In particular, it acts on $W_{\psi}$.

\begin{definition}[Stable center map on the Whittaker generator]\label{def:stable-center-on-Wpsi}
Let
\[
  \mathfrak{Z}^{\mathrm{st}}_{\checkG}
  \ :=\
  \Gamma(\LocSys_{\checkG},\OO_{\LocSys_{\checkG}}).
\]
The spectral action gives a canonical algebra homomorphism
\[
  \zeta_{\psi}\colon \mathfrak{Z}^{\mathrm{st}}_{\checkG}\longrightarrow
  \End_{D_{\mathrm{lis}}(\Bun_G,\Lambda)}(W_{\psi}).
\]
\end{definition}

\begin{proposition}[Factorization through the generic Bernstein center]\label{prop:stable-center-factor}
Assume $G$ is quasi-split and $\Lambda$ is a field of characteristic zero.
Then $\zeta_{\psi}$ factors canonically through the generic Bernstein center:
\[
  \mathfrak{Z}^{\mathrm{st}}_{\checkG}\ \longrightarrow\ \mathfrak{Z}_{\mathrm{gen}}(G(E))
  \ \hookrightarrow\ \End_{D_{\mathrm{lis}}(\Bun_G,\Lambda)}(W_{\psi}).
\]
For $G=\GL_n$ the last map is an isomorphism by Corollary~\ref{cor:EndWpsi-Bernstein}.
\end{proposition}

\begin{proof}[Proof sketch]
By Proposition~\ref{prop:EndWpsi-rep} the endomorphism ring of $W_{\psi}$ is the endomorphism ring of the
universal Whittaker representation.  Any central action on the automorphic category restricts to a central
action on $\Rep^\infty_\Lambda(G(E))$ on the neutral stratum, hence factors through the Bernstein center.
Since the universal Whittaker representation is supported only on the generic blocks,
the action factors through $\mathfrak{Z}_{\mathrm{gen}}(G(E))$.
\end{proof}

\subsection{Compatibility with parabolic restriction}

We now record a compatibility statement that is already accessible using the constant term computation
from Section~\ref{sec:CT-Whittaker}.  It will be used later as the ``gluing interface'' between centers
for $G$ and for Levi subgroups.

Let $P\subset G$ be a parabolic containing $B$, with Levi quotient $M$, and let $P^{-}$ be the opposite parabolic.
Let
\[
  f_{M}^{G}\colon \LocSys_{\checkG}\longrightarrow \LocSys_{\checkM}
\]
be the morphism induced by $\checkM\hookrightarrow \checkG$.

\begin{proposition}[Parabolic compatibility on endomorphisms]\label{prop:center-parabolic-compat}
The constant term functor $\mathrm{CT}_{P^{-}}$ induces an algebra homomorphism
\[
  \mathrm{CT}_{P^{-}}^{\sharp}\colon \End(W_{\psi})\longrightarrow \End(W_{\psi_M})
\]
and, under the maps $\zeta_{\psi}$ and $\zeta_{\psi_M}$ of Definition~\ref{def:stable-center-on-Wpsi}, the diagram
\[
\begin{tikzcd}
\Gamma(\LocSys_{\checkG},\OO) \arrow[r,"\zeta_{\psi}"] \arrow[d,"(f_M^{G})^{\ast}"'] &
\End(W_{\psi}) \arrow[d,"\mathrm{CT}_{P^{-}}^{\sharp}"] \\
\Gamma(\LocSys_{\checkM},\OO) \arrow[r,"\zeta_{\psi_M}"] &
\End(W_{\psi_M})
\end{tikzcd}
\]
commutes.
\end{proposition}

\begin{proof}[Proof sketch]
Apply $\mathrm{CT}_{P^{-}}$ to an endomorphism of $W_{\psi}$ and use Theorem~\ref{thm:CT-Whittaker} to identify
$\mathrm{CT}_{P^{-}}(W_{\psi})\simeq W_{\psi_M}$.
Commutativity of the diagram is a reformulation of Corollary~\ref{cor:CT-on-translates} in degree zero:
constant term is linear with respect to the spectral action, and on the spectral side this linearity is
precisely pullback along $f_M^G$.
\end{proof}

\subsection{A clean reduction for Task 4}

We end with a formulation that makes the remaining content of \textbf{Task 4} entirely explicit.

\begin{conjecture}[Stable center identification on the Whittaker generator]\label{conj:stable-center-iso}
The map $\zeta_{\psi}$ of Definition~\ref{def:stable-center-on-Wpsi} is an isomorphism:
\[
  \Gamma(\LocSys_{\checkG},\OO)\ \xrightarrow{\ \sim\ }\ \End(W_{\psi}).
\]
Equivalently, the stable Bernstein center coincides with the endomorphism algebra of the Whittaker generator.
\end{conjecture}

\begin{remark}[Translation to classical representation theory]
By Corollary~\ref{cor:EndWpsi-Bernstein}, Conjecture~\ref{conj:stable-center-iso} is equivalent to the statement
that the canonical map
\[
  \Gamma(\LocSys_{\checkG},\OO)\ \longrightarrow\ \mathfrak{Z}_{\mathrm{gen}}(G(E))
\]
is an isomorphism.
For $G=\GL_n$ this is the expected comparison between the stable center and the Bernstein center in the
absence of endoscopy, and it is compatible with Helm's interpretation of integral centers in terms of
Galois deformation theory \cite{HelmWhittakerBernstein}.
\end{remark}


%--------------------------------------------------------------------
\section{Monadicity and full faithfulness}\label{sec:task4}
%--------------------------------------------------------------------

In this section we address \textbf{Task 4} from Section~\ref{subsec:summary-dependencies}:
\begin{itemize}[leftmargin=2em]
  \item construct and prove \emph{conservativity} of the right adjoint to the Whittaker functor, and
  \item reduce \emph{full faithfulness} to an explicit computation of the endomorphism algebra of the Whittaker
  generator, together with a verification on a set of compact generators on the spectral side.
\end{itemize}
The point is that the Whittaker generator is expected to play the role of a ``vacuum'' object in a
Gaitsgory-style proof: the entire comparison is forced by how the spectral monoidal category acts on this one
object, and full faithfulness becomes a monadic statement.

Throughout, we work with coefficients in a finite extension $\Lambda$ of $\mathbb{Q}_\ell$ with $\ell\neq p$ and
assume $G$ is quasi-split with fixed Whittaker data $(B,\psi)$.

%--------------------------------------------------------------------
\subsection{The Whittaker functor and its right adjoint}\label{subsec:task4-adjoints}
%--------------------------------------------------------------------

Assume Tasks~1--2 have been completed, so that the Whittaker assignment extends to a colimit-preserving functor
\[
  \widetilde{\Phi}_G\;:\;\IndCoh_{\mathrm{Nilp}}(\LocSys_{\checkG})\longrightarrow D_{\mathrm{lis}}(\Bun_G,\Lambda)_{\omega}
\]
whose restriction to perfect complexes with quasi-compact support agrees with
$F\mapsto F\ast \mathcal{W}_{G,\psi}$ (Section~\ref{sec:whittaker} and Section~\ref{sec:task2}).

\begin{proposition}[Existence of a continuous right adjoint]\label{prop:task4-right-adjoint-exists}
The functor $\widetilde{\Phi}_G$ admits a continuous (colimit-preserving on compactly generated subcategories)
right adjoint
\[
  \widetilde{\Phi}_G^{R}\;:\;D_{\mathrm{lis}}(\Bun_G,\Lambda)_{\omega}\longrightarrow
  \IndCoh_{\mathrm{Nilp}}(\LocSys_{\checkG}).
\]
\end{proposition}

\begin{proof}[Proof sketch]
Both categories are presentable stable $\infty$-categories, and $\widetilde{\Phi}_G$ preserves colimits by
construction (Task~2). The adjoint functor theorem for presentable categories gives the existence of
$\widetilde{\Phi}_G^{R}$.
\end{proof}

The adjunction supplies, for every
$\mathcal{F}\in \IndCoh_{\mathrm{Nilp}}(\LocSys_{\checkG})$ and
$\mathcal{A}\in D_{\mathrm{lis}}(\Bun_G,\Lambda)_{\omega}$, canonical equivalences
\[
  \mathrm{Map}\big(\widetilde{\Phi}_G(\mathcal{F}),\mathcal{A}\big)
  \ \simeq\
  \mathrm{Map}\big(\mathcal{F},\widetilde{\Phi}_G^{R}(\mathcal{A})\big).
\]
In particular, evaluating at $\mathcal{F}=\OO_{\LocSys_{\checkG}}$ gives a basic formula relating the
right adjoint to maps out of the Whittaker generator:
\begin{equation}\label{eq:task4-evaluate-at-unit}
  \mathrm{Map}\big(\mathcal{W}_{G,\psi},\mathcal{A}\big)
  \ \simeq\
  \mathrm{Map}\big(\OO_{\LocSys_{\checkG}},\widetilde{\Phi}_G^{R}(\mathcal{A})\big).
\end{equation}

%--------------------------------------------------------------------
\subsection{Conservativity of the right adjoint}\label{subsec:task4-conservative}
%--------------------------------------------------------------------

The first part of Task~4 is to show that the right adjoint is conservative.  This is a formal consequence of
the fact that $\mathcal{W}_{G,\psi}$ generates the Whittaker category by construction.

\begin{lemma}[Generators detect vanishing]\label{lem:generator-detects}
Let $\cC$ be a presentable stable $\infty$-category and let $G\in \cC$.
Assume $\cC$ is the smallest full stable subcategory closed under colimits that contains $G$.
Then the functor
\[
  \mathrm{Map}(G,-)\;:\;\cC\longrightarrow \mathrm{Sp}
\]
is conservative: if $\mathrm{Map}(G,X)\simeq 0$ then $X\simeq 0$.
\end{lemma}

\begin{proof}
Let $\cC_{X}:=\{Y\in \cC\mid \mathrm{Map}(Y,X)\simeq 0\}$.
Because $\cC$ is stable and presentable, $\cC_X$ is a full stable subcategory closed under colimits.
If $\mathrm{Map}(G,X)\simeq 0$ then $G\in \cC_X$, hence $\cC_X=\cC$ by the generation hypothesis.
In particular, $X\in \cC_X$, so $\mathrm{Map}(X,X)\simeq 0$, which forces $X\simeq 0$.
\end{proof}

\begin{proposition}[Conservativity of $\widetilde{\Phi}_G^{R}$]\label{prop:task4-right-adjoint-conservative}
The right adjoint $\widetilde{\Phi}_G^{R}$ is conservative on $D_{\mathrm{lis}}(\Bun_G,\Lambda)_{\omega}$.
\end{proposition}

\begin{proof}
Suppose $\widetilde{\Phi}_G^{R}(\mathcal{A})\simeq 0$.
Then the right-hand side of \eqref{eq:task4-evaluate-at-unit} vanishes, hence
$\mathrm{Map}(\mathcal{W}_{G,\psi},\mathcal{A})\simeq 0$.
Since $D_{\mathrm{lis}}(\Bun_G,\Lambda)_{\omega}$ is, by definition, generated under colimits by
$\mathcal{W}_{G,\psi}$ and stability (Definition~\ref{def:omega-subcategory}), Lemma~\ref{lem:generator-detects}
implies $\mathcal{A}\simeq 0$.
\end{proof}

\begin{remark}
Proposition~\ref{prop:task4-right-adjoint-conservative} is the clean ``right adjoint is conservative'' input
needed for Barr--Beck type arguments.  The proof uses no geometry beyond the definition of the Whittaker
subcategory.
\end{remark}

%--------------------------------------------------------------------
\subsection{Endomorphisms of the Whittaker generator}\label{subsec:task4-End}
%--------------------------------------------------------------------

We now turn to the second part of Task~4: compute $\End(\mathcal{W}_{G,\psi})$ and relate it to the spectral
center.  We restate the key outcome from Section~\ref{sec:endo-Whittaker} in a form tailored for monadicity.

\begin{proposition}[Endomorphisms reduce to the neutral stratum]\label{prop:task4-End-reduction}
There is a canonical identification
\[
  \End_{D_{\mathrm{lis}}(\Bun_G,\Lambda)}(\mathcal{W}_{G,\psi})
  \ \simeq\
  \End_{G(E)}\!\left(\mathrm{c}\text{-}\mathrm{ind}_{U(E)}^{G(E)}(\psi)\right).
\]
\end{proposition}

\begin{proof}
This is Proposition~\ref{prop:EndWpsi-rep}.
\end{proof}

Let $\Gamma(\LocSys_{\checkG},\OO)$ denote the algebra of global functions on the parameter stack.
The spectral action (Theorem~\ref{thm:spectral-action}) yields an algebra map
\[
  \zeta_{\psi}\;:\;\Gamma(\LocSys_{\checkG},\OO)\longrightarrow \End(\mathcal{W}_{G,\psi}).
\]

\begin{theorem}[Generic Bernstein center description]\label{thm:task4-generic-Bernstein}
Assume $\Lambda$ has characteristic zero.
There exists a canonical factorization
\[
  \Gamma(\LocSys_{\checkG},\OO)\ \longrightarrow\ \mathfrak{Z}_{\mathrm{gen}}(G(E))\ \longrightarrow\
  \End_{D_{\mathrm{lis}}(\Bun_G,\Lambda)}(\mathcal{W}_{G,\psi}),
\]
where $\mathfrak{Z}_{\mathrm{gen}}(G(E))$ denotes the product of the Bernstein centers of the generic Bernstein
blocks.  For $G=\GL_n$ the second arrow is an isomorphism.
\end{theorem}

\begin{proof}[Proof sketch]
This is Proposition~\ref{prop:stable-center-factor} together with Corollary~\ref{cor:EndWpsi-Bernstein}, which in
turn use the representation-theoretic input of Bushnell--Henniart and Helm
(Theorem~\ref{thm:BH-Helm-EndGG}).
\end{proof}

\begin{conjecture}[Stable center identification on the Whittaker generator]\label{conj:task4-stable-center-iso}
The map $\zeta_{\psi}$ is an isomorphism:
\[
  \Gamma(\LocSys_{\checkG},\OO)\ \xrightarrow{\ \sim\ }\ \End(\mathcal{W}_{G,\psi}).
\]
\end{conjecture}

\begin{remark}
Conjecture~\ref{conj:task4-stable-center-iso} is the \emph{precise} endomorphism computation needed for the
monadic step below.  It is the local categorical incarnation of the identification of the stable Bernstein
center with functions on the parameter stack.
\end{remark}

%--------------------------------------------------------------------
\subsection{A full faithfulness criterion on perfect complexes}\label{subsec:task4-Perf-ff}
%--------------------------------------------------------------------

Before stating the monadic argument on the full nilpotent category, it is useful to record a concrete
``rank-one'' full faithfulness statement that already follows formally from
Conjecture~\ref{conj:task4-stable-center-iso}.

Let $\Phi_{\Perf}$ be the functor on perfect complexes with quasi-compact support
\[
  \Phi_{\Perf}\;:\;\Perf(\LocSys_{\checkG})_{\mathrm{q.c.}}\longrightarrow
  D_{\mathrm{lis}}(\Bun_G,\Lambda)_{\omega}^{\mathrm{c}},
  \qquad
  F\longmapsto F\ast \mathcal{W}_{G,\psi}.
\]

\begin{proposition}[Full faithfulness on perfect complexes from the center]\label{prop:task4-Perf-ff}
Assume Conjecture~\ref{conj:task4-stable-center-iso}.
Then $\Phi_{\Perf}$ is fully faithful.
\end{proposition}

\begin{proof}
Let $F,G\in \Perf(\LocSys_{\checkG})_{\mathrm{q.c.}}$.
Because $\Perf(\LocSys_{\checkG})$ is rigid, the dual $F^\vee$ exists and the spectral action satisfies
\[
  \mathrm{Map}\big(F\ast \mathcal{W}_{G,\psi},\,G\ast \mathcal{W}_{G,\psi}\big)
  \ \simeq\
  \mathrm{Map}\big(\mathcal{W}_{G,\psi},\,(F^\vee\otimes G)\ast \mathcal{W}_{G,\psi}\big).
\]
Using the adjunction \eqref{eq:task4-evaluate-at-unit} for $\widetilde{\Phi}_G$ restricted to perfect objects,
the right-hand side identifies with
\[
  \mathrm{Map}\big(\OO_{\LocSys_{\checkG}},\,F^\vee\otimes G\big)
\]
\emph{provided} the action of $\Gamma(\LocSys_{\checkG},\OO)$ on $\mathcal{W}_{G,\psi}$ coincides with the
tautological action on the unit, i.e.\ provided $\End(\mathcal{W}_{G,\psi})\simeq \Gamma(\LocSys_{\checkG},\OO)$.
This is exactly Conjecture~\ref{conj:task4-stable-center-iso}.
Finally, because $F$ is dualizable,
\[
  \mathrm{Map}\big(\OO_{\LocSys_{\checkG}},F^\vee\otimes G\big)\ \simeq\ \mathrm{Map}(F,G),
\]
which proves full faithfulness.
\end{proof}

\begin{remark}
Proposition~\ref{prop:task4-Perf-ff} is a genuine ``monoidal'' step: it shows that, on dualizable (perfect)
objects, full faithfulness is controlled by a single endomorphism algebra computation.
Extending beyond perfect complexes requires the d\'evissage and monadic arguments below.
\end{remark}

%--------------------------------------------------------------------
\subsection{Barr--Beck reduction for full faithfulness on $\IndCoh_{\mathrm{Nilp}}$}\label{subsec:task4-BB}
%--------------------------------------------------------------------

We now formulate the monadic reduction that is the real content of Task~4.

\begin{lemma}[Full faithfulness and the unit map]\label{lem:task4-unit}
Let $L:\cC\to \cD$ be a functor with right adjoint $R$.
Then $L$ is fully faithful if and only if the unit $\eta:\mathrm{id}_{\cC}\to RL$ is an isomorphism.
Moreover, if $\cC$ is compactly generated and $RL$ preserves colimits, it suffices to check that $\eta$ is an
isomorphism on a set of compact generators of $\cC$.
\end{lemma}

\begin{proof}
The first statement is standard for adjunctions.
For the second, the full subcategory of objects on which $\eta$ is an isomorphism is stable under colimits and
cones; if it contains a set of compact generators, it must be all of $\cC$.
\end{proof}

Apply Lemma~\ref{lem:task4-unit} to $L=\widetilde{\Phi}_G$ and $R=\widetilde{\Phi}_G^{R}$.
Set
\[
  \mathbb{T}_G\ :=\ \widetilde{\Phi}_G^{R}\circ \widetilde{\Phi}_G
\]
for the resulting monad on $\IndCoh_{\mathrm{Nilp}}(\LocSys_{\checkG})$.

\begin{proposition}[Monadic reduction for full faithfulness]\label{prop:task4-monadic-reduction}
Assume:
\begin{enumerate}[label=(\alph*), leftmargin=2em]
  \item $\widetilde{\Phi}_G$ exists and preserves colimits (Task~2),
  \item $\widetilde{\Phi}_G^{R}$ is conservative (Proposition~\ref{prop:task4-right-adjoint-conservative}),
  \item $\IndCoh_{\mathrm{Nilp}}(\LocSys_{\checkG})$ is compactly generated by a set of compact objects
  $\mathcal{G}\subset \Coh_{\mathrm{Nilp}}(\LocSys_{\checkG})$, and
  \item for every $\mathcal{E}\in \mathcal{G}$, the unit map $\mathcal{E}\to \mathbb{T}_G(\mathcal{E})$ is an
  isomorphism.
\end{enumerate}
Then $\widetilde{\Phi}_G$ is fully faithful.  In particular, its restriction
\[
  \Phi_G^{\mathrm{c}}:\Coh_{\mathrm{Nilp}}(\LocSys_{\checkG})\to D_{\mathrm{lis}}(\Bun_G,\Lambda)_{\omega}^{\mathrm{c}}
\]
is fully faithful.
\end{proposition}

\begin{proof}
By Lemma~\ref{lem:task4-unit}, full faithfulness is equivalent to $\eta:\mathrm{id}\to \mathbb{T}_G$ being an
isomorphism.  The hypotheses guarantee that it is an isomorphism on compact generators, hence on all objects.
\end{proof}

\begin{remark}[How the endomorphism computation enters]
Evaluating $\mathbb{T}_G$ on the unit object gives
\[
  \mathbb{T}_G(\OO_{\LocSys_{\checkG}})\ \simeq\ \widetilde{\Phi}_G^{R}(\mathcal{W}_{G,\psi}),
\]
and the algebra of endomorphisms of $\mathcal{W}_{G,\psi}$ controls this object via
\eqref{eq:task4-evaluate-at-unit}.
In particular, Conjecture~\ref{conj:task4-stable-center-iso} is the statement that the induced map on
$\pi_0$-endomorphisms agrees with $\Gamma(\LocSys_{\checkG},\OO)$.
To upgrade this to condition (d) in Proposition~\ref{prop:task4-monadic-reduction}, one must check the unit
isomorphism on a generating set $\mathcal{G}$.
Axiom~\ref{ax:spectral-devissage} from Task~2 provides a candidate generating set, and Task~3 supplies the
parabolic functoriality needed to propagate the unit calculation from the unit object to these generators.
\end{remark}

%--------------------------------------------------------------------
\subsection{Parabolic functoriality and propagation of the unit calculation}\label{subsec:task4-propagate}
%--------------------------------------------------------------------

We record one concrete propagation mechanism already available from Section~\ref{sec:CT-Whittaker}:
compatibility with constant term functors.

Let $P\subset G$ be a parabolic with Levi quotient $M$ and opposite parabolic $P^{-}$.
Assume Task~3, so that spectral parabolic functors exist and constant term compatibility holds:
\[
  \mathrm{CT}_{P^{-}}\circ \widetilde{\Phi}_G\ \simeq\ \widetilde{\Phi}_M\circ \mathrm{CT}_{P^{-}}^{\mathrm{spec}}.
\]

\begin{proposition}[Unit isomorphisms descend under constant term]\label{prop:task4-unit-CT}
Assume:
\begin{enumerate}[label=(\alph*), leftmargin=2em]
  \item $\widetilde{\Phi}_G$ and $\widetilde{\Phi}_M$ exist with right adjoints,
  \item constant term compatibility holds as above, and
  \item the unit map $\mathcal{E}\to \mathbb{T}_G(\mathcal{E})$ is an isomorphism for some
  $\mathcal{E}\in \IndCoh_{\mathrm{Nilp}}(\LocSys_{\checkG})$.
\end{enumerate}
Then the unit map
\[
  \mathrm{CT}_{P^{-}}^{\mathrm{spec}}(\mathcal{E})\ \longrightarrow\
  \mathbb{T}_M\big(\mathrm{CT}_{P^{-}}^{\mathrm{spec}}(\mathcal{E})\big)
\]
is an isomorphism in $\IndCoh_{\mathrm{Nilp}}(\LocSys_{\checkM})$.
\end{proposition}

\begin{proof}[Proof sketch]
The compatibility identifies $\mathrm{CT}_{P^{-}}$ with a functor commuting with the Whittaker functors.
Passing to right adjoints yields a compatible identification of the monads $\mathbb{T}_G$ and $\mathbb{T}_M$
after applying $\mathrm{CT}_{P^{-}}^{\mathrm{spec}}$.
Thus the unit isomorphism for $\mathcal{E}$ implies the unit isomorphism after constant term.
\end{proof}

\begin{remark}
Proposition~\ref{prop:task4-unit-CT} is the local analogue of the way parabolic functoriality is used in the
proof of geometric Langlands to reduce full faithfulness from $G$ to Levi subgroups.
In the present setting it provides an induction mechanism for verifying condition (d) in
Proposition~\ref{prop:task4-monadic-reduction}.
\end{remark}

%--------------------------------------------------------------------
\subsection{Summary of Task 4}\label{subsec:task4-summary}
%--------------------------------------------------------------------

We summarize the concrete output of this section as follows.

\begin{itemize}[leftmargin=2em]
  \item The right adjoint $\widetilde{\Phi}_G^{R}$ exists and is conservative
  (Propositions~\ref{prop:task4-right-adjoint-exists} and \ref{prop:task4-right-adjoint-conservative}).
  \item Full faithfulness of $\widetilde{\Phi}_G$ reduces to a unit-isomorphism statement for the monad
  $\mathbb{T}_G=\widetilde{\Phi}_G^{R}\widetilde{\Phi}_G$ on a set of compact generators
  (Proposition~\ref{prop:task4-monadic-reduction}).
  \item The base case of this unit calculation is controlled by the endomorphism algebra of the Whittaker
  generator via the stable center map (Conjecture~\ref{conj:task4-stable-center-iso}).
  \item Parabolic functoriality (Task~3) provides a mechanism to propagate the unit calculation to further
  generators, reducing full faithfulness to an inductive verification along Levi subgroups
  (Proposition~\ref{prop:task4-unit-CT}).
\end{itemize}

This completes the Task~4 blueprint.


%--------------------------------------------------------------------
\section{Essential surjectivity by gluing from Levi subgroups}\label{sec:task5}
%--------------------------------------------------------------------

In this section we address \textbf{Task 5} from Section~\ref{subsec:summary-dependencies}:
prove \emph{essential surjectivity} of the Whittaker functor by gluing from Levi subgroups using parabolic
functors and second adjointness.  This is the local analogue of the ``gluing'' step in geometric Langlands:
one reconstructs the whole category from Eisenstein series from proper Levi subgroups together with a cuspidal
part, and then argues by induction on semisimple rank.

We work at the level of presentable categories; at the end we explain how to pass to compact objects.

\subsection{The target statement and inductive framework}\label{subsec:task5-goal}

Assume Tasks~1--2, so that we have a colimit-preserving Whittaker functor
\[
  \widetilde{\Phi}_G\;:\;\IndCoh_{\mathrm{Nilp}}(\LocSys_{\checkG})
  \longrightarrow
  D_{\mathrm{lis}}(\Bun_G,\Lambda)_{\omega}.
\]
Assume Task~4, so that $\widetilde{\Phi}_G$ is fully faithful (for example by the monadic criteria of
Section~\ref{sec:task4}).

\begin{theorem}[Essential surjectivity template]\label{thm:task5-template}
Assume:
\begin{enumerate}[label=(\alph*), leftmargin=2em]
  \item for every proper Levi subgroup $M$ of $G$, the corresponding functor
  $\widetilde{\Phi}_M:\IndCoh_{\mathrm{Nilp}}(\LocSys_{\checkM})\to D_{\mathrm{lis}}(\Bun_M,\Lambda)_{\omega}$
  is an equivalence (induction hypothesis on semisimple rank),
  \item the functors $\widetilde{\Phi}_H$ are compatible with parabolic constant term and Eisenstein functors
  for all parabolics (Task~3), and
  \item $\widetilde{\Phi}_G$ induces an equivalence on the cuspidal subcategories (Task~6).
\end{enumerate}
Then $\widetilde{\Phi}_G$ is essentially surjective and hence an equivalence of presentable categories.
Consequently its restriction to compact objects
\[
  \Phi_G^{\mathrm{c}}:\Coh_{\mathrm{Nilp}}(\LocSys_{\checkG})\longrightarrow D_{\mathrm{lis}}(\Bun_G,\Lambda)_{\omega}^{\mathrm{c}}
\]
is an equivalence.
\end{theorem}

The rest of the section explains the gluing mechanism and proves Theorem~\ref{thm:task5-template} under the stated
assumptions.

\subsection{Cuspidal and Eisenstein subcategories on the automorphic side}\label{subsec:task5-auto}

Fix a proper parabolic subgroup $P\subsetneq G$ with Levi quotient $M$ and opposite parabolic $P^{-}$.
Let
\[
  \mathrm{CT}_{P}:\ D_{\mathrm{lis}}(\Bun_G,\Lambda)\to D_{\mathrm{lis}}(\Bun_M,\Lambda),
  \qquad
  \mathrm{Eis}_{P}:\ D_{\mathrm{lis}}(\Bun_M,\Lambda)\to D_{\mathrm{lis}}(\Bun_G,\Lambda)
\]
be the normalized constant term and Eisenstein series functors of Hamann--Hansen--Scholze
(Section~\ref{sec:parabolic}), so that $\mathrm{Eis}_{P}$ is right adjoint to $\mathrm{CT}_{P^{-}}$
(second adjointness, Theorem~\ref{thm:HHS-adjointness}).

\begin{definition}[Automorphic cuspidal subcategory inside the Whittaker category]\label{def:auto-cusp-omega}
Define the cuspidal subcategory of the Whittaker category by
\[
  D_{\mathrm{cusp}}(\Bun_G,\Lambda)_{\omega}
  \ :=\
  D_{\mathrm{lis}}(\Bun_G,\Lambda)_{\omega}
  \ \cap\
  \bigcap_{P\subsetneq G}\ker(\mathrm{CT}_{P}),
\]
where $P$ runs over all proper parabolic subgroups of $G$ (up to conjugacy).
\end{definition}

\begin{definition}[Automorphic Eisenstein-generated subcategory inside the Whittaker category]\label{def:auto-eis-omega}
Let $D_{\mathrm{Eis}}(\Bun_G,\Lambda)_{\omega}$ be the smallest full stable subcategory of
$D_{\mathrm{lis}}(\Bun_G,\Lambda)_{\omega}$ that is closed under colimits and contains, for every proper parabolic
$P\subsetneq G$ with Levi quotient $M$, the essential image of the restricted functor
\[
  \mathrm{Eis}_P:\ D_{\mathrm{lis}}(\Bun_M,\Lambda)_{\omega}\to D_{\mathrm{lis}}(\Bun_G,\Lambda)_{\omega}.
\]
\end{definition}

\begin{proposition}[Automorphic generation in the Whittaker category]\label{prop:task5-auto-generation}
The category $D_{\mathrm{lis}}(\Bun_G,\Lambda)_{\omega}$ is generated under colimits by
$D_{\mathrm{cusp}}(\Bun_G,\Lambda)_{\omega}$ and $D_{\mathrm{Eis}}(\Bun_G,\Lambda)_{\omega}$.
\end{proposition}

\begin{proof}[Proof sketch]
By Theorem~\ref{thm:HHS-generation}, the full category $D_{\mathrm{lis}}(\Bun_G,\Lambda)$ is generated under
colimits by the cuspidal subcategory and by Eisenstein series objects from proper Levi subgroups.
Intersecting with the colimit-closed subcategory $D_{\mathrm{lis}}(\Bun_G,\Lambda)_{\omega}$ gives the statement
provided one knows that:
\begin{itemize}[leftmargin=2em]
  \item $D_{\mathrm{lis}}(\Bun_G,\Lambda)_{\omega}$ is stable under constant term functors and Eisenstein functors
  from Levi Whittaker categories, and
  \item the cuspidal condition is detected by vanishing of all constant terms.
\end{itemize}
The first point follows from the constant term computation for the Whittaker generator
(Theorem~\ref{thm:CT-Whittaker} and Corollary~\ref{cor:CT-preserves-omega}) together with the fact that the
Whittaker category is stable under the spectral action and the parabolic functors are compatible with Hecke and
hence with the spectral action (Section~\ref{sec:parabolic}).
\end{proof}

\subsection{Cuspidal and Eisenstein subcategories on the spectral side}\label{subsec:task5-spectral}

Assume Task~3, so that spectral parabolic functors exist and preserve nilpotent singular support
(Theorem~\ref{thm:parabolic-preserve-nilp}).
For a parabolic $P\subset G$ with Levi $M$, write
\[
  \mathrm{CT}_{P}^{\mathrm{spec}}:\IndCoh_{\mathrm{Nilp}}(\LocSys_{\checkG})\to
  \IndCoh_{\mathrm{Nilp}}(\LocSys_{\checkM}),
  \qquad
  \mathrm{Eis}_{P}^{\mathrm{spec}}:\IndCoh_{\mathrm{Nilp}}(\LocSys_{\checkM})\to
  \IndCoh_{\mathrm{Nilp}}(\LocSys_{\checkG})
\]
for the spectral constant term and Eisenstein functors.

\begin{definition}[Spectral cuspidal subcategory]\label{def:spec-cusp}
Define the spectral cuspidal subcategory by
\[
  \IndCoh_{\mathrm{Nilp}}(\LocSys_{\checkG})_{\mathrm{cusp}}
  \ :=\
  \bigcap_{P\subsetneq G}\ker\big(\mathrm{CT}_{P}^{\mathrm{spec}}\big).
\]
\end{definition}

\begin{definition}[Spectral Eisenstein-generated subcategory]\label{def:spec-eis}
Let $\IndCoh_{\mathrm{Nilp}}(\LocSys_{\checkG})_{\mathrm{Eis}}$ be the smallest full stable subcategory of
$\IndCoh_{\mathrm{Nilp}}(\LocSys_{\checkG})$ that is closed under colimits and contains the essential images of
$\mathrm{Eis}_{P}^{\mathrm{spec}}$ for all proper parabolics $P\subsetneq G$.
\end{definition}

\begin{axiom}[Spectral generation by cuspidal and Eisenstein parts]\label{ax:spec-generation}
The category $\IndCoh_{\mathrm{Nilp}}(\LocSys_{\checkG})$ is generated under colimits by
$\IndCoh_{\mathrm{Nilp}}(\LocSys_{\checkG})_{\mathrm{cusp}}$ and
$\IndCoh_{\mathrm{Nilp}}(\LocSys_{\checkG})_{\mathrm{Eis}}$.
\end{axiom}

\begin{remark}
Axiom~\ref{ax:spec-generation} is the local analogue of the gluing statements in global geometric Langlands:
nilpotent singular support is expected to be the exact condition that makes parabolic functors behave well enough
for such a generation statement to hold, compare \cite{AGSingSupp} and the use of parabolic gluing in the proof
strategy of Gaitsgory--Raskin.
In a complete write-up, we expect to deduce Axiom~\ref{ax:spec-generation} from a more structural ``gluing''
theorem on the spectral side (for example using a stratification by Levi types).
\end{remark}

\subsection{Essential image and closure properties}\label{subsec:task5-image}

Let
\[
  \cI_G\ \subset\ D_{\mathrm{lis}}(\Bun_G,\Lambda)_{\omega}
\]
denote the essential image of $\widetilde{\Phi}_G$.

\begin{lemma}[The essential image is stable under colimits]\label{lem:image-colimits}
The essential image $\cI_G$ is closed under small colimits in $D_{\mathrm{lis}}(\Bun_G,\Lambda)_{\omega}$.
\end{lemma}

\begin{proof}
The functor $\widetilde{\Phi}_G$ preserves colimits, hence its essential image is closed under colimits.
\end{proof}

\begin{lemma}[The essential image contains Eisenstein series from Levi images]\label{lem:image-eisenstein}
Assume Task~3 (parabolic compatibility).
Let $P\subsetneq G$ be a parabolic with Levi quotient $M$.
Then
\[
  \mathrm{Eis}_{P}\big(\cI_M\big)\ \subset\ \cI_G,
\]
where $\cI_M$ is the essential image of $\widetilde{\Phi}_M$.
\end{lemma}

\begin{proof}
By Eisenstein compatibility (Conjecture~\ref{conj:parabolic-compat-full} or its proven form),
\[
  \mathrm{Eis}_{P}\circ \widetilde{\Phi}_M\ \simeq\ \widetilde{\Phi}_G\circ \mathrm{Eis}_{P}^{\mathrm{spec}}.
\]
Thus every object in $\mathrm{Eis}_P(\cI_M)$ lies in the essential image of $\widetilde{\Phi}_G$.
\end{proof}

\subsection{Induction on semisimple rank}\label{subsec:task5-induction}

Assume the induction hypothesis: for every proper Levi subgroup $M$ of $G$, the functor $\widetilde{\Phi}_M$ is an
equivalence.  Then $\cI_M=D_{\mathrm{lis}}(\Bun_M,\Lambda)_{\omega}$.

\begin{proposition}[The essential image contains the Eisenstein-generated subcategory]\label{prop:image-contains-Eis}
Assume Task~3 and the induction hypothesis for all proper Levi subgroups.
Then
\[
  D_{\mathrm{Eis}}(\Bun_G,\Lambda)_{\omega}\ \subset\ \cI_G.
\]
\end{proposition}

\begin{proof}
By Lemma~\ref{lem:image-eisenstein} and the induction hypothesis, for each proper parabolic $P\subsetneq G$ with
Levi quotient $M$ we have
\[
  \mathrm{Eis}_P\big(D_{\mathrm{lis}}(\Bun_M,\Lambda)_{\omega}\big)
  \ =\
  \mathrm{Eis}_P(\cI_M)
  \ \subset\
  \cI_G.
\]
Since $\cI_G$ is colimit-closed (Lemma~\ref{lem:image-colimits}), it contains the colimit-closure of these images,
which is exactly $D_{\mathrm{Eis}}(\Bun_G,\Lambda)_{\omega}$ by definition.
\end{proof}

\subsection{Reduction to the cuspidal case}\label{subsec:task5-cuspidal-reduction}

By Proposition~\ref{prop:task5-auto-generation}, the Whittaker category is generated under colimits by its
cuspidal subcategory and by the Eisenstein-generated subcategory.
The previous proposition handles the Eisenstein-generated part.
Thus essential surjectivity reduces to showing that cuspidal objects lie in the essential image.

\begin{proposition}[Essential surjectivity reduces to cuspidal objects]\label{prop:task5-reduce-to-cusp}
Assume:
\begin{enumerate}[label=(\alph*), leftmargin=2em]
  \item Task~3 and the induction hypothesis for all proper Levi subgroups, so that
  $D_{\mathrm{Eis}}(\Bun_G,\Lambda)_{\omega}\subset \cI_G$ (Proposition~\ref{prop:image-contains-Eis}),
  \item the essential image $\cI_G$ contains $D_{\mathrm{cusp}}(\Bun_G,\Lambda)_{\omega}$, and
  \item the essential image is closed under colimits (Lemma~\ref{lem:image-colimits}).
\end{enumerate}
Then $\widetilde{\Phi}_G$ is essentially surjective.
\end{proposition}

\begin{proof}
By (a), $\cI_G$ contains $D_{\mathrm{Eis}}(\Bun_G,\Lambda)_{\omega}$.
By (b), it contains the cuspidal subcategory.
By (c), it contains the colimit-closed stable subcategory generated by these two subcategories, which is all of
$D_{\mathrm{lis}}(\Bun_G,\Lambda)_{\omega}$ by Proposition~\ref{prop:task5-auto-generation}.
Thus $\cI_G=D_{\mathrm{lis}}(\Bun_G,\Lambda)_{\omega}$.
\end{proof}

\begin{remark}
Condition (b) is exactly the missing input from Task~6: an identification of the cuspidal subcategories on the
spectral and automorphic sides.  Once that is established, Task~5 completes essential surjectivity by induction.
\end{remark}

\subsection{Proof of Theorem~\ref{thm:task5-template}}\label{subsec:task5-proof-template}

\begin{proof}[Proof sketch]
Assume (a)--(c) of Theorem~\ref{thm:task5-template}.
By Task~4, $\widetilde{\Phi}_G$ is fully faithful.
By (a) and Task~3, Proposition~\ref{prop:image-contains-Eis} applies, so the essential image contains the
Eisenstein-generated subcategory.
By (c), the essential image also contains the cuspidal subcategory.
Therefore Proposition~\ref{prop:task5-reduce-to-cusp} implies essential surjectivity.
Hence $\widetilde{\Phi}_G$ is an equivalence of presentable categories.

Finally, since both sides are compactly generated and $\widetilde{\Phi}_G$ preserves compact objects (Task~1),
the induced functor on compact objects
\[
  \Phi_G^{\mathrm{c}}:\Coh_{\mathrm{Nilp}}(\LocSys_{\checkG})\to D_{\mathrm{lis}}(\Bun_G,\Lambda)_{\omega}^{\mathrm{c}}
\]
is also an equivalence.
\end{proof}

\subsection{What remains beyond Task 5}\label{subsec:task5-what-remains}

Task~5 reduces the global essential surjectivity problem to two remaining inputs:
\begin{itemize}[leftmargin=2em]
  \item a spectral generation statement (Axiom~\ref{ax:spec-generation}, or a theorem replacing it), and
  \item the cuspidal identification (Task~6), which is expected to be the local analogue of multiplicity one.
\end{itemize}
Once these are supplied, the combination of Tasks~1--5 yields the full equivalence predicted by
Conjecture~\ref{conj:categorical-geom-whittaker}.


%--------------------------------------------------------------------
\section{The cuspidal range and multiplicity one}\label{sec:task6}
%--------------------------------------------------------------------

This section addresses \textbf{Task 6} from Section~\ref{subsec:summary-dependencies}:
identify the \emph{cuspidal} subcategories on the spectral and automorphic sides.
This is the remaining input in Theorem~\ref{thm:task5-template} needed to finish essential surjectivity by
parabolic gluing (Task~5).

Conceptually, Task~6 is the local analogue of the ``multiplicity one'' step in the proof of geometric Langlands
\cite{GRProofV}: one must show that the cuspidal part is rigid enough that it is controlled by a finite amount of
stacky spectral data (component groups of centralizers), and that the chosen Whittaker datum singles out the
generic member in each packet.

Throughout we work with coefficients in a finite extension $\Lambda$ of $\mathbb{Q}_{\ell}$, where $\ell\neq p$.

%--------------------------------------------------------------------
\subsection{Cuspidal parameters and the cuspidal locus on the spectral stack}\label{subsec:task6-cuspidal-locus}
%--------------------------------------------------------------------

Let $\LocSys_{\checkG}$ denote the stack of Langlands parameters used throughout this paper (for example, the
Dat--Helm--Kurinczuk--Moss stack, equipped with a quasi-smooth derived enhancement as in
Section~\ref{sec:spectral}).

\begin{definition}[Cuspidal parameter]\label{def:task6-cuspidal-parameter}
A (geometric) point $\phi\in \LocSys_{\checkG}(\overline{\Lambda})$ is called \emph{cuspidal} if it does not
factor through any proper Levi subgroup of $\checkG$ (equivalently, it is not induced from a parameter for any
proper Levi).
\end{definition}

\begin{remark}
In classical local Langlands terminology, one often uses ``discrete'' or ``elliptic'' for parameters not
factoring through proper Levi subgroups.  We use the adjective ``cuspidal'' to emphasize its interaction with
vanishing of constant terms.
\end{remark}

\begin{definition}[Cuspidal locus]\label{def:task6-cuspidal-locus}
Let $\LocSys_{\checkG}^{\mathrm{cusp}}\subset \LocSys_{\checkG}$ be the full substack consisting of cuspidal
parameters.
\end{definition}

A defining expectation is that cuspidality on the spectral side is detected by vanishing of spectral constant
terms.

\begin{conjecture}[Cuspidality detected by spectral constant term]\label{conj:task6-cusp-detected}
Let $\mathcal{F}\in \IndCoh_{\mathrm{Nilp}}(\LocSys_{\checkG})$.
Then $\mathcal{F}$ belongs to the spectral cuspidal subcategory
\[
  \IndCoh_{\mathrm{Nilp}}(\LocSys_{\checkG})_{\mathrm{cusp}}
  \ :=\
  \bigcap_{P\subsetneq G}\ker(\mathrm{CT}_{P}^{\mathrm{spec}})
\]
if and only if $\mathcal{F}$ is set-theoretically supported on $\LocSys_{\checkG}^{\mathrm{cusp}}$.
\end{conjecture}

\begin{remark}
Conjecture~\ref{conj:task6-cusp-detected} is the spectral counterpart of the automorphic definition of cuspidal
objects by vanishing of constant terms (Definition~\ref{def:auto-cusp-omega}).
It is one place where the nilpotent singular support condition is expected to matter: it should prevent
pathological extensions supported on noncuspidal strata from surviving in the cuspidal intersection of kernels.
\end{remark}

%--------------------------------------------------------------------
\subsection{Residual gerbes and component groups}\label{subsec:task6-gerbes}
%--------------------------------------------------------------------

Let $\phi\in \LocSys_{\checkG}^{\mathrm{cusp}}(\overline{\Lambda})$ be a cuspidal parameter.
Write $Z_{\checkG}(\phi)$ for its centralizer in $\checkG$ and
\[
  S_{\phi}\ :=\ \pi_0\!\big(Z_{\checkG}(\phi)\big)
\]
for the component group.

Let $\mathcal{G}_{\phi}\hookrightarrow \LocSys_{\checkG}$ denote the residual gerbe at $\phi$.
In good situations (for example on a Deligne--Mumford locus) this residual gerbe is isomorphic to the classifying
stack $B Z_{\checkG}(\phi)$.

A particularly clean situation occurs when $Z_{\checkG}(\phi)$ is finite, or at least has finite component group
and trivial Lie algebra in the relevant directions.

\begin{definition}[Strongly cuspidal point]\label{def:task6-strongly-cuspidal}
A cuspidal point $\phi$ is called \emph{strongly cuspidal} if the residual gerbe $\mathcal{G}_{\phi}$ is
Deligne--Mumford and has finite automorphism group, so that
$\mathcal{G}_{\phi}\simeq B S_{\phi}$ for a finite group $S_{\phi}$.
\end{definition}

\begin{proposition}[Nilpotent singular support is automatic on a finite residual gerbe]\label{prop:task6-nilp-automatic}
Let $\phi$ be strongly cuspidal, so that $\mathcal{G}_{\phi}\simeq B S_{\phi}$ with $S_{\phi}$ finite.
Then the nilpotent singular support condition on $\IndCoh(\mathcal{G}_{\phi})$ is automatic:
\[
  \IndCoh_{\mathrm{Nilp}}(\mathcal{G}_{\phi})\ =\ \IndCoh(\mathcal{G}_{\phi}).
\]
Moreover, on compact objects one has a canonical identification
\[
  \Coh(\mathcal{G}_{\phi})\ \simeq\ \Rep_{\Lambda}(S_{\phi})^{\mathrm{fd}},
\]
and on ind-completions
\[
  \IndCoh(\mathcal{G}_{\phi})\ \simeq\ D\big(\Rep_{\Lambda}(S_{\phi})\big),
\]
the derived category of (possibly infinite-dimensional) $\Lambda$-representations of $S_{\phi}$.
\end{proposition}

\begin{proof}[Proof sketch]
If $\mathcal{G}_{\phi}\simeq B S_{\phi}$ with $S_{\phi}$ finite, then $\mathcal{G}_{\phi}$ is smooth of
dimension zero, so its singularity stack is the zero section.
Hence every ind-coherent sheaf has singular support contained in the zero section, which is nilpotent.
The identification $\Coh(BS_{\phi})\simeq \Rep_{\Lambda}(S_{\phi})^{\mathrm{fd}}$ is standard.
Ind-completing yields the derived representation category.
\end{proof}

\begin{remark}
Proposition~\ref{prop:task6-nilp-automatic} explains why the cuspidal range is expected to be the easiest
spectral region: the microlocal condition becomes trivial, and the entire subtlety is encoded by the
\emph{stackiness} (component groups).
\end{remark}

%--------------------------------------------------------------------
\subsection{Localizing the automorphic category at a cuspidal parameter}\label{subsec:task6-localize-auto}
%--------------------------------------------------------------------

Let $D_{\mathrm{lis}}(\Bun_G,\Lambda)_{\omega}$ be the Whittaker-generated automorphic category
(Definition~\ref{def:omega-subcategory}), equipped with the spectral action of
$\QCoh(\LocSys_{\checkG})$ (Proposition~\ref{prop:QCoh-action}).

Let $\phi$ be a strongly cuspidal point and let $i_{\phi}:\mathcal{G}_{\phi}\hookrightarrow \LocSys_{\checkG}$
be its residual gerbe.  Let $\OO_{\mathcal{G}_{\phi}}\in \QCoh(\LocSys_{\checkG})$ denote the extension by zero of
the structure sheaf of $\mathcal{G}_{\phi}$.

\begin{definition}[Cuspidal block of the Whittaker category]\label{def:task6-auto-block}
Define the \emph{$\phi$-block} of the Whittaker category to be the full subcategory
\[
  D_{\mathrm{lis}}(\Bun_G,\Lambda)_{\omega,\phi}
  \ :=\
  \OO_{\mathcal{G}_{\phi}}\ast D_{\mathrm{lis}}(\Bun_G,\Lambda)_{\omega}.
\]
Let
\[
  \mathcal{W}_{\phi}\ :=\ \OO_{\mathcal{G}_{\phi}}\ast \mathcal{W}_{G,\psi}
  \ \in\ D_{\mathrm{lis}}(\Bun_G,\Lambda)_{\omega,\phi}
\]
be the localized Whittaker generator.
\end{definition}

\begin{lemma}[Cuspidal blocks are cuspidal]\label{lem:task6-block-cuspidal}
Assume Task~3 (parabolic compatibility of constant term with the spectral action).
If $\phi$ is cuspidal, then every object of $D_{\mathrm{lis}}(\Bun_G,\Lambda)_{\omega,\phi}$ is cuspidal in the
sense of Definition~\ref{def:auto-cusp-omega}.
\end{lemma}

\begin{proof}[Proof sketch]
Let $P\subsetneq G$ be a proper parabolic with Levi quotient $M$.
By constant term compatibility on Whittaker translates (Corollary~\ref{cor:CT-on-translates}) one has, for any
$\mathcal{A}\in D_{\mathrm{lis}}(\Bun_G,\Lambda)_{\omega}$,
\[
  \mathrm{CT}_{P}\big(\OO_{\mathcal{G}_{\phi}}\ast \mathcal{A}\big)
  \ \simeq\
  \big((f_M^G)^{\ast}\OO_{\mathcal{G}_{\phi}}\big)\ast \mathrm{CT}_{P}(\mathcal{A}),
\]
where $f_M^G:\LocSys_{\checkG}\to \LocSys_{\checkM}$ is induced by $\checkM\hookrightarrow \checkG$.
If $\phi$ is cuspidal, the pullback $(f_M^G)^{\ast}\OO_{\mathcal{G}_{\phi}}$ is zero because $\phi$ does not
factor through $\checkM$.
Therefore $\mathrm{CT}_{P}$ vanishes on $\OO_{\mathcal{G}_{\phi}}\ast \mathcal{A}$ for every $\mathcal{A}$, as
claimed.
\end{proof}

\begin{remark}
Lemma~\ref{lem:task6-block-cuspidal} is the basic mechanism by which cuspidal spectral support forces cuspidality
on the automorphic side.
It is the local analogue of the idea that cuspidal Hecke eigensheaves have vanishing constant terms.
\end{remark}

%--------------------------------------------------------------------
\subsection{A geometric multiplicity one statement}\label{subsec:task6-multone}
%--------------------------------------------------------------------

In the classical representation theory of $p$-adic groups, a generic irreducible representation admits a
Whittaker model and the space of Whittaker functionals is one-dimensional.
In our categorical setting, the correct replacement is a statement about maps out of the Whittaker generator.

\begin{definition}[Whittaker coefficient functor]\label{def:task6-Whittaker-coeff}
Define a functor
\[
  \mathrm{Wh}_{\psi}\;:\; D_{\mathrm{lis}}(\Bun_G,\Lambda)_{\omega}\longrightarrow D(\Lambda)
\]
by
\[
  \mathrm{Wh}_{\psi}(\mathcal{A})\ :=\ \mathrm{RHom}_{D_{\mathrm{lis}}(\Bun_G,\Lambda)}(\mathcal{W}_{G,\psi},\mathcal{A}).
\]
\end{definition}

By the adjunction in Task~4 (equation \eqref{eq:task4-evaluate-at-unit}), $\mathrm{Wh}_{\psi}$ factors through the
right adjoint $\widetilde{\Phi}_G^{R}$:
\[
  \mathrm{Wh}_{\psi}(\mathcal{A})
  \ \simeq\
  \mathrm{RHom}_{\IndCoh_{\mathrm{Nilp}}(\LocSys_{\checkG})}\big(\OO_{\LocSys_{\checkG}},
  \widetilde{\Phi}_G^{R}(\mathcal{A})\big).
\]
On the $\phi$-block, one expects $\widetilde{\Phi}_G^{R}$ to land in $\IndCoh(\mathcal{G}_{\phi})\simeq
D(\Rep_{\Lambda}(S_{\phi}))$, so the Whittaker coefficient is expected to compute invariants under $S_{\phi}$.

\begin{conjecture}[Geometric multiplicity one in the cuspidal range]\label{conj:task6-multone}
Let $\phi$ be strongly cuspidal and let
$\mathcal{A}\in D_{\mathrm{lis}}(\Bun_G,\Lambda)_{\omega,\phi}$ be an object corresponding under the expected
cuspidal equivalence to an irreducible representation $\rho$ of $S_{\phi}$.
Then
\[
  H^0\big(\mathrm{Wh}_{\psi}(\mathcal{A})\big)\ \cong\
  \begin{cases}
    \Lambda & \text{if $\rho$ is the trivial representation,}\\
    0 & \text{otherwise,}
  \end{cases}
\]
and higher cohomology groups of $\mathrm{Wh}_{\psi}(\mathcal{A})$ vanish.
\end{conjecture}

\begin{remark}
Conjecture~\ref{conj:task6-multone} is the categorical incarnation of:
\begin{itemize}[leftmargin=2em]
  \item uniqueness of Whittaker models, and
  \item the Whittaker normalization of local Langlands, which predicts that the generic member of an
  $L$-packet corresponds to the trivial representation of the component group.
\end{itemize}
It is also the precise input needed to identify the \emph{cuspidal block} as the regular module category for
$\Rep_{\Lambda}(S_{\phi})$.
\end{remark}

%--------------------------------------------------------------------
\subsection{Cuspidal block equivalence and its role in Task~5}\label{subsec:task6-block-equivalence}
%--------------------------------------------------------------------

We now state the desired cuspidal identification that feeds into Task~5.

\begin{conjecture}[Cuspidal block equivalence]\label{conj:task6-cuspidal-block}
Let $\phi$ be strongly cuspidal with residual gerbe $\mathcal{G}_{\phi}\simeq B S_{\phi}$.
Then the Whittaker functor induces an equivalence of presentable stable $\infty$-categories
\[
  \IndCoh(\mathcal{G}_{\phi})
  \ \xrightarrow{\ \sim\ }\
  D_{\mathrm{lis}}(\Bun_G,\Lambda)_{\omega,\phi},
\]
compatible with the $\QCoh(\mathcal{G}_{\phi})\simeq \Rep_{\Lambda}(S_{\phi})$-module structures.
Under this equivalence, the unit object $\OO_{\mathcal{G}_{\phi}}$ corresponds to the localized Whittaker
generator $\mathcal{W}_{\phi}$.
\end{conjecture}

\begin{proposition}[Task~6 implies the cuspidal input in Task~5]\label{prop:task6-implies-task5}
Assume Conjecture~\ref{conj:task6-cuspidal-block} holds for all strongly cuspidal points $\phi$ and that every
cuspidal object in $D_{\mathrm{lis}}(\Bun_G,\Lambda)_{\omega}$ is a colimit of objects lying in cuspidal blocks.
Then the induced functor
\[
  \widetilde{\Phi}_G\;:\;\IndCoh_{\mathrm{Nilp}}(\LocSys_{\checkG})_{\mathrm{cusp}}
  \longrightarrow
  D_{\mathrm{cusp}}(\Bun_G,\Lambda)_{\omega}
\]
is essentially surjective and fully faithful, hence an equivalence.
\end{proposition}

\begin{proof}[Proof sketch]
By definition, the spectral cuspidal category is generated under colimits by objects supported on cuspidal
residual gerbes (compare Conjecture~\ref{conj:task6-cusp-detected}).
On each gerbe $\mathcal{G}_{\phi}$, Conjecture~\ref{conj:task6-cuspidal-block} identifies the image of the
restriction of $\widetilde{\Phi}_G$ with the full block
$D_{\mathrm{lis}}(\Bun_G,\Lambda)_{\omega,\phi}$, which lies in the automorphic cuspidal subcategory by
Lemma~\ref{lem:task6-block-cuspidal}.
Assuming that cuspidal objects decompose into such blocks, this proves essential surjectivity on the cuspidal
subcategory.  Full faithfulness follows from Task~4.
\end{proof}

\begin{remark}
Proposition~\ref{prop:task6-implies-task5} is the precise bridge between Task~6 and Task~5:
once the cuspidal blocks are identified, the Levi-gluing mechanism in Section~\ref{sec:task5} finishes the
full equivalence.
\end{remark}

%--------------------------------------------------------------------
\subsection{Sanity checks: tori and general linear groups}\label{subsec:task6-examples}
%--------------------------------------------------------------------

\begin{remark}[Tori]
If $G=T$ is a torus, there are no proper parabolic subgroups, so every object is cuspidal.
The parameter stack $\LocSys_{\checkT}$ is (up to mild stackiness coming from automorphisms) an algebraic torus,
and the nilpotent singular support condition is the zero-section condition.
In this case, Task~6 is essentially the entire conjecture, and the expected equivalence reduces to the
description of sheaves on $\Bun_T$ in terms of characters, which is compatible with the abelian local Langlands
correspondence.
\end{remark}

\begin{remark}[$G=\GL_n$]
For $G=\GL_n$, the component groups $S_{\phi}$ for cuspidal parameters are expected to be trivial.
Thus Conjecture~\ref{conj:task6-cuspidal-block} predicts that each cuspidal block is equivalent to the derived
category of $\Lambda$-vector spaces.  In other words, the cuspidal block should be generated by a single object
(the cuspidal Hecke eigensheaf), and geometric multiplicity one becomes the assertion that its Whittaker
coefficient is one-dimensional.
This is the categorical avatar of the fact that a discrete series $L$-packet for $\GL_n$ is a singleton.
\end{remark}


\bibliographystyle{alpha}
\begin{thebibliography}{99}

\bibitem{FSGeometrization}
L.~Fargues and P.~Scholze.
\newblock \emph{Geometrization of the local Langlands correspondence}.
\newblock \href{https://arxiv.org/abs/2102.13459}{arXiv:2102.13459}.

\bibitem{FSReview}
L.~Fargues and P.~Scholze.
\newblock \emph{The Langlands program and the moduli of bundles on the curve}.
\newblock IH\'ES Summer School notes (2022).
\newblock \href{https://webusers.imj-prg.fr/~laurent.fargues/IHES_summer_school_2022.pdf}{PDF}.

\bibitem{DHKMParameters}
J.-F.~Dat, D.~Helm, R.~Kurinczuk, and G.~Moss.
\newblock \emph{Moduli of Langlands parameters}.
\newblock J. Eur. Math. Soc. 27 (2025), 1827--1927.
\newblock \href{https://ems.press/journals/jems/articles/14298647}{EMS Press page}.

\bibitem{HHSGeometricEis}
L.~Hamann, D.~Hansen, and P.~Scholze.
\newblock \emph{Geometric Eisenstein series I: finiteness theorems}.
\newblock \href{https://arxiv.org/abs/2409.07363}{arXiv:2409.07363}.

\bibitem{MilesGluingHN}
J.~Miles.
\newblock \emph{Gluing sheaves along Harder--Narasimhan strata of $\Bun_G$}.
\newblock \href{https://arxiv.org/abs/2511.11327}{arXiv:2511.11327}.

\bibitem{AGSingSupp}
D.~Arinkin and D.~Gaitsgory.
\newblock \emph{Singular support of coherent sheaves, and the geometric Langlands conjecture}.
\newblock \href{https://arxiv.org/abs/1201.6343}{arXiv:1201.6343}.

\bibitem{GRProofI}
D.~Gaitsgory and S.~Raskin.
\newblock \emph{Proof of the geometric Langlands conjecture I: construction of the functor}.
\newblock \href{https://arxiv.org/abs/2405.03599}{arXiv:2405.03599}.

\bibitem{GRProofIV}
D.~Arinkin, D.~Beraldo, L.~Chen, J.~F{\ae}rgeman, D.~Gaitsgory, K.~Lin, S.~Raskin, and N.~Rozenblyum.
\newblock \emph{Proof of the geometric Langlands conjecture IV: ambidexterity}.
\newblock \href{https://arxiv.org/abs/2409.08670}{arXiv:2409.08670}.

\bibitem{GRProofV}
D.~Gaitsgory and S.~Raskin.
\newblock \emph{Proof of the geometric Langlands conjecture V: the multiplicity one theorem}.
\newblock \href{https://arxiv.org/abs/2409.09856}{arXiv:2409.09856}.

% New bibitems needed for Section \ref{sec:CT-Whittaker}

\bibitem{BushnellHenniartGW}
C.~J.~Bushnell and G.~Henniart.
\newblock \emph{Generalized Whittaker models and the Bernstein center}.
\newblock American Journal of Mathematics \textbf{125} (2003), no.~3, 513--547.

\bibitem{MatringeGWJ}
N.~Matringe.
\newblock \emph{Generalized Whittaker functions and Jacquet modules}.
\newblock \href{https://arxiv.org/abs/2009.01624}{arXiv:2009.01624}.

% New bibitems for Section \ref{sec:endo-Whittaker}

\bibitem{HelmWhittakerBernstein}
D.~Helm.
\newblock \emph{Whittaker models and the integral Bernstein center for $\GL_n$}.
\newblock Duke Mathematical Journal \textbf{165} (2016), no.~9, 1597--1628.
\newblock \href{https://arxiv.org/abs/1210.1789}{arXiv:1210.1789}.

% New bibitem referenced in Section \ref{sec:task2} (if not already present)

\bibitem{AGKRRVRestrictedVariation}
D.~Arinkin, D.~Gaitsgory, D.~Kazhdan, S.~Raskin, N.~Rozenblyum, and Y.~Varshavsky.
\newblock \emph{The stack of local systems with restricted variation and geometric Langlands theory with nilpotent singular support}.
\newblock \href{https://arxiv.org/abs/2010.01906}{arXiv:2010.01906}.

% ============================================================
% PATCH 3 (new bibitems to add — “current + missing infrastructure”)
% Add these to your bibliography; then cite them in the obvious spots:
%   - diamonds/sheaf theory: Scholze diamonds
%   - stable Bernstein center background: Haines, BKV, Varma
%   - GL_n in families / block-parameter comparison: Emerton–Helm, Helm, Helm–Moss
%   - FS compatibility check example: Hamann (Gan–Takeda)
%   - averaging functors (often cited in FS ecosystem): Anschütz–Le Bras
% ============================================================

\bibitem{ScholzeDiamonds}
P.~Scholze.
\newblock \emph{\'Etale cohomology of diamonds}.
\newblock \href{https://arxiv.org/abs/1709.07343}{arXiv:1709.07343}.

\bibitem{AnschutzLeBrasAveraging}
J.~Ansch\"utz and A.~Le~Bras.
\newblock \emph{Averaging functors in the Fargues--Scholze program}.
\newblock \href{https://arxiv.org/abs/2104.04701}{arXiv:2104.04701}.

\bibitem{HainesStableCenter}
T.~Haines.
\newblock \emph{The stable Bernstein center and test functions for Shimura varieties}.
\newblock In \emph{Automorphic Forms and Galois Representations, Vol.\ 2},
London Math.\ Soc.\ Lecture Note Ser.\ \textbf{415}, Cambridge Univ.\ Press, 2014.

\bibitem{BKVStableCenter}
R.~Bezrukavnikov, D.~Kazhdan, and Y.~Varshavsky.
\newblock \emph{A categorical approach to the stable center conjecture}.
\newblock Ast\'erisque \textbf{369} (2015), 27--97.

\bibitem{VarmaStableCenterNotes}
S.~Varma.
\newblock \emph{Some comments on the stable Bernstein center}.
\newblock \href{https://mathweb.tifr.res.in/~sandeepv/stable_center_classical_groups.pdf}{PDF}.

\bibitem{EmertonHelmFamilies}
M.~Emerton and D.~Helm.
\newblock \emph{The local Langlands correspondence for $\mathrm{GL}_n$ in families}.
\newblock Ann.\ Sci.\ \'Ec.\ Norm.\ Sup\'er.\ (4) \textbf{47} (2014), no.~4, 655--722.
\newblock \href{https://arxiv.org/abs/1104.0322}{arXiv:1104.0322}.

\bibitem{HelmBernsteinCenterWk}
D.~Helm.
\newblock \emph{On the Bernstein center of the category of smooth $W(k)[\mathrm{GL}_n(F)]$-modules}.
\newblock \href{https://arxiv.org/abs/1201.1874}{arXiv:1201.1874}.

\bibitem{HelmMossFamilies}
D.~Helm and G.~Moss.
\newblock \emph{Converse theorems and the local Langlands correspondence in families}.
\newblock \href{https://arxiv.org/abs/1801.01862}{arXiv:1801.01862}.

\bibitem{HamannGanTakeda}
L.~Hamann.
\newblock \emph{The Fargues--Scholze correspondence for $\mathrm{GSp}_4$ and the Gan--Takeda local Langlands correspondence}.
\newblock \href{https://arxiv.org/abs/2501.13370}{arXiv:2501.13370}.


\end{thebibliography}

\end{document}
